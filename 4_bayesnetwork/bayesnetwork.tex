% Thesis Spring Bloom Chapter
% Michael Malick
% 2015-10-12

\chapter[Environmental Pathways and Salmon Recruitment]{Accounting for multiple
pathways in the connections among climate variability, ocean processes, and coho
salmon recruitment in the Northern California Current\footnotemark[1]}

\footnotetext[1]{A version of this chapter appears as 
  Malick, M.J., S.P. Cox, R.M. Peterman, T.C. Wainwright, and W.T. Peterson.
  2015. Accounting for multiple pathways in the connections among climate
  variability, ocean processes, and coho salmon recruitment in the Northern
  California Current. Canadian Journal of Fisheries and Aquatic Sciences
  72:1552-1564. \url{http://doi.org/10.1139/cjfas-2014-0509}.}


\section{Abstract}

Pathways linking climate to population dynamics of higher-trophic-level fish
species such as Pacific salmon often involve a hierarchy in which regional-scale
physical and biological processes mediate the effects of large-scale climate
variability. We used probabilistic networks to investigate 17 potential
ecological pathways linking climate to Oregon coho salmon recruitment. We found
that pathways originating with the Pacific Decadal Oscillation were the most
influential on recruitment with the net effect being 2 to 4 times greater than
for pathways originating with the North Pacific Gyre Oscillation or Oceanic
Ni\~{n}o
Index. Among all environmental variables, sea surface temperature and an index
of juvenile salmon prey biomass had the greatest effects on recruitment with a
76\% chance of recruitment being equal to or below average given that ocean
temperatures were above average and a 34\% chance of recruitment being below
average given that prey biomass was above average. Our results provide evidence
that shifts in climate patterns could strongly influence recruitment
simultaneously through multiple ecological pathways and highlight the importance
of quantifying cumulative effects of these pathways on higher-trophic-level
species.



\section{Introduction}

Pacific salmon (\emph{Oncorhynchus} spp.) populations along the Northeast
Pacific coast exhibit large inter-annual and inter-decadal fluctuations in adult
abundances. Changes in large-scale climate patterns are often associated with
variability in salmon recruitment, although there are many intermediate-scale
processes that can link climate and salmon \citep{Mueter2002a, Beamish2004b,
Drinkwater2010a, Malick2015a}. In particular, several regional-scale
oceanographic variables are associated with both large-scale climate patterns
and salmon recruitment, including sea surface temperature (SST), upwelling
intensity, and ocean transport \citep{King2011, Chavez2003a, Keister2011a}.
However, most research on relationships between climate variability and salmon
recruitment simplify the ecological system by considering only direct effects of
climate on recruitment (Fig. \ref{fig:bn:1}a). For instance, multiple studies show
correlations between the Pacific Decadal Oscillation (PDO) and indices of salmon
survival \citep{Mantua1997a, Burke2013, Malick2009a} without further
investigating possible pathways of bottom-up or top-down processes linking the
two.

Pathways linking climate to the dynamics of higher-trophic-level fish species
such as salmon often involve a hierarchy in which regional-scale physical and
biological processes mediate the effects of large-scale climate variability
(Fig. \ref{fig:bn:1}b) \citep{Drinkwater2010a, Ottersen2010a, Dippner2006}. For example,
there are at least two hypothesized pathways connecting the PDO with salmon
recruitment in the Northern California Current \citep{Wells2008a, Keister2011a}.
Under the first hypothesis, regional-scale SST and juvenile salmon prey biomass
act as intermediaries between the PDO and recruitment \citep{Daly2013,
Cole2000a}, whereas under the second hypothesis, regional-scale ocean transport
and copepod community composition act as intermediaries \citep{Bi2011a,
Keister2011a}. These hypothesized pathways include processes that occur at
several temporal, spatial, and functional scales, and therefore, represent the
ecological system more realistically than assuming direct relationships between
climate patterns and salmon recruitment \citep{Levin1992a, Ottersen2010a,
Bakun1996a, Hunt2002a}.

Despite the more intuitive appeal of the hierarchical pathway perspective on
relationships between climate and salmon recruitment, it remains incomplete
because it assumes a stationary ecosystem structure.  Abrupt or persistent
changes in climate patterns can substantially alter physical and biological
processes in coastal ecosystems, potentially influencing high-trophic level
species through numerous ecological pathways (Fig. \ref{fig:bn:1}c) \citep{Anderson1999a,
Mantua1997a}. This implies a more complex hierarchy in which the relative
strengths of alternative pathways may change over time. However, there has been
little research on the relative importance of particular pathways on salmon
recruitment or on the joint effect of multiple pathways linking climate to fish
recruitment in general.

In this study, we investigate how multiple ecological pathways potentially link
climate and oceanographic processes to wild Oregon coho salmon (\emph{O.
kisutch}) recruitment. Specifically, we developed two probabilistic network
models, similar to Fig. \ref{fig:bn:1}c, to determine the joint effect of multiple ecological
pathways on coho salmon recruitment as well as the relative strength of specific
pathways. In addition, we investigated two time periods to determine whether the
dominant pathways changed over time. Our use of probabilistic networks allowed
us to (1) clearly and intuitively model recruitment as a function of multiple
ecological pathways, (2) quantify the effects of both direct and indirect
effects of environmental variables on salmon recruitment, and (3) account for
uncertainty in the relationships among variables by describing the relationships
probabilistically rather than deterministically \citep{Varis1995a}. We also
identify important environmental variables that could be used as indicators of
salmon recruitment and contribute to the understanding of the mechanisms that
control recruitment of Pacific salmon in the Northern California Current region.



\section{Methods}

\subsection{Overview}

We used data for nine environmental variables to estimate the relative strength
and net effects of 17 ecological pathways on recruitment of wild Oregon coho
salmon. The pathways were organized into two independent probabilistic networks
(e.g., Fig. \ref{fig:bn:1}c), a physical network and a biophysical network, which we used to
perform two analyses. First, to determine the relative strength of each of the
17 pathways within the networks, we used partial correlation coefficients to
estimate the strength of each link in the networks. Second, to quantify the
joint effect of multiple pathways on coho salmon recruitment, we used fitted
probabilistic networks along with Monte Carlo sampling to estimate conditional
posterior probability distributions for various levels of coho salmon
recruitment, given several scenarios (i.e., sets of conditions) for the
environmental variables.


\subsection{Data sources}

\subsubsection{Coho salmon recruitment}

Oregon's wild coho salmon populations are divided into three discrete
evolutionarily significant units (ESU) \citep{Weitkamp1995a, Lawson2007a}. The
focus of our study is on the Oregon Coast ESU, the largest one, which extends
from the mouth of the Columbia River south to Cape Blanco (Fig. \ref{fig:bn:2}). It contains
21 independent coho salmon populations (i.e., populations that were historically
self-supporting) located in several different river basins \citep{Lawson2007a}.
Oregon Coast coho salmon rear mainly in coastal streams and rivers, but some
populations rear primarily in coastal lakes and have a distinct life history and
different population dynamics than the river populations \citep{Lawson2004,
PFMC2013}. Because of this, we restricted our analysis to the river populations
only. In the past, there was also substantial production of hatchery coho salmon
on the Oregon coast -- we have excluded this production from our analysis, and
concentrate on just the wild production.

Annual aggregate adult recruitment and escapement estimates for the wild river
coho salmon populations within the Oregon Coast ESU were available for brood
years 1968-2009 from the Pacific Fisheries Management Council \citep{PFMC2013,
Rupp2012}. Recruitment estimates were generated from adult escapement and
harvest rate estimates, where escapement was estimated through statistical
expansion of survey counts in a subset of stream reaches within the Oregon Coast
ESU \citep{Lewis2010}. For return years 1971 through 1990, adult escapements
were monitored using spawner surveys on standard index areas along the Oregon
coast. Since 1990 a stratified random sampling design has been implemented,
which covers all spawning habitats within the Oregon Coast ESU
\citep{Jacobs1998, Lewis2010}. Because spawner survey methods prior to 1990 did
not allow reliable reconstruction of population-specific abundance, we used
aggregate recruitment data across all coho salmon populations in the Oregon
Coast ESU (Fig. \ref{fig:bn:3}), which is consistent with the current pre-season forecasting
methods used by the Pacific Fisheries Management Council \citep{PFMC2013}.

We chose to focus on wild coho salmon production instead of hatchery production
because of a potential mismatch between the biological data used in this study
and the geographic location of coho salmon hatcheries. The most widely used and
reliable source of hatchery data in the Northern California Current region is
the Oregon Production Index \citep{Logerwell2003a, Koslow2002a, Cole2000a},
which is largely composed of data for Columbia River hatcheries (90\% Columbia
River fish since 1991) \citep{PFMC2013}. Because the Columbia River is
approximately 425 km north of the sampling locations used to produce the
biological data set, the biological variables may not be representative of early
ocean conditions of coho salmon entering the ocean from the Columbia River. In
addition, Columbia River fish enter the ocean in the Columbia River plume, which
can have different dynamics than other coastal areas due to the large freshwater
influence \citep{Hickey1998}.

While our main focus was on total recruitment, we also investigated an index of
coho salmon productivity. To create the productivity index, we fit a
Beverton-Holt model (log\textsubscript{e}(R/S) = log\textsubscript{e}(a) -
log\textsubscript{e}(1 + bS)) \citep{Beverton1957a} to the spawner-recruitment
time series and then calculated the residuals. We used this residual series as
our productivity index, which describes inter-annual variability in productivity
(in units of log\textsubscript{e}(R/S)) after accounting for density-dependent
effects of spawner abundance (Supporting materials Fig. \ref{fig:bn:s1}).


\subsubsection{Environmental variables}

Nine environmental variables were included in the probabilistic networks (Table
\ref{tab:bn:1}; footnotes in that table indicate the data sources). Three of the variables
represent large-scale climate patterns, which reflect variability over thousands
of kilometers \citep{King2011}. The first, PDO, is defined as the leading
principle component of monthly SST anomalies in the North Pacific poleward of
20$^{\circ}$N \citep{Mantua1997a}.  Second, the North Pacific Gyre Oscillation (NPGO) is
defined as the second principle component of monthly sea-surface-height
anomalies in the North Pacific and represents variability that is orthogonal to
the PDO over the period 1950--2010 \citep{Di-Lorenzo2008a}. Third, we used the
Oceanic Ni\~{n}o Index (ONI) to index variability associated with El Ni\~{n}o and La
Ni\~{n}a events; it is defined as the 3-month running average of SST anomalies in
the Ni\~{n}o 3.4 region (120$^{\circ}$W-170$^{\circ}$W and 5$^{\circ}$S-5$^{\circ}$N) \citep{Trenberth1997}. Unlike the
PDO and NPGO, which have most of their variance at decadal and inter-decadal
periods, the ONI has most of its variance at inter-annual time scales
\citep{Sarachik2010a}. Because large-scale climate variables are believed to set
the stage for regional-scale physical and biological processes, each of the
three large-scale variables was averaged over the months of December-March in
the winter prior to smolt out-migration \citep{Mantua1997a, Yeh2011,
DiLorenzo2013a}.

Four of the environmental variables represent physical oceanographic variability
on a smaller, regional scale. First, we used monthly National Oceanic and
Atmospheric Administration extended reconstructed SST version 3b data to index
regional-scale variability in SST off the coast of Oregon \citep{Smith2008a}.
Monthly SST values were averaged over January-June for a 2$^{\circ}$x2$^{\circ}$ grid cell
centered on 44$^{\circ}$N 126$^{\circ}$W. The January-June period was chosen because research has
suggested that coastal SST can strongly influence salmon survival at time
periods just prior to and during smolt out-migration \citep{Mueter2005a}.

Second, we used the Bakun upwelling index to represent inter-annual variability
in upwelling intensity, where intensity was quantified as the volume of surface
water transported offshore caused by geostrophic wind fields \citep{Bakun1973,
Schwing1996}. Daily values for the upwelling index were available for 1970-2011
for the 45$^{\circ}$N 125$^{\circ}$W station.  We averaged the upwelling index over March-April to
represent ocean conditions just-prior to the spring transition and smolt
out-migration \citep{Logerwell2003a, Lawson1997}. Third, to index inter-annual
variability in the start date of the upwelling season, we used the Bakun
upwelling index to calculate the spring transition date as the day of the year
corresponding to the minimum value of the cumulative upwelling index
\citep{Bakun1973, Bograd2009}, where the cumulative upwelling index was
calculated by taking the daily cumulative sum of the Bakun upwelling index
starting on January 1 of each year. Fourth, we used deep water temperature
(i.e., temperature at 50m depth) at a station five miles off the coast of
Newport, Oregon to index inter-annual variability of the source of waters which
upwell along the Oregon coast. This source water is thought to be primarily
influenced by wind intensity and large-scale climate patterns such as the PDO
and NPGO \citep{Jacox2014, Chhak2007}. In particular, when northerly winds are
strong, water from a deeper (and thus colder) offshore source upwells onto the
shelf; when winds are weak, waters upwell from a shallower (thus warmer) source.

The two remaining environmental variables represent inter-annual variability in
regional-scale biological (rather than physical) processes. First, to index prey
availability of juvenile fish available to coho salmon during their first summer
at sea, we used the average biomass (mg carbon / 1000 m\textsuperscript{3}) of
those ichthyoplankton species that in January-March will develop into the
individuals that the coho salmon will eat in summer (primarily sand lance and
osmerids).  Sampling of ichthyoplankton occurred from 1998-2011 (Table \ref{tab:bn:1}). All
fish larvae were identified to the species level and a subset of lengths were
taken for each species. Length--to--biomass conversions were made using
published values, and total biomass at each station was estimated
\citep{Peterson2012a}. Details of the ichthyoplankton sampling procedures can be
found in \citet{Daly2013}. Although our ichthyoplankton biomass variable indexes
prey resource availability prior to smolt out-migration (i.e., January-March),
previous research has indicated that ichthyoplankton biomass during this period
is correlated with coho salmon survival \citep{Daly2013}.

Second, to index the quality of food (rather than the quantity) available to
coho salmon during their early marine residency, we used the average
May-September log\textsubscript{10} biomass (mg carbon / m\textsuperscript{3})
of three primary copepod species, \emph{Pseudocalanus mimus}, \emph{Acartia
longiremis}, and \emph{Calanus marshallae}. These copepod species are associated
with northern water sources and generally have a higher lipid content than
copepod species characteristic of other water sources off the coast of Oregon
\citep{Lee2006, Hooff2006a}. Copepods were sampled biweekly from 1998-2011
during May-September at the NH05 station along the Newport Hydrographic Line.
Details of the copepod sampling procedures can be found in \citet{Lamb2005a},
\citet{Peterson2003a}, and \citet{Bi2011b}.  We used a May-September average to
represent conditions experienced by coho salmon during and just after smolt
out-migration \citep{Bi2011a}.

In general, all environmental variables were averaged over a temporal period
corresponding to either the winter prior to ocean entry or the first summer the
coho salmon were in the ocean to represent conditions coho salmon experience
during the first ocean summer (Table \ref{tab:bn:1}). Unless stated otherwise, all reported
years correspond to the ocean entry year for the coho salmon cohort.


\subsection{Probabilistic networks}

Probabilistic networks are a class of graphical models that permit the explicit
and intuitive modeling of ecological networks while also taking uncertainties
into account explicitly \citep{Pearl1988a, Varis1995a}. A complete probabilistic
network is composed of three parts: (1) a set of variables, (2) a network
structure in the form of a directed acyclic graph, and (3) a set of local
probability distributions associated with each variable \citep{Heckerman1996a}.
These three components of a probabilistic network produce a joint probability
distribution over all variables in a network (also known as the global
distribution for the network).

Our probabilistic network analysis consisted of four steps. First, we
constructed two directed acyclic graphs, which represented alternative network
structures, using the nine environmental variables. Second, we estimated the
strength of each link and pathway in the networks using partial correlation
coefficients. Third, we fit the probabilistic networks by estimating the
parameters of the local probability distributions for each network. Fourth, we
used the fitted networks to estimate conditional posterior probability
distributions for recruitment given various scenarios for the environmental
variables.


\subsubsection{Network structures}

We used the nine environmental variables to construct two probabilistic network
structures (Figs. \ref{fig:bn:4} and \ref{fig:bn:5}) that represented the hypothesized structure of the
ecological system. Both network structures took the form of directed acyclic
graphs, meaning neither network contained feedback loops. Within the networks,
ovals represent variables and arrows connecting variables indicate dependencies
among the variables.  The networks in this study contain three types of
variables (1) variables with no incoming arrows (root variables), (2) variables
with incoming and outgoing arrows (intermediate variables), and (3) variables
with only incoming arrows (in our networks recruitment was the only variable
with no outgoing arrows). Because intermediate variables can be both dependent
and independent variables within the network, we refer to variables at the base
of an arrow as parent variables and variables at the tip of an arrow head as
child variables, as is the convention for such analyses of probabilistic network
models \citep{Koller2009a, Korb2004a}.

The first network structure was a physical network based on only physical
environmental variables for coho salmon ocean entry years 1970-2011. The
physical network structure included 7 variables, 10 links among the variables,
and 9 pathways connecting large-scale climate variables with recruitment (Fig.
4). The second network structure was a biophysical network that combined
physical and biological environmental variables for ocean entry years 1998-2011.
The biophysical network structure had 10 variables, 13 links, and 8 pathways
connecting climate variables and recruitment (Fig. \ref{fig:bn:5}).

Both the physical and biophysical network structures were organized in a
spatial, temporal, and functional manner to represent bottom-up forcing on coho
salmon recruitment. For example, large-scale climate and oceanographic patterns
of variability were designated as root variables and were averaged over the
winter months prior to smolt outmigration (Table \ref{tab:bn:1}). These large-scale variables
were independent of each other in the networks (as indicated by the absence of
arrows connecting these variables in the networks) and directly influenced a set
of regional-scale physical oceanographic variables such as SST that represented
variability in the late winter and spring (Table \ref{tab:bn:1}). In the physical network,
SST was directly linked to coho salmon recruitment (Fig. \ref{fig:bn:4}), whereas in the
biophysical network the regional-scale physical variables directly influenced a
set of regional-scale biological variables (e.g., copepod biomass), which were
then directly connected to coho salmon recruitment (Fig. \ref{fig:bn:5}).


\subsubsection{Pathway and link strength}

We used partial correlation coefficients to quantify the strength of each link
in the network graphs \citep{Zar1999a, Scutari2010, Yang2011}.  Coefficients
were computed for each link in a network by correlating two variables connected
by an arrow while accounting for the effects of other variables that had
incoming arrows to the child variable of the arrow of interest. For example, the
partial correlation coefficient for the link connecting ichthyoplankton biomass
and coho salmon recruitment in the biophysical network was computed by
correlating these two variables, after removing the effect of copepod biomass on
ichthyoplankton biomass and recruitment. To help identify the pathways with the
strongest associations between pairs of variables (i.e., relative pathway
strength), we averaged the absolute value of the partial correlation
coefficients for each link in a particular pathway connecting large-scale
climate variables with coho salmon recruitment.  For example, to estimate the
relative strength of the pathway including the PDO, SST, and coho salmon
recruitment in the physical network, we averaged the partial correlation
coefficients for the link between the PDO and SST and between SST and
recruitment.


\subsubsection{Network parameter estimation}

Both the physical and biophysical probabilistic networks took the form of linear
Gaussian probabilistic networks where the local probability distributions
associated with each variable were assumed to be Gaussian and the joint
distribution of all variables in the network was assumed to be multivariate
normal \citep{Shachter1989a, Koller2009a}. Parameters of local distributions
were estimated using linear regression models fit by maximum likelihood. For
variables with incoming arrows, the regression models were fit with the child
variable as the response variable and the parent variables as the predictor
variables. For example, in the biophysical network (Fig. \ref{fig:bn:5}), parameters for the
local distribution for SST were estimated using a linear regression model where
SST was the dependent variable and the ONI and PDO were the independent
variables. For root variables, models were fit with only an intercept term. For
coho salmon recruitment, models were fit using natural log-transformed coho
salmon recruitment data. Model fitting was performed using R and the Bayesian
network package \texttt{bnlearn} \citep{Scutari2010, Rcore2013a}.

Unlike conventional path analysis \citep{Wright1934}, the parameters of the
regression models (i.e., the parameters of the local distributions) were not of
direct interest in our probabilistic network analysis \citep{Korb2004a,
Koller2009a}. Instead, the fitted regression parameters were used along with a
Monte Carlo sampling algorithm to query the joint probability distribution of
the probabilistic network, which allowed us to estimate conditional posterior
probability distributions for recruitment given various scenarios for the
environmental variables, as explained in the following two sections
\citep{Henrion1988a}.


\subsubsection{Posterior distributions}

Using the fitted probabilistic networks, we estimated two sets of conditional
posterior probability distributions for coho salmon recruitment to quantify the
effect of the environmental variables on recruitment. For both sets of posterior
distributions, we first discretized the predictor environmental variables into
two categories, above or below the arithmetic mean value. The choice of using
two categories for the predictor variables was partly due to the low sample
sizes available for the environmental variables and to simplify presentation of
the results \citep{Koller2009a}. We then estimated conditional posterior
probabilities for a range of recruitment values that corresponded to the
observed recruitment data, which allowed us to summarize the posterior
probabilities using cumulative probability distributions.

For the first set of posterior distributions, we estimated the probability of
recruitment being less than a range of abundance values given values of a single
environmental variable in the network (i.e., we only specified conditions for a
single environmental variable at a time). For instance, we estimated the
conditional probability that recruitment would be less than or equal to 150 000
salmon given that SST was above average. For environmental variables not
directly connected to recruitment (e.g., the PDO), this set of posterior
distributions accounts for all pathways connecting that variable and recruitment
by propagating through the network uncertainty about the relationships among
pairs of variables (see the Posterior sampling section for details). That is,
this set of posterior distributions quantifies the joint effect of all pathways
specified in the network connecting the environmental variable and recruitment.
We evaluated two scenarios for each environmental variable corresponding to the
variable either being above or below average.

For the second set of posterior distributions, we estimated the probability of
recruitment being less than a range of abundance levels, given that all parent
variables of recruitment were either above of below their mean value. For the
physical network, this corresponded to estimating the conditional probability of
various levels of recruitment, given values for both SST and the spring
transition, whereas the conditioning variables on recruitment for the
biophysical network were ichthyoplankton biomass and copepod biomass. We
estimated probabilities for four scenarios of environmental variables. Because
both networks had two variables directly linked to recruitment, scenarios
included cases in which both environmental variables were either above or below
average and the two cases in which one of the environmental variables was above
average and the other was below average.

Each set of posterior distributions included a single probability distribution
for each discrete case of an environmental variable (i.e., above and below
average). To facilitate the interpretation of results, we calculated the maximum
difference between cumulative probability distributions (\(\Delta p\)) for each
discrete case of an environmental variable. For instance, \(\Delta p\) for the
two cumulative probability distributions showing the effects of SST on
recruitment was calculated by finding the maximum vertical difference (i.e.,
probability) between the cumulative probability distributions for recruitment
given above- and below-average SST conditions. To more easily compare results
from the physical and biophysical networks, we also estimated the posterior
probability that coho salmon recruitment would be less than or equal to 150 000
salmon, which is approximately equal to average recruitment for 1970-2011 (149
152 salmon).


\subsubsection{Posterior sampling}

We estimated the conditional posterior probability distributions for recruitment
using logic sampling (also known as forward sampling), which is a type of
rejection sampling \citep{Henrion1988a, Korb2004a}. As an example, to estimate
the conditional probability of recruitment being below 150 000 salmon given that
the PDO was above average in the physical network, we first sampled values for
the three large scale variables independently of each other weighting by the
prior distribution for each variable. We then sampled values for SST and
upwelling weighting by the known values of the large-scale variables.  The
spring transition was then sampled, weighting by the known values of SST and
upwelling. Finally, recruitment values were sampled, weighting by the known
values of both parent variables. The probability of recruitment being below 150
000 given an above average PDO value was then estimated by dividing the number
of samples where recruitment was less than 150 000 and the PDO was above average
by the number of samples where the PDO was above average.

More generally, the estimation algorithm consisted of sampling from the joint
posterior distribution, where the samples were weighted either by the prior
distribution for variables with no parents or the value of the parent variables.
The prior distributions for the root variables corresponded to the observed
distribution over the period included in the model and were sampled
independently for each root variable. Samples were only retained if the value of
the sampled evidence variable of interest (e.g., PDO is above average) was the
same as the value specified in the analysis. The conditional probability for
coho salmon recruitment given the evidence was then computed as the number of
samples where both the evidence and recruitment values matched the specified
value divided by the total number of samples where the sampled evidence values
match the specified value \citep{Henrion1988a}. For each analysis, we generated
1 000 000 samples from the posterior distribution in order to ensure that events
with low probabilities were sufficiently sampled \citep{Koller2009a}.



\section{Results}

\subsection{Pathway and link strength}

In both networks, the pathway with the highest relative strength originated with
the PDO (Table \ref{tab:bn:2}). In the physical network, the pathway with the highest average
link strength included the PDO, SST, and recruitment (average of the absolute
values of the two relevant correlations = 0.54; Fig. \ref{fig:bn:4} and Table \ref{tab:bn:2}) and the
pathway with the second highest relative strength included the ONI, SST, and
recruitment (average correlation = 0.46). The pathway with the strongest
association among variables in the biophysical network was nearly identical to
the strongest pathway in the physical network, but also included ichthyoplankton
biomass (average correlation = 0.54; Fig. \ref{fig:bn:5} and Table \ref{tab:bn:2}), while the pathway with
the second highest relative strength went from the PDO through deep temperature
and copepod biomass to recruitment (average correlation = 0.53).

In the physical network, the two environmental variables with a direct effect on
recruitment (SST and spring transition) had a negative relationship with
recruitment indicating that cooler surface temperatures and an earlier spring
transition date are associated with higher recruitment (Fig. \ref{fig:bn:4}). Between these
two variables, SST had a considerably stronger relationship with recruitment
than the spring transition with a partial correlation coefficient more than
twice as strong (Fig. \ref{fig:bn:4}). In the biophysical network, both variables with a
direct effect on recruitment (ichthyoplankton and copepod biomass) had a
positive relationship with recruitment, suggesting higher prey biomass is
associated with increased recruitment, although the relationship between
ichthyoplankton biomass and recruitment was twice as strong as the relationship
between copepod biomass and recruitment (Fig. \ref{fig:bn:5}).


\subsection{Posterior distributions}

In the probabilistic analysis, the variables with the strongest overall effect
(i.e., the joint effect of all pathways connecting a single environmental
variable and recruitment) on the probability of recruitment were regional-scale
physical and biological variables with a direct effect on recruitment (Figs. 
\ref{fig:bn:6} and \ref{fig:bn:7}). In the physical network, SST had the strongest effect on recruitment with
\(\Delta p\) = 0.29 (Fig.  \ref{fig:bn:6}d), which was moderately larger than the next most
influential variable, the PDO (\(\Delta p\) = 0.18; Fig. \ref{fig:bn:6}b). For the
biophysical network, ichthyoplankton biomass had the strongest effect on
recruitment with \(\Delta p\) = 0.40, which was considerably stronger than all
other variables in the network (Fig. \ref{fig:bn:7}). Differences in steepness of the two
conditional probability distributions for the physical network meant that there
was a 76\% chance that recruitment would be 150 000 or less when SST was above
average and a 49\% chance of recruitment being equal to or below that level when
SST was below average (Fig. \ref{fig:bn:6}d). Likewise, for the biophysical network, there
was a 34\% chance that recruitment would be 150 000 or less when ichthyoplankton
biomass was greater than average and a 73\% chance when ichthyoplankton biomass
was less than average (Fig. \ref{fig:bn:7}g).

Among the three large-scale climate variables, the PDO had the strongest overall
effect on recruitment in both networks with a \(\Delta p\) between 2 and 4 times
greater than for the ONI and NPGO (Figs. \ref{fig:bn:6} and \ref{fig:bn:7}). In particular, a warm PDO (i.e.,
when the PDO was above average) was associated with lower recruitment. For
example, there was a 71\% chance that recruitment would be 150 000 salmon or
less when the PDO was in a warm phase for the physical network and a 62\% chance
for the biophysical network (Figs. \ref{fig:bn:6}b and \ref{fig:bn:7}b). When the PDO was cool, the
probability of recruitment being equal to or below 150 000 was considerably
less, with a 54\% chance in the physical network and a 45\% chance in the
biophysical network. In contrast, for the NPGO and ONI, the probability of
recruitment being 150 000 or less was nearly identical, regardless of whether
these variables were above or below average.

When both parent variables of recruitment in the physical network were above
average, that is, when SST was warm and the spring transition occurred late, the
cumulative probability distribution for recruitment (thick grey curve in Fig.
8a) was considerably steeper compared to when SST was cool and the spring
transition occurred early (thick black curve in Fig. \ref{fig:bn:8}a). This difference in
steepness corresponded to an 81\% chance recruitment would be equal to or below
150 000 salmon when both conditioning variables were above average but only a
43\% when both variables were below average (Fig. \ref{fig:bn:8}a). For the biophysical
network, when both variables that index coho salmon prey resources (i.e.,
ichthyoplankton and copepod biomass) were above average, the cumulative
probability distribution for recruitment was considerably less steep (thick grey
curve in Fig. \ref{fig:bn:8}b) than when the prey resource indices were below average (thick
black curve in Fig. \ref{fig:bn:8}b). This difference in the cumulative probability
distributions equated to a 25\% chance that recruitment would be 150 000 salmon
or less when both prey indices were above average and a 81\% chance recruitment
would be equal to or below that level when both prey indices were below average.

When oceanographic conditions were mixed, that is, when one parent variable of
recruitment was above average and the other was below average, the probability
that recruitment would be 150 000 salmon or less tended to be more influenced by
SST than the spring transition date in the physical network and by
ichthyoplankton biomass than copepod biomass in the biophysical network (Fig.
8). For instance, the cumulative probability distribution when ichthyoplankton
biomass was below average and copepod biomass was above average (black dashed
curve in Fig. \ref{fig:bn:8}b) was moderately steeper compared to when ichthyoplankton
biomass was above average and copepod biomass was below average (grey dashed
curve in Fig. \ref{fig:bn:8}b).


\subsection{Productivity index}

The results from the networks fitted using an index of coho salmon productivity
were qualitatively the same as the results shown above for the networks fitted
using total coho salmon recruitment. The rank order of the pathways with the
highest average partial correlation coefficients was identical for both the
physical and biophysical networks (Supporting materials Table \ref{tab:bn:s1}). Similarly, the
rank order of the influence of each environmental variable on coho salmon (as
indicated by \(\Delta p\)) was identical for both the physical and biophysical
networks (Supporting materials Figs. \ref{fig:bn:s2}, \ref{fig:bn:s3}, and
\ref{fig:bn:s4}).



\section{Discussion}

In this study, we estimated the joint effect and relative strength of multiple
ecological pathways on coho salmon recruitment in the Northern California
Current to better understand the mechanisms linking climate variability and
salmon recruitment. We found (1) pathways originating with the PDO were the most
influential on recruitment with a joint effect considerably larger than for the
ONI or NPGO, (2) warm ocean years (i.e., when surface temperatures were above
average) were associated with reduced salmon prey biomass as well as lower
recruitment levels compared to cool years, and (3) the probability of coho
salmon recruitment being below average was most strongly influenced by
regional-scale SST and ichthyoplankton biomass. These results suggest that
shifts in climate and ocean conditions resulting from natural variability or
anthropogenic climate change could influence salmon recruitment through multiple
mechanisms.

Our findings indicate there were multiple pathways with high average link
strength connecting the PDO and recruitment of Oregon coho salmon, suggesting
that a single large-scale climate event can influence salmon recruitment
simultaneously through multiple mechanisms. In the pathway with the strongest
associations among variables, SST and ichthyoplankton biomass mediated the
effects of the PDO on recruitment. This result broadly agrees with those of
several previous studies that suggest thermal environments are important for the
recruitment processes of higher-trophic level species \citep{Martins2012,
Planque1999, Hunt2011a}. Although temperature can influence salmon directly, for
example, by influencing metabolic or growth rates \citep{Mortensen2000a,
Farley2007b}, the occurrence of ichthyoplankton biomass as an intermediary
between SST and recruitment in the biophysical network suggests that the
indirect effect of temperature through bottom-up forcing can also strongly
influence recruitment. In particular, it appears that cooler ocean temperatures
are associated with increased prey resources for juvenile salmon. This is
consistent with the findings of \citet{Daly2013} and also supports the idea of a
combined influence of ocean temperature and prey resources on juvenile salmon
\citep{Pearcy1992a}.

The pathway with the second strongest association among variables included deep
water temperature and copepod biomass as intermediaries between the PDO and
recruitment, indicating changes in ocean current patterns and the subsequent
advection of zooplankton into the Northern California Current region may also
influence recruitment. Off the Oregon coast, zooplankton are generally
associated with one of two community structures, (1) a northern community that
has low species diversity and large copepod species that are rich in lipids, or
(2) a southern community that has high species diversity and small copepod
species that are poor in lipids \citep{Hooff2006a}. The observed negative
relationship between deep water temperature and copepod biomass supports the
findings of previous studies that indicated the lipid-rich zooplankton community
is associated with the transport of cooler water from northern areas into the
Northern California Current \citep{Keister2011a}. Furthermore, the positive
relationship between copepod biomass and recruitment suggests increased
lipid-rich copepod prey resources are associated with higher salmon recruitment
\citep{Bi2011a}.

In both networks, cool PDO conditions (i.e., when the PDO was below the
long-term average) were associated with increased recruitment compared to warm
conditions. Furthermore, cool periods were also associated with cooler deep
water temperatures (indicative of increased equator-ward transport), increased
upwelling, increased ichthyoplankton biomass, and a more northern copepod
community composition off the coast of Oregon.  Although numerous previous
studies have indicated similar associations among these environmental variables
during cool PDO regimes \citep{King2011, Peterson2003b, Keister2011a,
Mantua1997a}, our results extend those findings by explicitly quantifying the
uncertainty in these relationships in the form of posterior probabilities. For
example, the results for the biophysical network indicated that under warm PDO
conditions, there was a 62\% chance that recruitment would be below the
long-term average. In contrast, under cool PDO conditions, there was only a 45\%
chance recruitment would be below average. Although this suggests that cool
ocean conditions are beneficial for coho salmon, it also indicates there is
considerable uncertainty about recruitment levels even when the PDO is below
average. This uncertainty may partly arise due to our focus on recruitment,
which also includes variability from the freshwater life phase, although our
sensitivity analysis using productivity showed almost the same results as using
total recruitment.

We found that ichthyoplankton biomass tended to be more influential than copepod
biomass on recruitment with an effect twice as strong. The importance of
ichthyoplankton biomass over copepod biomass was surprising because several
previous studies have reported strong positive relationships between the biomass
of northern copepod species and coho salmon survival \citep{Bi2011a,
Ruzicka2011, Peterson2003b}.  However, the finding is consistent with evidence
indicating that coho salmon diets during the first ocean summer are primarily
composed of small fish species (by percent weight) such as Pacific sand lance
and osmerids \citep{Brodeur2007a, Weitkamp2008a}. The importance of
ichthyoplankton biomass likely reflects a bottom-up forcing mechanism where
increased ichthyoplankton biomass results in increased growth rates, body size,
and marine survival of coho salmon. However, because ichthyoplankton are also
prey for numerous other species \citep{Gladics2014, Miller2010, Miller2007} and
because ocean conditions that influence ichthyoplankton biomass may also
influence other marine species, this variable may act as a surrogate for other
biological processes that directly influence survival such as predator
distributions, abundances, or diets.

The result that upwelling did not have a strong effect on recruitment deviates
from earlier findings that increased upwelling intensity is associated with
increased marine survival of Oregon coho salmon \citep{Fisher1988a,
Logerwell2003a}. This difference in results may be due to at least three
factors. First, in this study we used total recruitment from wild coho salmon
stocks, which includes variability associated with the freshwater life phase,
whereas the previous studies used marine survival of hatchery reared coho salmon
as the response variable. Second, there is some evidence that the relationship
between upwelling and salmon survival may not be stationary. In particular,
\citet{Botsford2002} and \citet{Pearcy1997} indicated that the previous strong
correlation between upwelling and coho salmon survival in the Northern
California Current has broken down since the early 1990s.  Third, the weak
relationship may also be due to how upwelling was indexed. To index upwelling
intensity, we averaged the daily Bakun upwelling index over the months of March
and April. However, this index does not differentiate between magnitude and
duration of upwelling events within this period. In particular, sustained wind
speeds over a certain threshold may reduce ecosystem productivity due to
transport of nutrients and phytoplankton out of the system \citep{Botsford2006,
Botsford2003}. Therefore, it is possible that shorter-term upwelling ``events''
on the scale of days or weeks may be more important for determining productivity
in the coastal ecosystem than the seasonal upwelling average.

Some results from the physical and biophysical networks were similar, including
showing the importance of the PDO and SST in the strongest pathways and the
minimal influence of the NPGO and upwelling on recruitment. These similarities
between the networks, which were fit using different but overlapping years,
suggests that the major pathways and the most important climate and physical
variables for determining recruitment did not differ greatly between the
1970-2011 and 1998-2011 time periods. In addition, this similarity indicates
that our results are not sensitive to the different network structures that we
used to connect large-scale climate and regional-scale physical variables.

The parameters of the local distributions in our probabilistic networks were
estimated using linear regression models, therefore, we implicitly assumed these
relationships were stationary (i.e., the parameters were constant over time)
\citep{Walters1987a}. However, sharp changes in the abundance and productivity
index for Oregon Coast coho salmon over the past 40 years (Fig. \ref{fig:bn:3}
and Supporting materials Fig. \ref{fig:bn:s1}) may reflect non-stationarity in the recruitment time
series. Detecting non-stationarity and its causes is often problematic in
fisheries abundance time series due to small sample sizes, the lack of contrast
in the data, and confounding with environmental conditions and changes in
harvest management \citep{Walters1987a, Peterman2009a}. In the case of Oregon
Coast coho salmon, the short time series make it difficult to detect
non-stationarity, as opposed to large variations, even though there is
moderately good contrast in the recruitment estimates. In addition, potential
changes in the underlying relationships between recruitment and the
environmental variables may be confounded with other factors including changes
in harvest management strategies, or changes in freshwater habitat. Because the
results of the physical and biophysical networks were similar (suggesting there
was little change in relationships between the entire study period and the most
recent 14 years), we believe our stationarity assumption is valid for the time
periods investigated, although, we caution against extrapolating the reported
relationships outside the temporal bounds used to fit the probabilistic
networks.

Our network modeling approach explicitly depicted the hypothesized ecological
network, however, relationships within the network are correlative rather than
causative. Therefore, our results could be confounded with variables or pathways
not included in the networks. For example, the networks presented in this study
only included pathways representing bottom-up forcing. It is likely though, that
shifts in large-scale climate patterns also influence top-down forcing pathways,
for instance, by influencing the distribution of predators of juvenile coho
salmon \citep{Pearcy2002a, Perry2005a}. Therefore, our findings merely represent
an initial step in understanding how salmon recruitment is influenced by the
ecological network: further research is needed into other ecological pathways to
more clearly identify the ecological mechanisms.

While our research focused on using the fitted probabilistic networks to better
understand how environmental conditions influence coho salmon, the networks
could also be used as a tool to help managers of salmon fisheries by providing
pre-season forecasts of recruitment given different scenarios of environmental
conditions \citep{Nyberg2006a, Araujo2013}. The probabilistic network approach
has several advantages over more traditional modeling and forecasting methods
(e.g., stock recruitment models with environmental effects) including explicitly
depicting the underlying ecological network, accounting for multiple pathways
and indirect effects, and presenting results in a probabilistic form. In
particular, such explicit representation of uncertainty is an important aspect
of using ecological models to guide decision-making \citep{Clark2001a}.

In conclusion, our results demonstrate that large-scale climate patterns can
strongly influence coho salmon recruitment simultaneously through multiple
ecological pathways. This suggests that multiple ecological mechanisms may
underlie the large fluctuations observed in adult returns of Pacific salmon. In
particular, it appears that both thermal regimes and prey resources are
important processes in the mechanisms connecting climate variability and salmon
recruitment. Taken together, these conclusions highlight the importance of
quantifying the cumulative effects of these pathways to better understand how
future changes in climate patterns will influence higher-trophic level species
\citep{Ainsworth2011a, Fulton2011}.



\section{Acknowledgments}

We thank the numerous people who collected and compiled the physical and
biological oceanographic data over the past decades. In particular, we thank Jay
Peterson, Leah Feinberg, Tracy Shaw, Jennifer Menkel and Jennifer Fisher.
Funding for this research was provided by Simon Fraser University and a grant
from the Canada Research Chairs Program to R.M.  Peterman. We also thank Franz
Mueter and two anonymous referees for helpful comments and discussions.



\section{Tables}
% Table captions are located in the R files that produce the tables

% Variable Summary Table
% 2016-10-31


{\small
  \libertineLF

  \begin{ThreePartTable}
    \begin{TableNotes}
      {\footnotesize
        \item[a] 1: \url{http://www.cpc.ncep.noaa.gov}; 
          2: \url{http://jisao.washington.edu/pdo/PDO.latest}; 
          3: \url{http://www.o3d.org/npgo/npgo.php}; 
          4: \url{http://www.ncdc.noaa.gov/ersst/}; 
          5: \url{http://www.pfel.noaa.gov}; 
          6: \citet{Peterson2012a}; 
          7: \citet{PFMC2013}.
        }
    \end{TableNotes}
  
    \begin{longtable}{lllll}
      \caption[Summary of environmental variables used to construct the
        probabilistic networks.]{Summary of environmental variables used to
        construct the probabilistic networks. Seasonal average indicates the period
        over which each variable was averaged. Extent refers to the north-south
        spatial area that the variable covers (degrees latitude).} \\ 
      \hline
      % latex table generated in R 3.3.1 by xtable 1.8-2 package
% Wed Nov  2 13:52:13 2016
\begin{tabular}{lllll}
  \hline
Variable & Seasonal Average & Extent & Years & Source\textsuperscript{a} \\ 
  \hline
ONI & December-March & 5N-5S & 1970-2011 & 1 \\ 
  PDO & December-March & 20-65N & 1970-2011 & 2 \\ 
  NPGO & December-March & 25-62N & 1970-2011 & 3 \\ 
  SST & January-June & 43-45N & 1970-2011 & 4 \\ 
  Upwelling & March-April & 43.5-46.5N & 1970-2011 & 5 \\ 
  Spring Transition &  & 43.5-46.5N & 1970-2011 & 5 \\ 
  Deep Temperature & May-September & 44.6N & 1998-2011 & 6 \\ 
  Ichthyoplankton & January-March & 44.6N & 1998-2011 & 6 \\ 
  Copepod Biomass & May-September & 44.6N & 1998-2011 & 6 \\ 
  Coho Recruitment &  & 43-46N & 1970-2011 & 7 \\ 
   \hline
\end{tabular}

      \hline
      \insertTableNotes
      \label{tab:bn:1}
    \end{longtable}

  \end{ThreePartTable}
}

% \noindent \textsuperscript{a} 1: \url{http://www.cpc.ncep.noaa.gov}; 
%   2: \url{http://jisao.washington.edu/pdo/PDO.latest}; 3:
%   \url{http://www.o3d.org/npgo/npgo.php}; 4:
%   \url{http://www.ncdc.noaa.gov/ersst/}; 5: \url{http://www.pfel.noaa.gov}; 6:
%   \citet{Peterson2012a}; 7: \citet{PFMC2013}.


\noindent \textsuperscript{a} 1: \url{http://www.cpc.ncep.noaa.gov}; 
  2: \url{http://jisao.washington.edu/pdo/PDO.latest}; 3:
  \url{http://www.o3d.org/npgo/npgo.php}; 4:
  \url{http://www.ncdc.noaa.gov/ersst/}; 5: \url{http://www.pfel.noaa.gov}; 6:
  \citet{Peterson2012a}; 7: \citet{PFMC2013}.

\newpage



% latex table generated in R 3.2.2 by xtable 1.7-4 package
% Tue Oct 13 11:12:23 2015
{\small
\begin{longtable}{lll}
\caption[Relative pathway strength for each pathway connecting large-scale
         climate variables and coho salmon recruitment.]{Relative pathway strength for each pathway connecting large-scale
         climate variables and coho salmon recruitment in the physical and
         biophysical networks; $\overline{r}$ gives the average of the absolute
         value of the partial correlation coefficients for each link in a
         pathway.} \\ 
  \hline
Network & Pathway & $\overline{r}$ \\ 
  \hline
Physical & PDO, SST  & 0.54 \\ 
   & ONI, SST  & 0.46 \\ 
   & NPGO, SST  & 0.43 \\ 
   & PDO, SST, Spring Transition  & 0.35 \\ 
   & ONI, SST, Spring Transition  & 0.30 \\ 
   & ONI, Upwelling, Spring Transition  & 0.28 \\ 
   & NPGO, SST, Spring Transition  & 0.28 \\ 
   & PDO, Upwelling, Spring Transition  & 0.27 \\ 
   & NPGO, Upwelling, Spring Transition  & 0.24 \\ 
  Biophysical & PDO, SST, Ichthyoplankton  & 0.54 \\ 
   & PDO, Deep Temperature, Copepod Biomass  & 0.53 \\ 
   & ONI, SST, Ichthyoplankton  & 0.50 \\ 
   & NPGO, Deep Temperature, Copepod Biomass  & 0.49 \\ 
   & PDO, Upwelling, Spring Transition, Copepod Biomass  & 0.36 \\ 
   & PDO, SST, Spring Transition, Copepod Biomass  & 0.35 \\ 
   & ONI, Upwelling, Spring Transition, Copepod Biomass  & 0.35 \\ 
   & ONI, SST, Spring Transition, Copepod Biomass  & 0.33 \\ 
  \hline
\label{tab:bn:2}
\end{longtable}
}


\newpage



\section{Figures}

\begin{figure}[htbp]
  \centering \includegraphics[scale=0.75]{./4_bayesnetwork/figures/fig1_intro_path.pdf}
  \caption{Schematic of pathways linking large-scale climate processes and
    Pacific salmon recruitment; (a) pathway where climate has a direct effect on
    recruitment, (b) pathway where climate effects on recruitment are mediated
    by a regional-scale oceanographic process (e.g., upwelling), (c) climate
    effects are mediated by multiple regional-scale variables resulting in two
    pathways connecting climate and recruitment.}
  \label{fig:bn:1}
\end{figure}

\begin{figure}[htbp]
  \centering \includegraphics[scale=0.9]{./4_bayesnetwork/figures/map.pdf}
  \caption{Study area showing the ocean-entry locations of the ten largest river
    basins located within the Oregon Coast coho salmon evolutionarily
    significant unit (ESU).}
  \label{fig:bn:2}
\end{figure}

\begin{figure}[htbp]
  \centering \includegraphics[scale=0.9]{./4_bayesnetwork/figures/recruit_ts.pdf}
  \caption{Time series of spawning stock size (dashed line) and the resulting
    total recruitment (solid line) for Oregon Coast coho salmon. Grey shaded
    region indicates the period used for the biophysical network.}
  \label{fig:bn:3}
\end{figure}

\begin{figure}[htbp]
  \centering \includegraphics[scale=0.75]{./4_bayesnetwork/figures/fig3_p3r.pdf}
  \caption{Directed acyclic graph for the probabilistic network fit using only
    physical environmental variables (called the physical network here). Ovals
    represent variables and arrows indicate dependencies among variables within
    the network. Numbers next to each arrow are the partial correlation
    coefficients. Thick solid arrows indicate the pathway with the highest
    average partial correlations (``link strength'', i.e., PDO to SST to coho
    recruitment), whereas the thick dashed arrow indicates the first link in the
    pathway with the second-highest average link strength.}
  \label{fig:bn:4}
\end{figure}

\begin{figure}[htbp]
  \centering \includegraphics[scale=0.75]{./4_bayesnetwork/figures/fig4_b3r.pdf}
  \caption{Directed acyclic graph for the probabilistic network fit using both
    physical and biological environmental variables (called the biophysical
    network here). Ovals represent variables and arrows indicate dependencies
    among variables within the network. Numbers next to each arrow are the
    partial correlation coefficients. Thick solid arrows indicate the pathway
    with the highest average partial correlations (i.e., link strength), whereas
    thick dashed arrows indicate the pathway with the second highest average
    link strength.}
  \label{fig:bn:5}
\end{figure}

\begin{figure}[htbp]
  \centering \includegraphics[scale=0.9]{./4_bayesnetwork/figures/p3r_post_vars.pdf}
  \caption{Cumulative probability distributions of coho salmon recruitment for
    the physical network conditioned on each variable in the network. Thick grey
    curves indicate the cumulative probability for recruitment given that the
    environmental variable is greater than average, whereas thick black dashed
    curves show cumulative probability when the environmental variable is less
    than average. Thin dotted lines indicate the cumulative probabilities for
    150 000 salmon. The \(\Delta p\) gives the maximum vertical difference in
    probability between the two cumulative distributions within a panel.}
  \label{fig:bn:6}
\end{figure}

\begin{figure}[htbp]
  \centering \includegraphics[scale=0.9]{./4_bayesnetwork/figures/b3r_post_vars.pdf}
  \caption{Cumulative probability distributions of coho salmon recruitment for
    the biophysical network conditioned on each variable in the network. Thick
    grey curves indicate the cumulative probability for recruitment given that
    the environmental variable is greater than average, whereas thick black
    dashed curves show cumulative probability when the environmental variable is
    less than average. Thin dotted lines indicate the cumulative probabilities
    for 150 000 salmon. The \(\Delta p\) gives the maximum vertical difference
    in probability between the two cumulative distributions within a panel.}
  \label{fig:bn:7}
\end{figure}

\begin{figure}[htbp]
  \centering \includegraphics[scale=0.9]{./4_bayesnetwork/figures/p3r_b3r_coho_post_parents.pdf}
  \caption{Cumulative conditional posterior probability distributions for coho
    salmon recruitment conditioned on the parent variables of recruitment for
    (a) the physical network (conditioned on sea surface temperature (SST) and
    spring transition (ST) date) and (b) the biophysical network (conditioned on
    ichthyoplankton biomass (Ich) and copepod biomass (Cope)). Thick solid grey
    curves indicate the cumulative probability for recruitment given that both
    conditioning variables are greater than average (+). Thick solid black
    curves show cumulative probabilities when both variables are less than their
    long-term averages (-). Thick dashed curves indicate scenarios when the
    conditioning variables are mixed. For the physical network, the grey dashed
    curve indicates cumulative probabilities when SST is above average and the
    spring transition is below average, whereas the dashed black curve shows the
    cumulative probabilities for the opposite conditions. For the biophysical
    network, the grey dashed curve indicates cumulative probabilities when
    ichthyoplankton biomass is above average and copepod biomass is below
    average, whereas the dashed black curve shows the cumulative probabilities
    for the opposite conditions. Thin dotted lines indicate the cumulative
    probabilities for 150 000 salmon.  The \(\Delta p\) gives the maximum
    vertical difference in probability between the two cumulative distributions
    where both conditioning variables are either above or below average within a
    panel.}
  \label{fig:bn:8}
\end{figure}

\newpage
\section{Supporting materials}

\subsection{Supporting Figures and Tables}

% Supplemental Pathway Strength Table
% 2016-10-31

{\small
  \libertineLF
  \begin{longtable}{lll}
    \caption[Relative pathway strength for each pathway connecting large-scale
      climate variables and the coho salmon productivity index.]{Relative
      pathway strength for each pathway connecting large-scale climate variables
      and the \emph{coho salmon productivity index} in the physical and
      biophysical networks; $\overline{r}$ gives the average of the absolute
      value of the partial correlation coefficients for each link in a pathway.} \\ 
    \hline
    % latex table generated in R 3.3.1 by xtable 1.8-2 package
% Mon Oct 31 10:18:27 2016
Network & Pathway & $\overline{r}$ \\ 
  \hline
Physical & PDO, SST  & 0.56 \\ 
   & ONI, SST  & 0.48 \\ 
   & NPGO, SST  & 0.44 \\ 
   & PDO, SST, Spring Transition  & 0.36 \\ 
   & ONI, SST, Spring Transition  & 0.30 \\ 
   & ONI, Upwelling, Spring Transition  & 0.28 \\ 
   & NPGO, SST, Spring Transition  & 0.28 \\ 
   & PDO, Upwelling, Spring Transition  & 0.27 \\ 
   & NPGO, Upwelling, Spring Transition  & 0.25 \\ 
  Biophysical & PDO, SST, Ichthyoplankton  & 0.58 \\ 
   & PDO, Deep Temperature, Copepod Biomass  & 0.55 \\ 
   & ONI, SST, Ichthyoplankton  & 0.54 \\ 
   & NPGO, Deep Temperature, Copepod Biomass  & 0.51 \\ 
   & PDO, Upwelling, Spring Transition, Copepod Biomass  & 0.37 \\ 
   & PDO, SST, Spring Transition, Copepod Biomass  & 0.36 \\ 
   & ONI, Upwelling, Spring Transition, Copepod Biomass  & 0.36 \\ 
   & ONI, SST, Spring Transition, Copepod Biomass  & 0.34 \\ 
  
    \hline
    \label{tab:bn:s1}
  \end{longtable}
}




\begin{figure}[htbp]
  \centering \includegraphics[scale=0.9]{./4_bayesnetwork/figures/S_productivity_ts.pdf}
  \caption{Time series of the coho salmon productivity index. The productivity
    index was calculated as the residuals from a Beverton-Holt spawner-recruit
    model. Grey shaded region indicates the period used for the biophysical
    network.}
  \label{fig:bn:s1}
\end{figure}

\begin{figure}[htbp]
  \centering \includegraphics[scale=0.9]{./4_bayesnetwork/figures/S_p3e_post_vars.pdf}
  \caption{Cumulative probability distributions of the \emph{coho salmon
    productivity index} for the physical network conditioned on each variable in
    the network. Thick grey curves indicate the cumulative probability for the
    productivity index given that the environmental variable is greater than
    average, whereas thick black dashed curves show cumulative probability when
    the environmental variable is less than average. Thin dotted lines indicate
    the cumulative probabilities for average productivity. The \(\Delta p\)
    gives the maximum vertical difference in probability between the two
    cumulative distributions within a panel.}
  \label{fig:bn:s2}
\end{figure}

\begin{figure}[htbp]
  \centering \includegraphics[scale=0.9]{./4_bayesnetwork/figures/S_b3e_post_vars.pdf}
  \caption{Cumulative probability distributions of the \emph{coho salmon
    productivity index} for the biophysical network conditioned on each variable
    in the network. Thick grey curves indicate the cumulative probability for
    the productivity index given that the environmental variable is greater than
    average, whereas thick black dashed curves show cumulative probability when
    the environmental variable is less than average. Thin dotted lines indicate
    the cumulative probabilities for average productivity. The \(\Delta p\)
    gives the maximum vertical difference in probability between the two
    cumulative distributions within a panel.}
  \label{fig:bn:s3}
\end{figure}

\begin{figure}[htbp]
  \centering \includegraphics[scale=0.9]{./4_bayesnetwork/figures/S_p3e_b3e_coho_post_parents.pdf}
  \caption{Cumulative conditional posterior probability distributions for the
    \emph{coho salmon productivity index} conditioned on the parent variables of
    recruitment for (a) the physical network (conditioned on sea surface
    temperature (SST) and spring transition (ST) date) and (b) the biophysical
    network (conditioned on ichthyoplankton biomass (Ich) and copepod biomass
    (Cope)). Thick solid grey curves indicate the cumulative probability for the
    productivity index given that both conditioning variables are greater than
    average (+). Thick solid black curves show cumulative probabilities when
    both variables are less than their long-term averages (-). Thick dashed
    curves indicate scenarios when the conditioning variables are mixed. For the
    physical network, the grey dashed curve indicates cumulative probabilities
    when SST is above average and the spring transition is below average,
    whereas the dashed black curve shows the cumulative probabilities for the
    opposite conditions. For the biophysical network, the grey dashed curve
    indicates cumulative probabilities when ichthyoplankton biomass is above
    average and copepod biomass is below average, whereas the dashed black curve
    shows the cumulative probabilities for the opposite conditions.  Thin dotted
    lines indicate the cumulative probabilities for average productivity. The
    \(\Delta p\) gives the maximum vertical difference in probability between
    the two cumulative distributions where both conditioning variables are
    either above or below average within a panel.} 
  \label{fig:bn:s4}
\end{figure}
