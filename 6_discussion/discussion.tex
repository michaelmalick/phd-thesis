\chapter{General Discussion}\label{general-discussion}

\begin{enumerate}
\def\labelenumi{\arabic{enumi}.}
\item
  In this thesis, I have sought to contribute to our knowledge of how
  environmental forcing pathways link climatic and ocean processes to
  dynamics of Pacific salmon populations in the Northeast Pacific Ocean.
  My second and third chapters apply a cross-system comparative approach
  to rigorously assess the evidence for population-level responses to
  inter-annual changes in two alternative meso-scale ocean processes. In
  chapter 2, I examine the hypothesis that phytoplankton dynamics in
  coastal ecosystems---which are largely determined by vertical ocean
  transport processes---are an important driver of Pacific salmon
  productivity. In chapter 3, I examine an alternative hypothesis that
  suggests that ocean processes driven by horizontal ocean transport,
  such as the advection of zooplankton, are equally important drivers of
  salmon productivity as processes driven by vertical ocean transport.
  My fourth chapter builds on the previous chapters, which each focused
  on a single meso-scale ocean process, by exploring the cumulative
  effects and relative importance of multiple environmental pathways on
  Pacific salmon dynamics. My fifth chapter applies an interdisciplinary
  approach to examine challenges to integrating highly-migratory
  anadromous fish species into place-based ecosystem-based management
  policies and provides practical recommendations for overcoming the
  identified challenges. In total, this thesis further develops our
  quantitative understanding about how climatic and ocean processes
  influence the population dynamics of Pacific salmon and in doing so
  contributes to reducing uncertainties about how environmental change
  impacts living marine resources.
\item
  My thesis makes several contributions to the study of environmental
  forcing in coastal ecosystems. The results presented in this thesis
  strongly suggest that climatic and ocean processes can impact salmon
  populations simultaneously through multiple environmental pathways and
  that the relative importance of these pathways is non-stationary
  through space. My second and third chapters provide evidence that
  environmental forcing pathways mediated by either vertical or
  horizontal ocean transport processes can strongly impact the dynamics
  of higher-trophic-level species. In both studies, however, the effects
  of environmental conditions were dependent on the geographic location
  of juvenile salmon ocean entry, suggesting that in some cases
  environmental forcing is non-stationary across space. A practical
  implication of this spatial non-stationarity is that relationships
  inferred from data in one location may not be applicable to another
  location. My fourth chapter further shows the need to account for
  multiple environmental pathways when considering how environmental
  forcing impacts higher-trophic-level species, e.g., by estimating the
  cumulative effects and relative importance of multiple hypothesized
  pathways linking climatic changes and the dynamics of exploited
  species. My fifth chapter indicates that considering spatial scales of
  the impacts of human and natural disturbances on highly-migratory
  species is an important component to integrating these species into
  ecosystem-based management policies.
\item
  Environmental variability is an intrinsic element of coastal
  ecosystems and can have profound impacts on ecosystem structure and
  function as well as the human-derived benefits from the system.
  Uncertainties about the impacts of changing environmental conditions
  on living marine resources are likely to always be present and
  effective decision-making in the presence of this uncertainty is a
  fundamental challenge of managing natural resources. Ultimately,
  understanding how environmental forcing impacts commercially valuable
  species is a necessary component to implementing management actions
  that are robust to a wide range of potential future scenarios.
\end{enumerate}
