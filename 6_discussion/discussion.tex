\chapter{General Discussion}\label{general-discussion}

\begin{enumerate}
\def\labelenumi{\arabic{enumi}.}
\item
  In this thesis, I have sought to contribute to our knowledge of how
  environmental forcing pathways link climatic and ocean processes to
  dynamics of Pacific salmon populations in the Northeast Pacific Ocean.
  My second and third chapters apply a cross-system comparative approach
  to rigorously assess the evidence for population-level responses to
  inter-annual changes in two meso-scale ocean processes. In chapter 2,
  I examine the hypothesis that phytoplankton dynamics in coastal
  ecosystems---which are largely determined by vertical ocean transport
  processes---are an important driver of Pacific salmon productivity. In
  chapter 3, I examine an alternative hypothesis that suggests that
  ocean processes driven by horizontal ocean transport, such as the
  advection of zooplankton, are equally important drivers of salmon
  productivity as processes driven by vertical ocean transport. My
  fourth chapter builds on the previous chapters, which each focused on
  a single meso-scale ocean process, by exploring the cumulative effects
  and relative importance of multiple environmental pathways on Pacific
  salmon dynamics. My fifth chapter applies an interdisciplinary
  approach to examine challenges to integrating highly-migratory
  anadromous fish species into place-based ecosystem-based management
  policies and provides practical recommendations for overcoming the
  identified challenges. In total, this thesis further develops our
  quantitative understanding about how climatic and ocean processes
  influence the population dynamics of Pacific salmon and in doing so
  contributes to reducing uncertainties about how environmental change
  impacts living marine resources.
\item
  My thesis makes several contributions to the study of environmental
  forcing in coastal ecosystems. My second and third chapters provide
  evidence that environmental forcing pathways mediated by either
  vertical or horizontal ocean transport processes can strongly impact
  the dynamics of higher-trophic-level species. In both studies,
  however, the effects of environmental conditions were dependent on the
  geographic location of juvenile salmon ocean entry, suggesting that in
  some cases environmental forcing is non-stationary across space. A
  practical implication of this spatial non-stationarity is that
  relationships inferred from data in one location may not be applicable
  to another location. My fourth chapter further shows the need to
  account for multiple environmental pathways when considering how
  environmental forcing impacts higher-trophic-level species, e.g., by
  estimating the cumulative effects and relative importance of multiple
  hypothesized pathways linking climatic changes and the dynamics of
  exploited species. My fifth chapter indicates that considering spatial
  scales of the impacts of human and natural disturbances on
  highly-migratory species is an important component to integrating
  these species into ecosystem-based management policies. Together, the
  results presented in this thesis strongly suggest that climatic and
  ocean processes can impact salmon populations simultaneously through
  multiple environmental pathways and that the relative importance of
  these pathways may be non-stationary through space.
\end{enumerate}

\section{Challenges}\label{challenges}

\begin{enumerate}
\def\labelenumi{\arabic{enumi}.}
\setcounter{enumi}{2}
\item
  This thesis also reveals several remaining challenges in understanding
  the effects of environmental forcing on living marine resources. A
  first challenge is disentangling the effects of environmental forcing
  that occur through bottom-up and top-down pathways. Changes in
  climatic and physical ocean processes can simultaneously impact food
  resource availability and predator distributions or abundances. Only
  considering environmental forcing pathways that represent bottom-up
  forcing is inherently incomplete and future research should strive to
  further understand interactions and cumulative effects of processes
  driving interactions across multiple trophic levels. For instance, as
  the climate warms, the relative importance of bottom-up and top-down
  pathways may shift as marine species track preferred environmental
  conditions {[}@Pinsky2013{]}. End-to-end ecosystem models (e.g.,
  Atlantis) provide one option for improving our understanding of
  interactions between bottom-up and top-down pressures on exploited
  species, however, in many systems data may not be available to
  parameterize such models. Alternatively, models of intermediate
  complexity, which can allow for environmental influence on trophic
  interactions, may prove useful in disentangling the effects of
  top-down and bottom-up pressures {[}@Plaganyi2014{]}.
\item
  An extension of this first challenge is recognizing the need to pursue
  evidence for multiple competing hypotheses to explain observed
  patterns in the dynamics of marine and anadromous fish species. Over a
  century ago, Chamberlin @Chamberlin1965 warned that ``We
  {[}scientists{]} are so prone to attribute a phenomenon to a single
  cause, that, when we find an agency present, we are liable to rest
  satisfied therewith, and fail to recognize that it is but one factor,
  and perchance a minor factor, in the accomplishment of the total
  result.'' Yet, broadening our focus beyond a single hypothesis or set
  of hypotheses focused around a single mechanism remains challenging.
\item
  Another challenge is identifying how relationships between
  environmental processes and demographic rates of marine and anadromous
  fish populations vary over time. Increasingly, evidence suggests that
  both environmental processes and demographic rates of marine and
  anadromous fish populations are non-stationary through time. Similar
  to the implications of spatial non-stationarity observed in this
  thesis, temporal non-stationarity in driver-response relationships
  implies that relationships inferred from past observation may not hold
  today and this non-stationarity may partially explain why
  environmental correlations frequently break down over time.
  Time-varying relationships present a particularly difficult challenge
  for projecting how future climate changes may impact living marine
  resources because using historical data to project future outcomes may
  be misleading. Advancements in quantitative methods, such as
  state-space modeling approaches, may provide a useful starting place
  to find clues on the prevalence of time-varying relationships.
\end{enumerate}

\section{Conclusion}\label{conclusion}

\begin{enumerate}
\def\labelenumi{\arabic{enumi}.}
\setcounter{enumi}{5}
\tightlist
\item
  The utility of improving our understanding of marine ecosystem
  dynamics is dependent on how this understanding can be used to inform
  applied management.
\end{enumerate}
