% General Discussion
% Michael Malick
% 2016-12-29

\chapter[General conclusions]{General conclusions}
\label{ch:conclude}

In this thesis, I have sought to contribute to our broader understanding of how
environmental forcing pathways link climatic and ocean processes to dynamics of
Pacific salmon populations in the Northeast Pacific Ocean. My second and third
chapters apply a cross-system comparative approach to rigorously assess the
evidence for population responses to inter-annual changes in two meso-scale
ocean processes. In chapter \ref{ch:bloom}, I investigate the hypothesis that
vertical ocean transport and subsequent coastal phytoplankton dynamics are
associated with changes in Pacific salmon productivity. In chapter \ref{ch:npc},
I examine an alternative hypothesis that suggests ocean processes driven by
horizontal ocean transport, such as the advection of zooplankton, are equally
important drivers of salmon productivity as processes driven by vertical ocean
transport. My fourth chapter builds on the previous chapters, which each focused
on a single meso-scale ocean process, by estimating the joint effects and
relative strength of multiple environmental forcing pathways on Pacific salmon
dynamics. My fifth chapter applies an interdisciplinary approach to examine
challenges to integrating highly-migratory anadromous fish species into
place-based ecosystem-based management policies and provides practical
recommendations for overcoming the identified challenges. In total, this thesis
further develops our quantitative understanding about how climatic and ocean
processes influence the population dynamics of Pacific salmon and in doing so
contributes to reducing uncertainties about how environmental change impacts
living marine resources.

My thesis makes two substantive contributions to our understanding of how
large-scale climate processes downscale to affect regional and local scale
dynamics of higher-trophic-level species. First, my thesis provides empirical
evidence that the dynamics of higher-trophic-level species respond to forcing
from multiple concurrent environmental forcing pathways. In chapters
\ref{ch:bloom} and \ref{ch:npc}, I show that environmental pathways mediated by
either vertical or horizontal ocean transport processes can impact the dynamics
of higher-trophic-level species. Although I investigate effects of phytoplankton
dynamics and ocean currents on salmon productivity individually, these
meso-scale ocean processes likely influence higher-trophic-levels species
simultaneously. Indeed, my fourth chapter indicates that large-scale climate
processes can impact Pacific salmon year-class strength via multiple
simultaneously operating environmental pathways. For instance, large-scale
climate variability indexed by the Pacific Decadal Oscillation can propagate to
affect regional and local scale dynamics of Pacific salmon through concurrent
pathways that are mediated by different ocean processes. This implies that only
considering a single mechanism may be insufficient to understand how
environmental forcing impacts living marine resources.

Second, my thesis provides empirical evidence that effects of environmental
forcing on higher-trophic-level species can be non-stationary across space. My
second and third chapters indicated that effects of the spring bloom initiation
date and horizontal ocean transport on salmon productivity were dependent on the
latitude of juvenile salmon ocean entry. For example, in chapter \ref{ch:bloom},
I show that effects of the spring bloom initiation date on pink salmon
productivity were opposite in sign for stocks that enter the ocean south and
north of 55.7$^{\circ}$N. Similarly, in chapter three my results indicate that
ocean current patterns are strongly associated with changes in salmon
productivity for stocks that enter the ocean in the southern Northeast Pacific
upwelling domain, but not for stocks that enter the ocean in the northern
downwelling domain, suggesting that different environmental forcing pathways may
drive salmon productivity in northern and southern areas. A practical
implication of this spatial non-stationarity is that relationships inferred from
data in one location may not be applicable to another location.

Collectively, my thesis highlights the need to pursue evidence for multiple
competing hypotheses to explain observed spatial and temporal changes in
demographic rates of exploited species. Over a century ago, T.C. Chamberlin
warned that
\begin{quote}
  We [scientists] are so prone to attribute a phenomenon to a single
  cause, that, when we find an agency present, we are liable to rest satisfied
  therewith, and fail to recognize that it is but one factor, and perchance a
  minor factor, in the accomplishment of the total result.
  \citep[p. 756]{Chamberlin1965}
\end{quote}
Yet, broadening our focus beyond a single hypothesis or set of hypotheses
focused around a single mechanism remains challenging \citep{Hare2014}. As I
show in this thesis, competing environmental forcing hypotheses are not
necessarily mutually exclusive and future research efforts should strive to
understand the cumulative effects and relative importance of a broad range of
environmental forcing hypotheses. In particular, increasing our understanding of
how the relative importance of different environmental pathways changes through
space and time may be an important component to estimating the current and
future impacts of climatic change on coastal ecosystems and the fish populations
they support.

Environmental variability is an intrinsic element of coastal ecosystems and can
have profound impacts on ecosystem structure and function. As climatic processes
and ecosystem dynamics change at unprecedented rates, finding effective
strategies to integrate empirical information about natural and anthropogenic
forcing into management decisions may be critical to maintaining viable and
productive living marine resources. My fifth chapter highlights two potential
strategies that may prove effective, including dynamic in-season management of
commercial fisheries and scenario analysis. Ultimately, uncertainties about the
impacts of changing environmental conditions on living marine resources are
likely to always be present. Effective decision-making in the face of this
uncertainty is vital to preserving the ecological, social, and economic benefits
generated by marine ecosystems.


