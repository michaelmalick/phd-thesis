% Policy Chapter
% Michael Malick
% 2017-02-09

\chapter[Integrating salmon into ecosystem-based management
  policies]{Confronting challenges to integrating Pacific salmon into
  ecosystem-based management policies\footnotemark[1]}
\label{ch:ebm}

\footnotetext[1]{S.P. Cox and M.B. Rutherford are co-authors on this chapter,
  which is in preparation for submission to a journal.}


\section{Abstract}

Ecosystem-based management is an increasingly prominent paradigm for the
management of living marine resources with a focus on maintaining ecosystem
level properties and processes in the presence of anthropogenic and natural
disturbances. Although highly migratory marine and anadromous fish species often
disproportionately contribute to the structure and function of ecosystems,
incorporating these species into ecosystem-based management policies remains
challenging because they spend a considerable portion of time outside the
boundaries that define a particular management area. In this paper, we use two
case studies to examine how challenges arising from ecosystem openness,
imperfect information, and ecosystem complexity can impede efforts to integrate
highly migratory Pacific salmon (\emph{Oncorhynchus} spp.) into ecosystem-based
management policies. Our analysis highlights three main factors that hinder more
effective integration: (1) uncertainties about the impacts of human activities
and ecological processes that occur in geographically distant jurisdictional
areas or at spatial scales larger than the ecosystem-based management area, (2)
spatial asymmetries in the distribution of costs and benefits associated with
management decisions (i.e., positive or negative externalities), and (3) static
management policies that prevent updating management decisions in a timely
manner when ecosystem conditions change or new information becomes available.
Given these factors, we suggest two potential strategies to address migratory
challenges. First, we recommend the creation of an international ecosystem
synthesis group to facilitate the collection, analysis, and dissemination of
ecological, social, and policy information across national and other
jurisdictional boundaries. Second, we recommend the expanded use of dynamic
in-season management policies that allow rapid updating of management decisions
based on evolving information about ecosystem conditions. Our findings further
suggest that ecosystem-based management policies need to explicitly account for
mismatches in the scale at which ecosystem services are provided by highly
migratory species and the scale at which human activities and natural processes
impact those services.



\section{Introduction}

Over the past few decades, ecosystem-based management (EBM) has emerged as a
leading paradigm for the management of living marine resources in many parts of
the world, with a focus on maintaining ecosystem level properties and processes
(e.g., nutrient cycles and trophic interactions) in the presence of
anthropogenic and natural disturbances \citep{Engler2015, Skjoldal2008,
Fletcher2008, Constable2011, Olsson2008}. A key principle underlying EBM is that
distinct boundaries demarcate the management area, with the boundaries ideally
being chosen based on the ecological properties of the system rather than
existing socio-political boundaries \citep{Long2015, Engler2015}. However, in
addition to the difficulties of managing across existing jurisdictional
boundaries, most marine ecosystems cannot be easily discretized into manageable
units because ecosystem boundaries remain open, i.e., organisms, energy, or
matter can move across specified ecosystem boundaries \citep{ONeill2001}.

Ecosystem openness presents a key challenge to integrating highly migratory
species into EBM policies because these species frequently move across ecosystem
and jurisdictional boundaries. Indeed, highly migratory marine and anadromous
fish species provide critical connectivity between geographically distant
ecosystems and these species often play a disproportionately large role in the
structure and function of ecosystems by translocating organic material,
nutrients, and energy \citep{Lundberg2003, Heupel2015}. However, factors
external to the management area may strongly affect the status and contributions
of species that spend a considerable portion of their life cycle outside the
human defined boundaries of an ecosystem. In particular, this movement across
management and jurisdictional boundaries may lead to mismatches between the
scale of management and the biology of a migratory fish stock \citep{Cash2006a,
Epstein2015}. To overcome these potential mismatches, some EBM initiatives are
using very large spatial boundaries (e.g.,~large marine ecosystems) in an
attempt to capture full ecosystems; however, even these larger EBM initiatives
are often not successful at including the full life cycle of many highly
migratory species, such as Pacific salmon (\emph{Oncorhynchus} spp.;
\citealp{Field2006a, Sherman1999, Wang2004}). Jurisdictional challenges
make it unrealistic to expect that management will be able to scale up to the
necessary level to capture the full spatial range of highly migratory species
\citep{Cowan2012, Lascelles2014}. Therefore, in this study, we focus on the
challenges associated with accounting for and incorporating highly migratory
species into existing local and regional scale EBM initiatives.

A second major source of challenges to integrating highly migratory marine
species into EBM is the quality of the available information. In order to
practice EBM, we must identify and estimate cumulative impacts from a diverse
suite of physical, biological, and human influences on marine ecosystems across
multiple spatio-temporal scales \citep{Lascelles2014, Halpern2008a}. However,
our ability to study and manage these cumulative impacts is limited because we
often have imperfect information (e.g., non-existent data or data with
potentially large observation error) about the status of particular ecosystem
components or the drivers of ecosystem dynamics. For instance, information is
usually lacking to separate particular cause and effect relationships because we
tend to study only the most economically dominant components and drivers of
ecosystems. A third source of challenges is that even when good information is
available, ecosystem complexity (i.e., numerous ecological processes interacting
in multiple, and often non-linear ways) can limit our ability to recognize the
drivers of patterns and processes in natural systems because underlying
cause-effect relationships may be spontaneous and non-stationary over space and
time \citep{Hsieh2005a, Burkett2005a, Scheffer2001a}. In other words, new
cause-effect relationships may emerge while others disappear by the time we
accumulate enough information to understand any particular one
\citep{Myers1998b}.

These three features---ecosystem openness, imperfect information, and ecosystem
complexity---are clearly evident in coupled marine-terrestrial ecosystems along
the west coast of North America, where policy-makers are struggling to integrate
highly migratory Pacific salmon into EBM policies. In this paper, we first
examine how challenges arising from openness, imperfect information, and
ecosystem complexity can impede efforts to integrate highly migratory Pacific
salmon into EBM policies and then we explore potential strategies that could be
implemented to overcome these challenges. Although we focus on two case studies
involving Pacific salmon, the challenges and strategies discussed appear to be
widely applicable to other highly migratory marine or anadromous fish species
that move among jurisdictions and ecosystems.



\section{Salmon and ecosystem-based management}

Pacific salmon typically require a continuum of ecosystems spanning hundreds or
thousands of kilometers to complete their anadromous life cycle. The combined
extent of these ecosystems ranges from diverse headwaters of large river systems
to the pelagic ocean environment of the North Pacific Ocean, Gulf of Alaska, and
Bering Sea. Within these varied ecosystems, Pacific salmon provide numerous
ecosystem services, including contributing to critical ecosystem functions
(e.g., nutrient cycles in freshwater ecosystems), providing economic and food
provisioning services to commercial, subsistence, and recreational fishing
sectors, and contributing to social and cultural dimensions of coastal North
Pacific communities.

Declines in Pacific salmon abundance over the past few decades highlight their
importance in numerous regions along the west coast of North America. For
example, the Yukon and Kuskokwim River regions in western Alaska were declared
economic disaster areas following declining returns of adult chinook salmon
(\emph{O. tshawytscha}) and chum salmon (\emph{O. keta}) throughout the 1990s
and 2000s. Conflicts continue in these regions as government managers and
stakeholders struggle to allocate the diminished adult salmon returns among
fishery sectors \citep{Ebbin2002, Ebbin2003}. Similarly, in southern British
Columbia (BC), declining adult abundances of Fraser River sockeye salmon
(\emph{O. nerka}) throughout the 2000s resulted in limited opportunities for
commercial, subsistence, and recreational harvest and prompted a federal
judicial inquiry (the Cohen Commission) involving government officials,
scientists, and other stakeholders to determine the causes of the declines
\citep{Cohen2012}. Along the west coast of the United States, many salmon
populations have been extirpated and several others are listed as threatened or
endangered under the United States Endangered Species Act, severely reducing the
ecosystem services provided by Pacific salmon in this region \citep{Nehlsen1991,
NMFS2015, Williams2011b}.

Declining adult salmon abundances have widespread effects on ecosystem structure
and function that can detrimentally affect other valued components of
ecosystems, including economically important species. For example, migration of
adult salmon into freshwater provides a large influx of marine derived nutrients
to aquatic and terrestrial ecosystems \citep{Claeson2006, Johnston2004,
Chaloner2002}. This large subsidy of nutrients (e.g., nitrogen and phosphorous)
and organic matter (e.g., organic carbon) gets incorporated into multiple levels
of the food chain, providing critical connectivity between marine and
terrestrial ecosystems \citep{Claeson2006, Johnston2004}. Migrating and spawning
salmon, as well as post-spawning carcasses, are also a key food resource for
numerous predators and scavengers in marine, freshwater, and terrestrial systems
including birds, bears, whales, seals, and sea lions \citep{Ford2016,
Olesiuk1993a, Trites2007a, Hilderbrand1999b}. In some cases, such as orca whales
(\emph{Orcinus orca}) in the Salish Sea, adult chinook salmon are the primary
diet item for most of the year \citep{Ford2016}.

Maintaining ecosystem level properties such as nutrient cycles and trophic
linkages requires a holistic, ecosystem-based approach to management of living
marine resources because of the complex and non-linear connections among
ecosystem components \citep{Engler2015}. Indeed, several state and federal
agencies responsible for managing marine resources along the west coast of North
America have either started to implement EBM policies or are in the process of
developing EBM policies. For example, in 2007 the state of Washington (WA)
created the Puget Sound Partnership, a public-private partnership made up of
government agencies, scientists, and private groups, with the goal of
implementing an ecosystem-based management approach to resource use in Puget
Sound, WA \citep{Ruckelshaus2009, Samhouri2011a}. Similarly, in BC, the Marine
Plan Partnership for the North Pacific Coast (MaPP)---a co-led partnership
between the provincial government and First Nations---was created in 2011 to
facilitate EBM efforts along the BC North Coast. At the federal level, the
United States National Marine Fisheries Service recently released a policy
directive that establishes ``a framework of guiding principles to enhance and
accelerate the implementation of EBFM {[}ecosystem-based fisheries
management{]}'' \citep{NMFS2016}.

A common element of these EBM initiatives is that they are largely confined
within existing socio-political boundaries. For instance,
\citet{Ruckelshaus2009} notes that the ``Puget Sound Partnership is focusing on
ecosystem-based management of the marine waters and lands within Washington
State, while recognizing that the entire ecosystem spans Washington State and
British Columbia.'' For highly migratory species, such arbitrarily defined
ecosystem boundaries can result in a mismatch between the scale of the
management or jurisdictional area and the biology of the fish stock. For
example, an adult salmon returning to WA State from the Gulf of Alaska may pass
through as many as five jurisdictional areas in the marine environment,
including international waters, federally managed waters in Canada and the
United States, and state managed waters in Alaska and WA.

Spatial mismatches, arising from the limited spatial extent of the EBM area, the
openness of the ecosystems, and the large migratory range of Pacific salmon,
create challenges for integrating salmon into EBM policies. In particular,
natural processes and human activities occurring either at locations that are
geographically distant from the EBM area (e.g., targeted fisheries and bycatch
in non-targeted fisheries; Fig. \ref{fig:ebm:1}a) or at spatial scales larger than
the EBM area (e.g., environmental change and interactions between hatchery and
wild salmon; Fig. \ref{fig:ebm:1}b) may strongly impact salmon populations and
the effectiveness of local EBM policies. In the following section, we detail two
of these broad scale impacts, including fisheries that intercept salmon at
geographically distant locations and competition between wild and hatchery
salmon for limited food resources throughout the North Pacific. In doing so, we
highlight how imperfect information and ecosystem complexity contribute further
to the difficulties in overcoming these challenges.

\begin{figure}[htbp]
  \centering \includegraphics[scale=0.50]{./5_policy/figures/space.pdf}
  \caption[Schematic of two spatial mismatches between ecosystem-based
           management areas and the migratory range of Pacific
           salmon]{Schematic of two spatial mismatches between ecosystem-based
           management areas and the migratory range of Pacific salmon. The left
           panel (a) indicates a mismatch due to geographic distance between two
           interception fisheries and the ecosystem-based management area. Black
           dashed arrows in panel (a) indicate the migratory direction of adult
           salmon. The right panel (b) indicates a mismatch due to interactions
           between wild and hatchery salmon occurring at a larger spatial scale
           than the ecosystem-based management area. The black dashed arrows in
           panel (b) indicate the marine migratory route of wild salmon that
           originate within the ecosystem-based management area.}
  \label{fig:ebm:1}
\end{figure}



\section{Migratory challenges}

\subsection{Interception harvest of highly migratory salmon}

A common policy problem for migratory species occurs when commercial fisheries
intercept a species (or population) at multiple points along its migration route
\citep{Lascelles2014}. These interception fisheries pose a particularly
difficult challenge for local or regional scale EBM policies because the
harvesting is often outside the boundaries of the EBM decision-making area and
may be spread across numerous management jurisdictions. Further, interception
fisheries frequently result in externalities (both positive and negative) where
an asymmetric distribution of costs and benefits occurs from management
decisions made in up-migration or down-migration areas \citep{Scherer1990}. For
instance, users of the resource in down-migration areas often bear costs or
benefits of management decisions made in up-migration areas (e.g., decisions to
increase or decrease harvest rates) because the migratory species passes through
the up-migration area prior to entering the down-migration area. For Pacific
salmon, the down-migration area often represents the source location for a
population, i.e., the natal spawning locations, and harvest in up-migration
areas can reduce both the number of salmon available for harvest in the
down-migration area and the number of eggs deposited by spawning adults.

Adult and sub-adult Pacific salmon are harvested in coastal waters throughout
their migratory range by both targeted fisheries and as bycatch in fisheries
targeting other species. Targeted harvesting frequently occurs in mixed-stock
fisheries where salmon stocks with distant origins co-mingle with local stocks
and are harvested jointly. For example, the Southeast Alaska (SEAK) chinook
salmon troll fishery, the largest commercial chinook salmon fishery in SEAK,
harvests salmon originating from SEAK, BC, WA, and Oregon, with more
than 80\% of the chinook salmon catch originating outside SEAK
\citep{Templin2004}.

The challenges posed by interception fisheries for Pacific salmon have been
recognized for over a century and the bi-lateral Pacific Salmon Treaty between
the United States and Canada specifically deals with interception of salmon from
distant origins \citep{Knight2000, Noakes2005b}. In particular, the `State of
Origin' principle within the treaty states that the primary harvest rights and
burden of conservation of Pacific salmon stocks are assigned to the
jurisdictional area where the stock originates. However, such policies developed
to achieve equity in salmon harvesting across the migratory range of adult
salmon can be unsuccessful for numerous reasons including (1) inability to
selectively harvest certain salmon stocks within mixed stock fisheries because
information is lacking about which salmon stocks are currently migrating through
a particular harvest area, and (2) over harvest (or under harvest) of salmon in
up-migration areas because of an incomplete understanding of the complex
ecosystem dynamics that drive variability in adult salmon returns.

This latter reason is exemplified by chinook salmon fisheries management along
the west coast of North America. Under the Pacific Salmon Treaty the SEAK
chinook salmon troll fishery is regulated using aggregate abundance-based
management where the total allowable catch in a given year is set based on
pre-season abundance forecasts estimated by the Chinook Technical Committee
\citep[Annex IV, Chapter 3, Section 6]{PST2014}. The pre-season
abundance forecasts, however, are often not accurate representations of actual
total run size due to difficulties in forecasting salmon productivity. In other
words, salmon productivity is often linked to natural and anthropogenic drivers
in complex and nonlinear ways and we often lack information to identify and
predict particular cause and effect relationships in a timely manner
\citep{Peterman2012, Malick2016a, Myers1998b}. For instance, from 2009 to 2013,
the pre-season chinook salmon abundance index for SEAK overestimated abundance
in all but one year (i.e., 2013), resulting in disproportionately large chinook
harvests in SEAK (Fig. \ref{fig:ebm:2}; \citealp{CTC2015a}). In these years,
stakeholders in down-migration areas, e.g., Puget Sound, bear the cost of
uncertainty associated with the pre-season forecasts in the form of reduced
adult salmon returns and the ecosystem services they provide. Moreover, the
management of chinook salmon under the Pacific Salmon Treaty is largely
implemented using static reference points that are estimated to produce the
maximum sustained yield for a stock (or group of stocks) without consideration
of other ecosystem services provided by salmon \citep{CTC2015b}. By focusing on
a single ecosystem service, in this case providing food and associated economic
benefits for humans, the other ecosystem roles of salmon in local or distant
areas are discounted or ignored, such as providing prey for orca whales and
other marine mammals and providing connectivity between marine and freshwater
ecosystems in the form of marine derived nutrients.

\begin{figure}[htbp]
  \centering \includegraphics[scale=0.9]{./5_policy/figures/ctc.pdf}
  \caption[Difference between pre-season and post-season allowable catch
           estimates for three North American chinook salmon
           fisheries]{Difference between pre-season and post-season allowable
           catch estimates for three North American chinook salmon fisheries
           managed using aggregate abundance-based management outlined in the
           Pacific Salmon Treaty. Positive values (red) indicate the pre-season
           estimate overestimated the allowable catch in a given year and
           fishery, whereas negative values (blue) indicate the pre-season
           estimate underestimated the allowable catch. All values are in
           thousands of adult chinook salmon. Values within each panel give the
           maximum and minimum relative error of the pre-season estimate
           compared to the post-season estimate.}
  \label{fig:ebm:2}
\end{figure}

At the same time, the costs of management decisions made within down-migration
areas may disproportionately be borne by stakeholders in down-migration areas
compared to stakeholders in up-migration areas. For example, the `State of
Origin' principle puts the responsibility for salmon conservation on the
jurisdictional area where the stock originates. However, efforts to restore
ecosystems or rebuild salmon abundances in down-migration areas may result in
positive externalities, in that the stakeholders in the down-migration area bear
the full costs of the rebuilding or restoration efforts, but users throughout
the migratory range of the salmon population gain the potential benefits of
increased salmon abundances.


\subsection{Salmon hatcheries}

Releasing hatchery-reared juvenile Pacific salmon into freshwater and marine
ecosystems is a common management practice throughout the North Pacific Rim. The
primary objective of these hatcheries is to augment wild salmon production for
either conservation purposes or to supplement commercial and other harvests.
Despite widespread debate about the effectiveness of hatchery releases to
increase adult salmon returns \citep{Hilborn2000a, Wertheimer2001a,
Hilborn2001a}, releases of juvenile Pacific salmon from hatcheries located
around the North Pacific Rim have increased ten fold over the past five decades,
exceeding seven billion salmon released in 2015 (Fig. \ref{fig:ebm:3}). These
hatchery salmon co-mingle with wild salmon throughout their marine migratory
range and compete with wild salmon and other species that feed at similar
trophic levels for a limited common-pool of prey resources throughout the North
Pacific Ocean.

\begin{figure}[htbp]
  \centering \includegraphics[scale=0.9]{./5_policy/figures/hatchery.pdf}
  \caption[Hatchery releases of juvenile Pacific salmon from North Pacific Rim
           nations]{Hatchery releases of juvenile Pacific salmon for (a) all
           North Pacific Rim nations combined and (b) each North Pacific Rim
           nation individually. For the United States, releases from Alaska are
           shown separate from those in the lower mainland because of the
           order-of-magnitude differences in total releases between the two.}
  \label{fig:ebm:3}
\end{figure}

Because Pacific salmon migrate and feed across broad regions of the North
Pacific Ocean and Bering Sea, competition occurs between wild and hatchery
salmon originating in different jurisdictions, nations, and continents. For
example, wild sockeye salmon originating in Bristol Bay, Alaska compete with
abundant hatchery pink salmon released from hatcheries in Russia and Japan
during their second and third years of ocean residency \citep{Ruggerone2003a}. A
major consequence of competition between hatchery and wild salmon across their
marine migratory range is density dependent growth, which can result in (1)
reduced age-specific body sizes of wild adult salmon, and (2) declines in wild
salmon population productivity \citep{Ruggerone2015, Ruggerone2003a}. For
instance, widespread declines in age-specific adult body sizes observed in many
salmon populations since the 1980s have been attributed to density dependent
growth in the marine environment \citep{Ricker1981a, Ishida1993a, Pyper1999a}.
Indeed, the large increases in hatchery releases, combined with the strong
evidence for reductions in adult body sizes and stock productivities, have led
to concerns about carrying capacity limitations of the North Pacific Ocean for
Pacific salmon \citep{Pearcy1999a, Nielsen2009a}.

These ocean-basin scale effects of hatchery salmon on wild salmon present a
challenge for EBM policies because they indicate that management actions
implemented for salmon originating in one jurisdictional area or nation can
affect salmon populations originating in distant jurisdictions or nations. In
other words, declines in wild stock productivity or age-specific adult body
sizes due to competition with hatchery salmon released from multiple nations can
strongly influence the provisioning of ecosystem services provided by Pacific
salmon within an EBM area. In particular, the ecological effects of hatchery
salmon on wild salmon populations can (1) reduce harvest opportunities for
stakeholders of affected stocks, (2) influence ecosystem structure and function
within an EBM area by affecting trophic interactions and connectivity between
marine and terrestrial ecosystems, and (3) inhibit conservation efforts within
an EBM area to recover threatened or endangered salmon populations
\citep{Ruggerone2003a, Nielsen2009a}.

Addressing the challenges posed by the ocean-basin scale effects of hatchery
salmon on wild salmon populations would likely require the creation of a new
international agreement, organization, or other institution to either regulate
hatchery releases or alter the incentives associated with releasing juvenile
hatchery salmon \citep{Holt2008b}. The common-pool prey resources that hatchery
and wild salmon compete for are both rivalrous and non-exclusive, i.e., the
consumption of prey by hatchery salmon reduces food availability for wild salmon
and it is difficult to exclude nations or agencies from releasing juvenile
hatchery salmon. As with many common-pool resource problems, this creates a
disincentive for a particular agency or nation to reduce hatchery releases
because that agency or nation receives most of the benefits from the releases,
in terms of harvest of returning adult hatchery salmon, but only bears a
fraction of the costs of deterioration of the prey resources, which are spread
across multiple nations \citep{Holt2008b}.

Even if such an international institution were created to manage hatchery
releases, determining an ecologically acceptable level of total hatchery
releases each year would be challenging because the productivity levels of the
ecosystems that comprise the North Pacific are non-stationary, varying on
inter-annual and inter-decadal scales corresponding to large-scale shifts in the
climate \citep{Hare1999a, Chavez2003a}. For example, an abrupt and unanticipated
reversal of the Pacific Decadal Oscillation in 1976/77 precipitated an
ecological regime shift that resulted in the species composition of the North
Pacific shifting from a crustacean-dominated system to a gadid- and
flatfish-dominated system \citep{Mantua1997a, Anderson1999a, Mueter2000a}.
Similarly, as the climate warms, the frequency of extreme events in marine
ecosystems (e.g., marine heat waves) is expected to increase, which can
influence hatchery-wild interaction by altering food resource availability
\citep{Jentsch2007, DiLorenzo2016}. For instance, between the winters of
2013--2014 and 2014--2015 offshore ocean temperatures in the Northeast Pacific
Ocean were anomalously warm (nearly 2.5$^{\circ}$C above the long term average)
resulting in widespread changes in ecosystem dynamics including reductions in
large lipid-rich prey resources for juvenile salmon \citep{Bond2015,
DiLorenzo2016}.



\section{Potential strategies to address migratory challenges}

A common element of the challenges discussed here is a mismatch between the
scale of management and the migratory life history of Pacific salmon, with the
challenges being exacerbated by imperfect information about cause and effect
relationships and complex non-stationary ecosystem dynamics. The two case
studies highlight three main factors that hinder more effective integration of
highly migratory species into EBM policies, including (1) uncertainties about
the impacts of human activities and ecological processes that occur at distant
locations or at scales larger than the EBM area, (2) spatial asymmetries in the
distribution of costs and benefits associated with management decisions, and (3)
static management policies that prevent updating management decisions in a
timely manner when ecosystem conditions change. In this section, we present two
strategies that could help overcome these problems: increased cross-scale
synthesis of information; and expanded near real-time data analysis of ecosystem
dynamics.


\subsection{Cross-scale synthesis of information}

Our first proposed strategy is the collection and synthesis of ecological,
economic, and policy information across jurisdictional areas along with the
dissemination of strategic management advice that is informed from this
broad-scale synthesis. Such advice could at least partially address two of the
factors hindering integration of highly migratory species into EBM policies:
uncertainty about impacts of processes outside the EBM area and asymmetrical
distribution of costs and benefits. Currently, there are numerous organizations
and programs that conduct ecological, economic, and policy research on Pacific
salmon at the local and regional scales. For example, the Arctic-Yukon-Kuskokwim
Sustainable Salmon Initiative, the National Oceanic and Atmospheric
Administration's Southeast Alaska Coastal Monitoring Program, and the Salish Sea
Pacific Salmon Marine Survival Project all collect empirical data on Pacific
salmon and other ecosystem components to increase knowledge about how ecosystem
dynamics and human activities influence Pacific salmon populations. These
research efforts are largely used to provide scientific and management advice at
the local and regional spatial scales. However, the limited spatial extent of
these research efforts omits social and ecological forces that occur at distant
locations or at scales larger than the management area, which can influence the
ecosystem services generated by Pacific salmon within the local or regional
management area, and therefore, should be taken into account in EBM.

The need to collect and synthesize information at larger scales and across
jurisdictional boundaries is not unique to Pacific salmon, and ecosystem working
groups have become increasingly common elements of international management
organizations because of their effectiveness at synthesizing information across
multiple jurisdictional areas \citep{Engler2015, Lascelles2014}. For example,
the International Commission for the Conservation of Atlantic Tunas, the
Inter-American Tropical Tuna Commission, and the Northwest Atlantic Fisheries
Organization have all created ecosystem working groups as a key part of their
research efforts. These working groups conduct cross-scale ecological and social
research that is used to provide strategic management advice to multiple
smaller-scale jurisdictional areas. For instance, the Northwest Atlantic
Fisheries Organization collects and synthesizes data from member states and, in
turn, disseminates strategic management advice (e.g., identifying vulnerable
marine ecosystems).

Following these examples, we recommend the creation of an international
ecosystem synthesis group to facilitate the collection and analysis of
ecological, economic, and policy information across jurisdictional areas in the
North Pacific and to disseminate strategic management advice to local and
regional scale EBM programs (Fig. \ref{fig:ebm:4}). The creation of an
international synthesis group would not necessarily require a new international
organization, but instead could be implemented through the modification of an
existing organization. In particular, the North Pacific Anadromous Fish
Commission (NPAFC), the organization charged with implementing the multi-lateral
Convention for the Conservation of Anadromous Stocks, would be an obvious
starting point for two reasons. First, the NPAFC has an existing mandate to
conduct scientific research ``for the purpose of the conservation of anadromous
stocks including, as appropriate, scientific research on other ecologically
related species'' \citep{CCAS1992}. Second, the NPAFC already has an established
structure for international cooperation among North Pacific Rim nations along
with established political relationships that would enable the dissemination of
strategic management advice.

\begin{figure}[htbp]
  \centering \includegraphics[scale=0.50]{./5_policy/figures/integrate.pdf}
  \caption[Schematic of hypothesized interactions between multiple
           ecosystem-based management areas and an international ecosystem
           synthesis group]{Schematic of hypothesized interactions between
           local and regional scale ecosystem-based management areas and an
           international ecosystem synthesis group. Solid arrows (red) represent
           the flow of data and information from ecosystem-based management
           programs to the international synthesis group, whereas dashed arrows
           (blue) represent the flow of strategic management advice from the
           international synthesis group to specific ecosystem-based management
           programs.}
  \label{fig:ebm:4}
\end{figure}

Based on the case studies provided here, we further suggest that a key focal
point of this synthesis group should be identifying critical uncertainties and
risks to ecosystem services provided by Pacific salmon from an array of human
activities and natural processes across their migratory range including current
and proposed management actions. Uncertainties arise in social and ecological
systems for numerous reasons including natural variability, imperfect
information about complex social and ecological processes, vague management
objectives, and limited control over fisheries and other human activities
\citep{Peterman2004a}. For EBM policies, these widespread uncertainties are
important to identify and quantify because they create risks for management
agencies, ecosystems, and communities that rely on salmon for economic and
cultural prosperity. Further, understanding how these uncertainties may affect
the outcomes of current and proposed management actions across multiple
spatio-temporal scales is a necessary component of implementing policies that
are robust to a wide range of potential future changes in the dynamics of
social-ecological systems.

Ecological risk assessment represents one well established framework for
identifying critical uncertainties and quantifying risks within
social-ecological systems. The primary goal of risk assessment is to estimate
``the magnitudes of adverse consequences that will arise from events that are
uncertain, and the chances (i.e.~probabilities) of those events and their
consequences occurring'' \citep{Peterman2004a}. Numerous risk assessment
techniques might be used successfully to assess uncertainties and risks to
ecosystem services provided by highly migratory species. For example,
\citet{Smith2007} and \citet{Hobday2011} outline a technique,
referred to as Ecological Risk Assessment for the Effects of Fishing (ERAEF),
that is based on a hierarchy of assessments in which each level in the hierarchy
acts as a screening process. Low risk threats are screened out at lower levels
in the hierarchy and higher risk threats are moved up the hierarchy for more
detailed evaluation. For EBM policies that need to integrate highly migratory
species, the ERAEF technique is appealing for at least two reasons. First, the
ERAEF method can be used to assess a wide range of threats or impacts and,
because the first level of the assessment hierarchy is primarily based on expert
opinion, threats or impacts with limited empirical information can be included
in the assessment. Second, the ERAEF method explicitly incorporates a
precautionary approach by assuming that a threat or impact is high risk unless
data or information are available that indicate otherwise.

In addition to risk assessment, scenario analysis is an increasingly popular
method for analyzing risks associated with current and proposed management
actions. Using scenario analysis and simulation modeling, researchers can
analyze multiple potential ecosystem or management outcomes given a set of
alternative management actions. These analyses can be used to inform managers
and decision-makers about the potential of different management actions to
impact social-ecological systems. Within fisheries science, the most widely used
framework for scenario analysis is management strategy evaluation where ``models
are used to simulate the behavior of ecosystems and provide the ability to
forecast changes in ecosystem state as a consequence of management scenarios and
decision rules'' \citep{Levin2009}. Scenario analysis in general, and management
strategy evaluation in particular, can thus be used to help identify policies
and management actions that have the greatest potential to achieve stated
management objectives, such as maintaining connectivity between marine and
terrestrial ecosystems. Further, with simulation modeling, a broad range of
potential management actions that span multiple spatial scales and
jurisdictional areas can be considered within a scenario analysis. For example,
\citet{Dorner2009a} and \citet{Dorner2013} used management strategy evaluation
to assess how alternative Pacific salmon stock assessment models and management
actions across several spatial scales perform under a wide variety of climate
change scenarios.

Although synthesis of ecological information, such as how climate change may
affect salmon food availability and hatchery-wild salmon competition in coastal
and pelagic ecosystem, would be a key function of the proposed synthesis group,
synthesis of economic and policy information would also be important. In
particular, the group could provide advice on how best to address the positive
and negative externalities arising from the migratory life histories of salmon
(i.e., the asymmetric spatial distribution of costs and benefits). The synthesis
group should seek to better understand how different policy instruments and
management actions could alter the incentives and behavior of stakeholders
across the migratory range of Pacific salmon. As discussed earlier, ecological
restoration efforts within a localized EBM area can result in positive
externalities if the benefits of the restoration accrue to stakeholders outside
the EBM area. This is a frequent occurrence for Pacific salmon stocks that are
harvested in interception fisheries. Policy instruments that could be
implemented to overcome this positive externality include side-payments from
stakeholders in distant locations that benefit from the restoration efforts, a
tax on the harvest of salmon originating in a particular area, or assigning
property rights to the resource (e.g., individual transferable quotas).
Determining the potential of these alternative policy instruments to achieve
stated policy objectives and how they may influence incentives of stakeholders
across broad geographic regions is a necessary component of overcoming the
challenges of effectively integrating Pacific salmon into EBM policies.


\subsection{Near real-time analysis of ecosystem dynamics}

Our second proposed strategy relies on the increased availability of near
real-time data analysis on ecosystem conditions and migration timing, which can
be used to inform in-season management decisions. Despite the complex dynamics
of ecosystems, several of the management prescriptions outlined in the Pacific
Salmon Treaty are static within a given year. For instance, in the aggregate
abundance-based management policy used to manage SEAK chinook fisheries, the
quota is set before the fishing season and is not updated regardless of the
actual timing or strength of the salmon runs. Similarly, since about 1990 the
magnitude of releases of juvenile salmon from hatcheries around the North
Pacific Rim has remained nearly constant, despite annual and decadal changes in
the productivity of pelagic and coastal ecosystems. In many cases, however,
information on current ecosystem conditions and migration timing can be
collected, analyzed, and disseminated in near-real time to help align management
decisions with current ecosystem conditions. Indeed, dynamic in-season
management, a management strategy that uses near real-time information to better
match management decisions with the state of currently observed ecosystem
conditions, is becoming increasingly popular for management and conservation of
highly migratory species \citep{Maxwell2015, Lewison2015}.

At the local scale (i.e., the scale of individual salmon fisheries), Pacific
salmon fisheries management has pioneered several methods for implementing
intensive in-season management policies. In Alaska, management of most marine
fisheries for Pacific salmon are adapted in-season using time and area closures
that are informed by near real-time data collected on the number of adult salmon
that escape the fisheries and move into freshwater to spawn. For instance, in
2013, the Alaska Department of Fish and Game issued 48 in-season management
decisions (i.e., emergency orders) regarding salmon fisheries around Kodiak
Island, Alaska, which allowed managers to use near real-time data to update
decisions based on pre-season forecasts and better match fisheries harvest with
the strength of salmon runs \citep{Jackson2013}. Using in-season information on
the number of adult spawners for management of salmon fisheries in this way,
however, requires that salmon harvesting occur concurrently with salmon entering
their natal rivers. For those chinook salmon in the SEAK fishery that originate
from natal rivers in BC, WA, or Oregon, information on salmon escapements to
those rivers may not be available until after harvest in Alaska has taken place,
so it would not be feasible to adaptively change quotas in-season based on
escapement.

It may be possible, however, to use in-season data about migration routes and
timing to increase the selective harvest of local origin and highly productive
stocks within mixed-stock salmon fisheries such as the SEAK chinook fisheries.
For example, management of several high value salmon fisheries (e.g., Bristol
Bay and Fraser River sockeye salmon fisheries) employ test fisheries and genetic
sampling to determine the population composition of the salmon currently
migrating through a particular location, which allows managers to close a
fishery if large proportions of threatened or low productivity populations are
currently migrating through the commercial fishing area \citep{Dann2013}.
Similar test fisheries and the subsequent near-real time analysis of genetic
sampling could be expanded to SEAK (as well as other areas that intercept salmon
bound for distant natal rivers) to determine which populations are migrating
through the commercial fishing area and potentially reduce the interception of
distance origin or low productivity stocks by informing in-season management
decisions about time or area closures.

In addition to expanding dynamic in-season management for salmon fisheries,
dynamic management could also be applied to management of juvenile salmon
releases from hatcheries. Historically, there has been minimal effort to modify
the magnitude of salmon releases in response to changing ocean conditions, which
results in either salmon hatcheries gradually increasing total salmon releases
or releasing approximately the same number of salmon each year regardless of
whether ocean conditions are favorable for salmon survival or not
\citep{Pearsons2010a}. While a consistent release strategy may allow hatcheries
to plan operations more efficiently, this could be detrimental to marine
ecosystems in general and wild Pacific salmon specifically. For example,
releasing a large number of hatchery-reared juvenile salmon into coastal
ecosystems when food availability is low may lead to high levels of competition
between hatchery and wild salmon for limited food resources. Instead, release
strategies could be updated on an annual basis in response to new information
about ecosystem conditions \citep{Pearsons2010a, Peterman1983a}. For instance,
the previously proposed ecosystem synthesis group could use an ecosystem model
of the North Pacific within a management strategy evaluation framework to
estimate how different hatchery release levels could impact wild salmon
populations given current ecosystem conditions.



\section{Conclusions}

We used two case studies from the Pacific salmon literature to examine how three
features of social-ecological systems---ecosystem openness, imperfect
information, and ecosystem complexity---present challenges to integrating
highly migratory Pacific salmon into EBM policies. The first example, chinook
salmon interception fisheries in SEAK, showed that human activities (in this
case, commercial fishing) in one location can influence the supply of ecosystem
services and incentives to conserve salmon populations in distant locations.
Stakeholders and managers in EBM areas located in down-migration areas from the
interception fisheries bear the costs of uncertainty (imperfect information
about ecosystem dynamics) associated with management decisions made in
up-migration areas. The second example, competition between hatchery and
wild salmon for a limited common-pool of prey resources, showed that processes
that occur over larger spatial scales than an EBM area can influence the
provisioning of ecosystem services within the EBM area. In this case, the
aggregate releases of juvenile salmon from North Pacific Rim nations resulted in
reductions in ecosystem services provided by salmon within a localized EBM area
due to reduced age-specific body sizes and productivity of wild salmon stocks.
This example further indicated that overcoming this collective action problem
would likely require estimating an acceptable level of aggregate hatchery
releases for current ecosystem conditions, which is impeded by large
uncertainties about the complex non-stationary dynamics of the ecosystems that
comprise the North Pacific Ocean.

Confronting the challenges arising from ecosystem openness, imperfect
information, and ecosystem complexity will require cooperation across
socio-political borders; unilateral management actions are unlikely to resolve
the challenges because many impacts occur outside the bounds of a given EBM
area. Our examination of the case studies suggested two potential strategies to
address these challenges. First, we recommend the creation of an international
ecosystem synthesis group that is charged with aggregating and analyzing
ecological, economic, and policy information from local and regional scale
research and management areas, along with disseminating strategic management
advice based on this synthesis. Second, we recommend the expansion of dynamic
in-season management practices to better match management actions to current
ecosystem conditions, which would require increased near real-time data
collection and analysis. Our findings further indicate that no single solution
is likely to overcome the challenges associated with integrating highly
migratory species into local or regional scale EBM policies; instead a
variety of strategies will likely need to be implemented to maintain key
ecosystem properties in the presence of human and natural disturbances. Finally,
our results suggest that ecosystem-based management policies should explicitly
account for mismatches in the scale at which ecosystem services are generated by
highly migratory species and the scale at which human activities and natural
processes impact those services.
