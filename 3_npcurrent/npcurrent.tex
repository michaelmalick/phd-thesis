% North Pacific Current Chapter
% Michael Malick
% 2016-11-25

\chapter[Horizontal ocean transport and salmon productivity]{Effects of the
  North Pacific Current on productivity of 163 Pacific salmon
  stocks\footnotemark[1]}
\label{ch:npc}

\footnotetext[1]{A version of this chapter appears as Malick, M.J., S.P. Cox,
  F.J. Mueter, B. Dorner, R.M. Peterman. Effects of the North Pacific
  Current on productivity of 163 Pacific salmon stocks. Fisheries Oceanography.
  \url{http://doi.org/10.1111/fog.12190}.}


\section{Abstract}

Horizontal ocean transport can influence the dynamics of higher-trophic-level
species in coastal ecosystems by altering either physical oceanographic
conditions or the advection of food resources into coastal areas. In this study,
we investigated whether variability in two North Pacific Current (NPC) indices
was associated with changes in productivity of North American Pacific salmon
stocks. Specifically, we used Bayesian hierarchical models to estimate the
effects of the north-south location of the NPC bifurcation (BI) and the NPC
strength, indexed by the North Pacific Gyre Oscillation (NPGO), on productivity
of 163 pink, chum, and sockeye salmon stocks. We found that for salmon stocks
located in Washington (WA) and British Columbia (BC), both the BI and NPGO had
significant positive effects on productivity, indicating that a
northward-shifted bifurcation and a stronger NPC are associated with increased
salmon productivity. For the WA and BC regions, the estimated NPGO effect was
over two times larger than the BI effect for pink and chum salmon, whereas for
sockeye salmon the BI effect was 2.4 times higher than the NPGO. In contrast to
WA and BC stocks, we found weak effects of both horizontal ocean transport
processes on productivity of salmon stocks in Alaska. Our results indicated that
horizontal transport pathways may strongly influence population dynamics of
Pacific salmon in the southern part of their North American ranges, but not the
northern part, suggesting that different environmental pathways may underlie
changes in salmon productivity in northern and southern areas for the species
under consideration.



\section{Introduction}

Environmental change can influence demographic rates of marine and anadromous
fish populations through multiple environmental pathways \citep{Ottersen2010a,
Drinkwater2010a}. It is often hypothesized that changes in atmospheric and
physical ocean conditions influence higher-trophic-level species via bottom-up
forcing that is mediated by vertical ocean transport \citep{Malick2015b,
DiLorenzo2013b, Ottersen2010a}. For example, upwelling of nutrient-rich water in
coastal areas is often assumed to drive primary and secondary production, which
in turn provide food for higher-trophic-level species \citep{Rykaczewski2008a}.
However, recent evidence from the California Current \citep{Bi2011b,
Keister2011a, Sydeman2011a} and Gulf of Alaska \citep{Stabeno2004a, Combes2009a,
Kline2010, Kline2008a} suggests that bottom-up forcing mediated by horizontal
transport (e.g., cross-shore or along-shore transport) may be equally important
for higher-trophic-level species \citep{DiLorenzo2013b}.

Changes in horizontal ocean transport, such as changes in ocean current
patterns, could influence higher-trophic-level species production by altering
foraging conditions. For Pacific salmon (\emph{Oncorhynchus} spp.), feeding
conditions and growth rates during the early marine life phase can strongly
influence stock productivity (i.e., the number of adult recruits produced per
spawner; \citealp{McGurk1996a, Farley2007b, Duffy2011, Malick2011a}). During this
critical period, juvenile salmon diets are largely composed of zooplankton and
other weakly-swimming or passive drifters \citep{Armstrong2008a, Beauchamp2007a,
Brodeur2007a}. Therefore, changes in ocean currents and subsequent advection of
potential prey into coastal areas may strongly influence juvenile salmon prey
availability or prey quality.

Indeed, in the Northern California Current region, a large-bodied lipid-rich
zooplankton community is associated with alongshore movement of cooler water
from northern areas into the region, whereas a small-bodied lipid-poor
zooplankton community is associated with the movement of warmer water from
southern and offshore areas into the region \citep{Bi2011b, Keister2011a}. The
lipid-rich northern zooplankton community, in particular, is associated with
higher coho salmon (\emph{O. kisutch}) survival in the Northern California
Current region \citep{Bi2011a}, suggesting that horizontal transport may be
important for salmon productivity in other regions. In addition, horizontal
transport may also be important for other salmon species, especially pink
(\emph{O. gorbuscha}), chum (\emph{O. keta}) and sockeye (\emph{O. nerka})
salmon, which tend to feed at a lower trophic level than coho salmon
\citep{Brodeur2007}.

In the Northeast Pacific Ocean, circulation is at least partially controlled by
the North Pacific Current (NPC; \citealp{Ware1989a, Cummins2007a}), which flows
approximately along 50$^{\circ}$N from west to east, bifurcating at the west
coast of North America into the northward flowing Alaska Current and the
southward flowing California Current (Fig. \ref{fig:npc:1}a; \citealp{Ware1989a,
Chelton1982a}). On average, the NPC bifurcates near the latitude of Vancouver,
BC, but the latitudinal position varies annually from southern Southeast Alaska
to southern Washington \citep{Cummins2007a, Sydeman2011a}. In addition to
variability in the positioning of the bifurcation, there is also inter-annual
variability in the strength of the NPC, measured as volume of water transported
per unit time \citep{Freeland2006a, Cummins2007a}. This volume is likely driven
by large-scale atmospheric and oceanographic patterns such as the North Pacific
Oscillation and the North Pacific Gyre Oscillation (NPGO;
\citealp{Di-Lorenzo2008a}).

In this study, we asked whether variability in the NPC can explain inter-annual
changes in productivity of 163 North American pink, chum, and sockeye salmon
stocks. Specifically, we evaluated the relationships between two indices of
variability in the NPC and productivity of those salmon stocks. One NPC index
represented inter-annual variability in the north-south positioning of the
bifurcation and the other represented inter-annual variability in strength.
Because the oceanography of coastal ecosystems in the Northeast Pacific differs
among geographic locations, we evaluated the relationships between salmon
productivity and the NPC indices separately for three large marine ecosystems in
the Northeast Pacific: the west coast of Washington and British Columbia, Gulf
of Alaska, and Bering Sea \citep{Sherman1999, Longhurst1995}. We used a Bayesian
hierarchical modeling approach to estimate both stock-specific and
ecosystem-level effects of the NPC on salmon productivity, which allowed us to
leverage the large number of available salmon data sets by using the stocks as
replicates within the analysis, reducing the chances of finding spurious
relationships between salmon productivity and the NPC indices \citep{Myers1998c,
Thorson2015b, Mueter2002a}.

\begin{figure}[htbp]
  \centering \includegraphics[scale=0.9]{3_npcurrent/figures/map-npc.pdf}
  \caption[Study area indicating the ocean entry locations for salmon
           stocks by species]{Study area indicating the ocean entry locations
           for (a) pink salmon stocks, (b) chum salmon stocks, and (c) sockeye
           salmon stocks. Solid circles (red) indicate stocks located in the
           West Coast ecosystem; solid squares (blue) indicate stocks located in
           the Gulf of Alaska ecosystem; solid triangles (green) indicate stocks
           in the Bering Sea ecosystem. Stocks are numbered consecutively from
           south to north for each species. In panel (a), the thick black arrow
           shows the North Pacific Current, which flows from west-southwest to
           east-northeast and the grey arrows show the bifurcation of the North
           Pacific Current into the northward flowing Alaska Current and
           southward flowing California Current.}
  \label{fig:npc:1}
\end{figure}



\section{Methods}

\subsection{Salmon data}

We used spawner (escapement) and total recruitment data (catch plus escapement)
for 163 wild sockeye (64 stocks), pink (46 stocks), and chum (53 stocks) salmon
stocks throughout their North American ranges (Fig. \ref{fig:npc:1}). The
duration of stock-specific data sets ranged from 12 to 56 brood years (i.e.,
years of spawning) 1950--2009, with mean time series lengths of 34 years for
pink salmon, 33 years for chum salmon, and 38 years for sockeye salmon. For pink
and chum salmon, data sets generally represented aggregations of adjacent salmon
populations, which helped ensure that catch records were properly attributed to
the correct spawning population. Details of the data sets can be found in
\citet{Peterman2012} and \citet{Malick2016a}.

We organized the salmon data sets into three large marine ecosystems based on
the ocean entry locations of each stock. All stocks that enter the ocean along
the west coast of Washington and British Columbia were grouped into the West
Coast ecosystem (WC). Stocks entering the ocean in Southeast Alaska and South
Central Alaska were grouped into the Gulf of Alaska ecosystem (GOA), and Western
Alaska stocks were grouped into the Bering Sea ecosystem (BS; Fig.
\ref{fig:npc:1}). Organization of the salmon data sets into three large marine
ecosystems was based on two pieces of information. First, the underlying
oceanographic processes tend to be substantially different across these regions
with the WC ecosystem being primarily an upwelling domain and the GOA ecosystem
being primarily a downwelling domain \citep{Ware1989a}. Second, several previous
studies, e.g., \citet{Malick2015a} and \citet{Mueter2002a}, have indicated that
regional-scale ocean conditions can have opposite effects on salmon productivity
in northern and southern regions with the dividing line occurring approximately
at the border between Southeast Alaska and British Columbia.


\subsection{North Pacific Current indices}

We used the Ocean Surface Current Simulations (OSCURS) model to compute
inter-annual variability in the north-south location of the NPC bifurcation
\citep{Ingraham1997a} and the NPGO to index broad-scale variability in the
strength of the NPC \citep{Di-Lorenzo2008a}.

The OSCURS model simulates trajectories of surface currents in the North Pacific
by adding wind velocity fields (derived from daily atmospheric
sea-level-pressure data) to the long-term mean geostrophic current fields. The
resulting simulated surface current trajectories have been shown to closely
match satellite-tracked drifters in the North Pacific \citep{Ingraham1997a}. We
developed an index for the north-south location of the NPC bifurcation by
generating annual trajectories for 215 simulated drifters for all years between
1967 and 2010, inclusive, using a procedure analogous to that of
\citet{Watters2008a}. In our case, drifters were seeded on a 1-degree grid in
the area bounded by -140$^{\circ}$W longitude eastward to the coast of North
America and from 40$^{\circ}$N to 55$^{\circ}$N latitude (Supporting materials
Fig. \ref{fig:npc:s1}). Simulated drifters were released annually on February 1
and the daily trajectory was tracked until June 30 to reflect ocean conditions
relevant to seaward-migrating juvenile salmon \citep{DiLorenzo2013a}. We indexed
the location of the bifurcation based on differences between the starting and
ending latitude of each drifter within a year. The annual bifurcation index (BI)
was calculated as the proportion of the 215 simulated drifters that ended south
of their starting latitude in a particular year (Supporting materials Fig.
\ref{fig:npc:s2}). Positive values of the index indicate a northward-shifted
bifurcation (majority of drifters ending south of their starting latitude),
whereas negative values of the index indicate a southward-shifted bifurcation
(majority of drifters ending north of their starting latitude).

The NPGO, defined as the second principal component of monthly sea surface
height anomalies in the North Pacific over the region
25$^{\circ}$N--62$^{\circ}$N, 180$^{\circ}$--110$^{\circ}$W
\citep{Di-Lorenzo2008a}, is thought to represent variability in sub-polar and
sub-tropical gyre strengths in the North Pacific, where higher NPGO values
indicate a strengthening of the gyres and increased NPC transport
\citep{Chhak2009a, Di-Lorenzo2009a, Di-Lorenzo2008a}. Positive values of the
NPGO are also associated with higher nutrient concentrations (e.g.,
NO\textsubscript{3}), higher salinity, and higher survival of both coho and
chinook salmon \citep{Di-Lorenzo2008a, Di-Lorenzo2009a, Kilduff2015}. Salmon
survival during the early marine life phase is believed to be the dominant
driver of overall stock productivity, and ocean conditions prior to salmon ocean
entry may strongly influence conditions experienced by salmon during this period
\citep{Parker1968a, Wertheimer2007a, Yeh2011, DiLorenzo2013a}. Therefore, we
averaged the NPGO over the months of December--March, which represents the
winter period just prior to ocean entry of salmon smolts.

For pink and chum salmon, which enter the ocean the first spring following
spawning, the BI and NPGO indices were offset by 1 year (e.g., salmon spawning
in 2000 were lined up with the BI for 2001 and the NPGO for the December 2000 to
March 2001 period). For sockeye salmon, which rear in lakes for one or two years
before entering the ocean, we used a weighted average of index values offset by
2 years and 3 years, respectively, with the weights equal to the stock-specific
average proportion of juveniles entering the ocean at either age two or three
\citep{Mueter2002a}.


\subsection{Modeling the data}

We modeled salmon stock productivity as a function of spawner abundance using
the standard Ricker model \citep{Ricker1954a},

\begin{equation}
y_{i,t} = \alpha_i + \beta_i S_{i,t} + \epsilon_{i,t},
\label{eq:npc:1}
\end{equation}

\noindent where \(y_{i,t}\) is the log\textsubscript{e} of recruits per spawner,
\(log_e({R_{i,t} / S_{i,t}})\), for stock \(i\) in year \(t\), \(\alpha_i\) is
the density-independent stock productivity at low spawning stock sizes,
\(\beta_i\) is the coefficient representing the strength of density-dependence,
and \(\epsilon_{i,t}\) is the residual error term assumed to be normally
distributed with mean 0 and variance \(\sigma^2_{\epsilon}\).

We estimated BI and NPGO effects on salmon productivity using a generalized
Ricker model in which the oceanographic variables were included as additional
predictor variables \citep{Quinn1999}, i.e.,

\begin{equation}
y_{i,t} = \alpha_i + \beta_i S_{i,t} + \gamma_{BI,i} BI_{t} +
\gamma_{NPGO,i} NPGO_{t} + \epsilon_{i,t},
\label{eq:npc:2}
\end{equation}

\noindent where \(\gamma_{BI,i}\) is the stock-specific coefficient for the BI and
\(\gamma_{NPGO, i}\) is the stock-specific coefficient for the NPGO index. The
annual value of the BI index was the same for all pink and chum salmon stocks,
as was the value of the NPGO index, but index values for sockeye salmon stocks
were stock-specific, as explained above. In addition, both the BI and NPGO
indices were standardized to a mean of 0 and a standard deviation (SD) of 1.

We included both the standard and generalized Ricker models in the analysis for
model comparison purposes, which allowed us to compare models with and without
the NPC indices to determine the relative importance of these terms in the model
(see model comparison section for details). Because data were not available to
calculate the BI prior to 1967, we fit both the standard and generalized Ricker
models using data from brood years 1966 and after.

We included a first-order autocorrelation model for residuals, i.e.,
\(\epsilon_{i,t} = \phi_i \epsilon_{i,t-1} + \delta_{i,t}\), where
\(\delta_{i,t} \sim N(0,\sigma_i^2)\) and \(\phi_i\) is the first-order
autocorrelation coefficient for stock \(i\) \citep{Mueter2002a, Chatfield2004}.
For both the standard and generalized forms of the Ricker model, the
autoregressive process was modeled as,

\begin{equation}
y_{i,t} = \left\{
    \begin{array}{ll}
    \hat{y}_{i,t} + \phi_i \epsilon_{i,t-1} + \delta_{i,t} & \text{for } t>1 \\
    \hat{y}_{i,t} + \delta_{i,t} & \text{for } t=1
    \end{array}
  \right.
\label{eq:npc:3}
\end{equation}

\noindent where \(\hat{y}_{i,t}\) is the predicted stock productivity from
either the standard Ricker model (eq. \ref{eq:npc:1}) or the generalized Ricker
model (eq. \ref{eq:npc:2}).


\subsection{Modeling the parameters}

The Bayesian hierarchical modeling approach is increasingly common in
multi-stock population dynamics research, in part because allowing dependence
among stock-specific parameters can improve parameter estimates
\citep{Gelman2004a, Thorson2015b}. In particular, modeling stock-specific
parameters (e.g., \(\alpha_i\) or \(\gamma_{BI,i}\)) as arising from a common
prior distribution (i.e., assuming stocks are exchangeable units) improves the
mean of parameter estimates where hyperparameters for the common distribution
are informed by data from all stocks \citep{Gauch2006}. In this study, we fit
species-specific Bayesian hierarchical models that used hierarchical prior
distributions for \(\alpha_i\), \(\gamma_{BI,i}\), and \(\gamma_{NPGO,i}\)
parameters, where the hierarchical priors were further defined by a set of
hyperprior distributions \citep{Gelman2004a}.

For each species-specific model, we assumed that the \(\alpha_i\) were
exchangeable across all stocks within a species and we used a normal prior
distribution, i.e., \(\alpha_i \sim N(\mu_{\alpha}, \tau^2_{\alpha})\) with
hyperparameters \(\mu_{\alpha}\) and \(\tau^2_{\alpha}\) representing the
overall mean and variance, respectively. We used a diffuse normal distribution,
\(\mu_{\alpha} \sim N(0, 10^3)\), for the hypermean \(\mu_{\alpha}\) and an
improper uniform prior for the hypervariance, \(\tau^2_{\alpha} \sim U(0, 25)\)
\citep{Gelman2006}.

We assumed that the \(\gamma_{BI,i}\), and \(\gamma_{NPGO,i}\) parameters were
only exchangeable among stocks within the same ecosystem (WC, GOA, BS) for each
of the BI and NPGO because ocean conditions can influence salmon stocks in
different ecosystems in opposite ways \citep{Mueter2002a, Malick2015a}. As
examples of our exchangeability assumption, parameters for all pink salmon
stocks in the WC ecosystem were assumed exchangeable and were assigned one prior
distribution (i.e., \(\gamma_{i,WC} \sim N(\mu_{\gamma_{WC}},
\tau^2_{\gamma_{WC}})\)), whereas all pink salmon stocks in the GOA ecosystem
were assumed exchangeable and were assigned a separate prior distribution (i.e.,
\(\gamma_{i,GOA} \sim N(\mu_{\gamma_{GOA}}, \tau^2_{\gamma_{GOA}})\)). For a
particular ecosystem and oceanographic variable, the hypermean \(\mu_{\gamma}\)
represents the mean ecosystem-level effect and the hypervariance
\(\tau^2_{\gamma}\) represents the ecosystem-level variance. Diffuse normal
prior distributions, \(\mu_{\gamma} \sim N(0, 10^3)\), and uniform prior
distributions, \(\tau^2_{\gamma} \sim U(0, 25)\), were used for the
ecosystem-level hypermeans and hypervariances, respectively \citep{Gelman2006}.

In contrast to the \(\alpha\) and \(\gamma\) parameters, which we assumed were
exchangeable across salmon stocks within a species or ecosystem, we treated the
remaining parameters, i.e., \(\beta_i, \sigma_i\), and \(\phi_i\), as
non-exchangeable (i.e., stock-specific) because the magnitudes of these
parameters can vary greatly among salmon stocks within a species and ecosystem
\citep{Mueter2002a, Malick2015a, Su2004a}. We assigned diffuse independent
priors for the density-dependence parameters, \(\beta_i \sim N(0, 10^3)\), and
assigned the variances, \(\sigma_i^2\), and autocorrelation coefficients,
\(\phi_i\), to be uniform priors, \(\sigma_i^2 \sim U(0, 25)\) and \(\phi_i \sim
U(-1, 1)\), respectively.

Because we were uncertain about the similarity of the \(\sigma_i^2\) and
\(\phi_i\) parameters across stocks within a species, we also fit several
simpler models in which \(\sigma_i^2\) and \(\phi_i\) were shared across stocks,
i.e., they were not stock specific. In total, for each species we fit five
standard Ricker models and five generalized Ricker models that differed in their
assumptions about \(\sigma_i^2\) and \(\phi_i\) (Table \ref{tab:npc:1}).

To better demonstrate the effects of the NPGO and BI on salmon productivity, we
also calculated the percent change in productivity given a one unit change in
the NPGO or BI. More specifically, we used the estimated ecosystem-level effects
of the NPGO and BI (i.e., \(\mu_{\gamma}\)) to calculate the estimated percent
change in productivity that would result from an increase in the BI or NPGO
corresponding to one SD above their respective long-term means (1967--2010).

\begin{table}[htbp]
  \small \centering \libertineLF
  \caption[Summary of Bayesian hierarchical models fit for each
           species]{Summary of Bayesian hierarchical models fit for each
           species. \# gives the model number; type indicates whether the
           model is a standard or generalized Ricker model; ``exchange''
           indicates the parameters were exchangeable across all stocks;
           ``ecosystem'' indicates the parameters were exchangeable across
           stocks within an ecosystem; ``same'' indicates the parameter was
           shared (i.e., the same) across stocks and ecosystems; ``different''
           indicates the parameter was stock-specific.}
  % latex table generated in R 3.3.1 by xtable 1.8-2 package
% Tue Nov  1 13:07:20 2016
\# & Type & $\alpha$ & $\beta$ & $\sigma^2$ & $\phi$ & $\gamma_{BI}$ & $\gamma_{NPGO}$ \\ 
  \hline
  1 & standard & exchange & different & same &  &  &  \\ 
    2 & standard & exchange & different & different &  &  &  \\ 
    3 & standard & exchange & different & same & same &  &  \\ 
    4 & standard & exchange & different & different & same &  &  \\ 
    5 & standard & exchange & different & different & different &  &  \\ 
    6 & generalized & exchange & different & same &  & ecosystem & ecosystem \\ 
    7 & generalized & exchange & different & different &  & ecosystem & ecosystem \\ 
    8 & generalized & exchange & different & same & same & ecosystem & ecosystem \\ 
    9 & generalized & exchange & different & different & same & ecosystem & ecosystem \\ 
   10 & generalized & exchange & different & different & different & ecosystem & ecosystem \\ 
  
  \label{tab:npc:1}
\end{table}


\subsection{Model fitting and diagnostics}

We estimated all model parameters using the Gibbs sampling algorithm implemented
in JAGS version 3.4.0 \citep{Plummer2003}. For each model, we ran five chains
with dispersed starting values. Each chain had a burn-in period of 10 000
iterations followed by 75 000 iterations that were monitored with a thinning
interval of 15, where the thinning interval was determined by monitoring
within-chain autocorrelation. We based posterior inference on a total of 25 000
posterior samples per parameter obtained by sampling 5 000 iterations per chain.
Gibbs chain convergence was assessed graphically (e.g., traceplots, histograms)
and via the Gelman-Rubin statistic \citep{Gelman1992, Brooks1998}. We assessed
model fits using posterior predictive checks, including fitted values, realized
residuals, and posterior predictive distributions \citep{Gelman2004a}.


\subsection{Model comparison}

We used the Watanabe-Akaike information criterion (WAIC) for model comparison
and model selection within species \citep{Watanabe2010, Gelman2013a}. The WAIC
measures the fit of a model to the data while also accounting for model
complexity. Both the ``fit'' and ``complexity'' terms of the WAIC were readily
computed from posterior samples of the parameters. The model fit was assessed
using the log-pointwise predictive density (\(lppd\)), whereas model complexity
was estimated as the effective number of model parameters (\(pD\);
\citealp{Watanabe2010, Gelman2013a}). The WAIC was then calculated as
\(\text{WAIC} = -2 * (lppd - pD)\). The model with the lowest WAIC value was
considered the most parsimonious and models within 3 WAIC units of the minimum
were considered equally plausible. Models with WAIC values greater than 10 more
than the minimum were rejected.


\subsection{Sensitivity analysis}

We tested the sensitivity of our results to two assumptions underlying the
oceanographic variables. First, to check whether our grid of simulated OSCURS
model drifters captured broad-scale surface current patterns in the North
Pacific, we re-calculated the BI using an expanded 1-degree grid that extended
from -145$^{\circ}$W longitude eastward to the west coast of North America and
from 35$^{\circ}$N to 59$^{\circ}$N latitude (Supporting materials Fig.
\ref{fig:npc:s1}). Second, we checked the sensitivity of our results to the
manner in which the BI was calculated. Specifically, we re-calculated the BI
following the method of \citet{Watters2008a}, where the annual index values were
calculated as the sum of the differences in the longitude of the drifter
ensemble (i.e, all drifters released at the same longitude) between the median
starting latitude on February 1 and the median ending latitude on June 30. This
contrasts with our original BI index, which was calculated as the proportion of
drifters that ended south of their starting latitude, by summarizing the start
and end latitudes of the drifters prior to calculating the BI index. We assessed
the sensitivity of our analysis to specifics of BI calculation by determining
strength of correlation between our original BI time series and the alternate BI
series, as well as by comparing model coefficients and rankings for models fit
using the alternate BI time series.

We also conducted an additional sensitivity analysis to test if our grouping of
salmon stocks into three large marine ecosystems adequately captured the spatial
distribution of the effects of the BI and NPGO. In particular, we fit
single-stock generalized Ricker models (i.e., eq. \ref{eq:npc:2}) to each of the
163 salmon stocks separately using maximum likelihood to identify potential
differences among stocks within our three ecosystem groupings. Unlike the
Bayesian hierarchical models where the \(\gamma_i\) coefficients are centered
around a common mean, in the single-stock models each stock is independent of
the other stocks, which allowed us to better determine if smaller spatial-scale
patterns were evident in the \(\gamma_{BI,i}\) and \(\gamma_{NPGO,i}\)
coefficients.



\section{Results}

\subsection{BI and NPGO indices}

The BI time series indicated substantial inter-annual variability in the
latitude of the NPC bifurcation ranging from 11\% of all drifters ending south
of their starting latitude in 1993 (i.e., the bifurcation was shifted southward)
to 74\% of all drifters ending south of their starting latitude in 2009 (i.e.,
the bifurcation was shifted northward; Fig. \ref{fig:npc:2}; Supporting
materials Fig. \ref{fig:npc:s2}). The BI index tended to have more inter-annual
variability than the NPGO with fewer series of consecutive positive or negative
values. For example, years with a northward-shifted bifurcation were often
followed by years with a southward shifted bifurcation, such as 1982--83,
1985--86, and 2009--10. The BI and NPGO indices were only weakly correlated
(\(r\) = 0.24), suggesting that they capture different modes of NPC variability.

\begin{figure}[htbp]
  \centering \includegraphics[scale=0.9]{3_npcurrent/figures/bi_npgo_line.pdf}
  \caption[Time series for the bifurcation index and North Pacific Gyre
           Oscillation]{Time series for the bifurcation index (BI; upper panel)
           and North Pacific Gyre Oscillation (NPGO; bottom panel). Both time
           series are standardized to a mean of 0 and a standard deviation of 1
           (i.e., standard deviation units, SDUs). Positive values of the BI
           reflect a northward-shifted bifurcation in the NPC, whereas negative
           BI values indicate a southward-shifted bifurcation. Positive values
           of the NPGO indicate a stronger NPC, whereas negative NPGO values
           indicate a weaker NPC.}
  \label{fig:npc:2}
\end{figure}


\subsection{BI and NPGO effects}

Hierarchical models that included the BI and NPGO indices fit the data
substantially better than models without these terms for all species, as
indicated by the WAIC (Table \ref{tab:npc:2}). The best models (i.e., models
with the lowest WAIC) showed that the strongest effects of the BI and NPGO were
on stocks in the WC ecosystem, where a northward-shifted bifurcation (i.e.,
positive BI values) and a stronger NPC (i.e., positive NPGO values) were
consistently associated with increased productivity of pink, chum, and sockeye
salmon, i.e., positive \(\gamma_{BI,i}\) and \(\gamma_{NPGO,i}\) values (Figs.
\ref{fig:npc:3} and \ref{fig:npc:4}). In contrast, the BI and NPGO effects on
salmon productivity tended to be weaker for stocks in the GOA ecosystem and less
consistent across species in the BS ecosystem than in the WC ecosystem (Figs.
\ref{fig:npc:3} and \ref{fig:npc:4}).

\begin{table}[htbp]
  \small \centering \libertineLF
  \caption[Model selection quantities for each fitted model]{Model selection
           quantities for each fitted model. \# gives the model number as
           defined in Table \ref{tab:npc:1}; Np gives the nominal number of
           parameters in a model; pD gives the effective number of parameters;
           and $\Delta$WAIC gives the WAIC value for each model relative to the
           model with the minimum WAIC value.}
  % latex table generated in R 3.3.2 by xtable 1.8-2 package
% Sat Dec 17 10:16:48 2016
\begin{tabular}{lrrrrrrrrr}
  \hline
  \multicolumn{1}{r}{} &
            \multicolumn{3}{c}{Pink} &
            \multicolumn{3}{c}{Chum} &
            \multicolumn{3}{c}{Sockeye} \\ 
  \cmidrule(lr){2-4} \cmidrule(lr){5-7} \cmidrule(lr){8-10} 
  \multicolumn{1}{l}{\#} & 
    \multicolumn{1}{l}{Np} & 
    \multicolumn{1}{l}{pD} & 
    \multicolumn{1}{l}{$\Delta$WAIC} & 
    \multicolumn{1}{l}{Np} & 
    \multicolumn{1}{c}{pD} & 
    \multicolumn{1}{c}{$\Delta$WAIC} & 
    \multicolumn{1}{c}{Np} & 
    \multicolumn{1}{c}{pD} & 
    \multicolumn{1}{l}{$\Delta$WAIC} \\ 
  \cmidrule{1-10} 
1 & 95 & 71 & 260.0 & 109 & 89 & 292.1 & 131 & 104 & 589.0 \\ 
  2 & 140 & 115 & 51.7 & 161 & 133 & 180.2 & 194 & 164 & 287.8 \\ 
  3 & 96 & 71 & 254.1 & 110 & 85 & 156.8 & 132 & 102 & 328.0 \\ 
  4 & 141 & 113 & 42.6 & 162 & 134 & 48.0 & 195 & 164 & 6.0 \\ 
  5 & 186 & 152 & 101.0 & 214 & 175 & 22.6 & 258 & 218 & 16.1 \\ 
  6 & 199 & 98 & 224.4 & 227 & 119 & 266.4 & 271 & 148 & 556.8 \\ 
  7 & 244 & 135 & 10.8 & 279 & 155 & 156.1 & 334 & 204 & 248.7 \\ 
  8 & 200 & 97 & 214.4 & 228 & 115 & 148.7 & 272 & 138 & 322.7 \\ 
  9 & 245 & 134 & \textbf{0.0} & 280 & 156 & 29.7 & 335 & 189 & \textbf{0.0} \\ 
  10 & 290 & 174 & 57.2 & 332 & 198 & \textbf{0.0} & 398 & 250 & 3.2 \\ 
   \hline
\end{tabular}

  \label{tab:npc:2}
\end{table}

\begin{figure}[htbp]
  \centering \includegraphics[scale=0.9]{3_npcurrent/figures/gamma_all.pdf}
  \caption[Posterior medians and 95\% credible intervals for the stock-specific
           bifurcation index and North Pacific Gyre Oscillation
           effects]{Posterior medians and 95\% credible intervals for the
           stock-specific bifurcation index (BI) and North Pacific Gyre
           Oscillation (NPGO) effects, \(\gamma_{BI,i}\) and
           \(\gamma_{NPGO,i}\), respectively, as derived from the best-fit
           models (9 and 10 in Table \ref{tab:npc:1}). Coefficients (in standard
           deviation units) are shown for pink salmon (panel a), chum salmon
           (panel b), and sockeye salmon (panel c). Within each panel,
           stock-specific estimates are grouped by ecosystem and stocks are
           ordered south (left) to north (right) where the stock number (x-axis)
           corresponds to the numbers in Fig. \ref{fig:npc:1}. Solid circles
           (black) indicate \(\gamma_{BI,i}\) median values and dashed black
           lines indicate the 95\% credible intervals for the BI effect. Solid
           squares (red) indicate \(\gamma_{NPGO,i}\) median values and dashed
           red lines indicate the 95\% credible intervals for the NPGO effect.
           Solid horizontal lines indicate posterior medians for the
           ecosystem-level effects, \(\mu_{\gamma}\).}
  \label{fig:npc:3}
\end{figure}

\begin{figure}[htbp]
  \centering \includegraphics[scale=0.9]{3_npcurrent/figures/gamma_density.pdf}
  \caption[Posterior distributions for the ecosystem-level effects of the
           bifurcation index and North Pacific Gyre Oscillation]{Posterior
           distributions for the ecosystem-level effects, \(\mu_{\gamma}\), of
           the bifurcation index (BI) and North Pacific Gyre Oscillation (NPGO)
           in standard deviation units. Distributions are shown for pink salmon
           (top row), chum salmon (middle row), and sockeye salmon (bottom row),
           as well as for the West Coast ecosystem (left column), Gulf of Alaska
           ecosystem (middle column) and Bering Sea ecosystem (right column).
           Solid lines (black) indicate distributions for the ecosystem-level BI
           effect and dashed lines (red) indicate distributions for the NPGO
           ecosystem-level effect.}
  \label{fig:npc:4}
\end{figure}

For pink and chum salmon stocks in the WC ecosystem, the estimated median
ecosystem-level effect (i.e., \(\mu_{\gamma}\)) for the NPGO was 2.4 times
higher than the BI for pink salmon and 2.5 times higher for chum salmon (Table
\ref{tab:npc:3}; Fig. \ref{fig:npc:4}). Similarly, the estimated median
stock-specific effects (i.e., \(\gamma_i\)) for the NPGO (which are centered
around the mean ecosystem-level effect) were consistently higher than for the
median stock-specific effects of the BI (Fig. \ref{fig:npc:3}; pink salmon:
\(\gamma_{NPGO}\) range = 0.22--0.29 vs. \(\gamma_{BI}\) range = 0.06--0.13;
chum salmon: \(\gamma_{NPGO}\) range = 0.12--0.15 vs. \(\gamma_{BI}\) range =
0.03--0.08). An increase in the NPGO by one SD above the long-term mean
(1967--2010) in any given year would be expected to result in 28.5\% and 15.1\%
higher recruits-per-spawner for pink and chum salmon, respectively (Fig.
\ref{fig:npc:5}).

In contrast to pink and chum salmon, sockeye salmon productivity in the WC
ecosystem was more strongly related to changes in the BI than the NPGO,
indicating that the location of the bifurcation has a stronger effect on sockeye
salmon productivity than NPC strength. The estimated median ecosystem-level
effect of the BI on WC sockeye salmon productivity was 2.4 times higher than for
the NPGO (Table \ref{tab:npc:3}; Fig. \ref{fig:npc:4}). Furthermore, the median
stock-specific effects of the BI on sockeye salmon productivity (i.e.,
\(\gamma_{BI}\) range = 0.09--0.13) were consistently higher than stock-specific
effects of the NPGO (\(\gamma_{NPGO}\) range = 0.02--0.08; Fig.
\ref{fig:npc:3}c). This stronger effect of the BI on sockeye salmon productivity
corresponded to an 11.7\% increase in productivity given a one SD-unit increase
in the BI compared to a 4.8\% increase in productivity given a one SD-unit
increase in the NPGO index (Fig. \ref{fig:npc:5}).

In the GOA ecosystem, there was no evidence for consistent stock-specific or
ecosystem-level effects of either the BI or NPGO on productivity of all three
species (Figs. \ref{fig:npc:3} and \ref{fig:npc:4}; Table \ref{tab:npc:3}). The
95\% credibility intervals for stock-specific effects included zero for all
species, stocks, and both the BI and NPGO indices (Fig. \ref{fig:npc:3}).
Similarly, the posterior distributions for the GOA ecosystem-level effects of
the BI and NPGO on salmon productivity were close to zero for all species (Fig.
\ref{fig:npc:4}; Table \ref{tab:npc:3}).

In the BS ecosystem, the effects of the BI and NPGO on salmon productivity were
quite variable across species. For BS pink salmon, the effects of the NPGO were
close to zero, and the effects of the BI were consistently positive
(\(\gamma_{BI}\) range = 0.08--0.54), although there was considerable
uncertainty associated with the pink salmon \(\gamma_i\) parameter estimates
(Figs. \ref{fig:npc:3} and \ref{fig:npc:4}; Table \ref{tab:npc:3}). In contrast,
for BS chum salmon the estimated effects of the NPGO were consistently positive
(\(\gamma_{NPGO}\) range = 0.08--0.12), but the BI effects were close to zero
(\(\gamma_{BI}\) range = -0.06--0.002). For sockeye salmon in the BS ecosystem,
the posterior distributions for the stock-specific and ecosystem-level effects
of the BI and NPGO were both near zero (Figs. \ref{fig:npc:3} and
\ref{fig:npc:4}; Table \ref{tab:npc:3}), indicating neither index has strong
effects on productivity of sockeye salmon in the Bering Sea.

\begin{table}[htbp]
  \small \centering \libertineLF
  \caption[Ecosystem-wide effects for the BI and NPGO indices from the
           best-fit models]{Ecosystem-wide effects (i.e., $\mu_{\gamma}$) for
           the BI and NPGO indices from the best-fit models (9 and 10 in Table
           \ref{tab:npc:1}). Values are in standard deviation units and show the
           median for $\mu_{\gamma}$ with 95\% credible intervals given in
           parentheses.}
  % latex table generated in R 3.3.2 by xtable 1.8-2 package
% Sat Dec 17 10:19:28 2016
\begin{tabular}{llrr}
  \hline
Species & Ecosystem & $\mu_{\gamma,BI}$ & $\mu_{\gamma,NPGO}$ \\ 
  \hline
Pink & WC & 0.104 (0.025, 0.186) & 0.251 (0.162, 0.339) \\ 
   & GOA & -0.024 (-0.086, 0.036) & 0.036 (-0.028, 0.099) \\ 
   & BS & 0.283 (-0.257, 0.864) & -0.031 (-0.415, 0.383) \\ 
  Chum & WC & 0.057 (-0.001, 0.117) & 0.141 (0.075, 0.204) \\ 
   & GOA & 0.013 (-0.043, 0.075) & -0.007 (-0.077, 0.060) \\ 
   & BS & -0.025 (-0.091, 0.044) & 0.096 (0.003, 0.186) \\ 
  Sockeye & WC & 0.111 (0.052, 0.170) & 0.047 (-0.032, 0.128) \\ 
   & GOA & -0.029 (-0.080, 0.023) & 0.026 (-0.049, 0.094) \\ 
   & BS & 0.012 (-0.083, 0.096) & 0.038 (-0.051, 0.126) \\ 
   \hline
\end{tabular}

  \label{tab:npc:3}
\end{table}

\begin{figure}[htbp]
  \centering \includegraphics[scale=0.9]{3_npcurrent/figures/effect_size.pdf}
  \caption[Percent change in productivity given a one unit increase in either
           the bifurcation index or North Pacific Gyre Oscillation]{Percent
           change in productivity given a one unit increase in either the
           bifurcation index (BI; grey) or North Pacific Gyre Oscillation (NPGO;
           red) for stocks in the (a) West Coast ecosystem, (b) Gulf of Alaska
           ecosystem, and (c) Bering Sea ecosystem. Vertical bars indicate the
           95\% credible interval for the estimated percent change.}
    \label{fig:npc:5}
\end{figure}

Model selection results indicated that for all three species there were no
alternative models with a WAIC within three units of the minimum WAIC (Table
\ref{tab:npc:2}). In addition, for pink and chum salmon, no models had a WAIC
within 10 units of the minimum WAIC, whereas sockeye salmon had two models that
had WAIC values within 10 units of the minimum (models 4 and 10; Table
\ref{tab:npc:2}). Models that included a stock-specific autocorrelation term
tended to fit the data better for chum salmon, whereas for pink and sockeye
salmon there was more support for an autocorrelation term shared across stocks
(Table \ref{tab:npc:2}). For all three species, there was also greater support
for models that included a stock-specific variance term compared to models that
only included a shared variance term (Table \ref{tab:npc:2}). Furthermore, all
models showed a lower number of effective parameters compared to the nominal
number of parameters, suggesting considerable borrowing of information across
stocks within the models (Table \ref{tab:npc:2}).


\subsection{Sensitivity analysis}

Estimated BI effects on salmon productivity were mostly insensitive to the
methods we used to calculate the index. For instance, our original BI index was
highly correlated with the alternative index calculated using a larger grid of
drifters (\(r\) = 0.94) and with the index calculated using the methods outlined
in \citet{Watters2008a} (\(r\) = 0.94). For the index calculated using a larger
grid of drifters, the rank order of the fitted models did not change compared to
the models fitted using the original BI index for all three species
(Supporting materials Table \ref{tab:npc:s1}). For the index based on
\citet{Watters2008a}, the rank order of models did not change for pink and chum
salmon, but were moderately sensitive for sockeye salmon (Supporting materials
Table \ref{tab:npc:s2}). In particular, the top three models were the same for
sockeye salmon for both indices, although the order was different, with a model
without the BI or NPGO having the lowest WAIC when the alternate
\citet{Watters2008a} BI index was used (model 4; Supporting materials Table
\ref{tab:npc:s2}). In addition, the single-stock model analysis did not indicate
substantial differences in the effects of the BI and NPGO across stocks at
smaller spatial scales than those of the three large marine ecosystems
(Supporting materials Fig. \ref{fig:npc:s3}).



\section{Discussion}

In this study, we estimated the effects of two modes of variability in the NPC
on productivity of 163 pink, chum, and sockeye salmon stocks to better
understand how horizontal ocean transport pathways could influence population
dynamics of Pacific salmon. We found that indices of north-south positioning of
the NPC bifurcation and the NPC strength both had strong estimated effects on
pink, chum, and sockeye salmon productivity and were included in the best-fit
hierarchical models for each species. We also found that the most consistent
effects of the BI and NPGO were on salmon stocks in the WC ecosystem. There, a
northward-shifted bifurcation and increased NPC strength was associated with
increased salmon productivity. Finally, we found that neither index was
correlated with salmon productivity for stocks in the Gulf of Alaska, and there
were less consistent effects for salmon stocks in the Bering Sea than in the
West Coast ecosystem.

Our result that variability associated with the NPC is an important driver of
changes in Pacific salmon productivity, particularly for stocks in the WC
ecosystem, is consistent with the results of several previous studies that have
indicated that horizontal transport pathways can strongly influence coastal
marine ecosystems in the Northeast Pacific \citep{Kilduff2015, Sydeman2011a,
Batten2007a}. In particular, our finding that a stronger NPC (i.e., a positive
NPGO) is associated with increased salmon productivity in the WC ecosystem
corresponds with the result of \citet{Kilduff2015}, which showed that the
dominant modes of variability for hatchery coho and chinook salmon survival
rates along the west coast of North America are significantly and positively
related to the NPGO index. Although the mechanisms linking variability in the
NPGO and salmon productivity are not clear, several studies have indicated that
broad-scale variability in the strength of the sub-polar and sub-tropical gyres,
as indexed by the NPGO, is linked with changes in salinity, nutrients, and chl-a
concentrations in coastal ecosystems in the Northeast Pacific
\citep{Di-Lorenzo2008a, Chenillat2012}. This suggests that the effects of the
NPGO on salmon productivity may be mediated by changes in physical and
biological oceanographic conditions that affect prey availability in those
ecosystems.

Our results further suggested that a northward-shifted positioning of the NPC
was associated with increased salmon productivity in the WC ecosystem. This is
consistent with previous research that indicated that the majority
(\textasciitilde{}64\%) of variability in biological productivity in the
Northern California Current is related to the north-south location of the NPC
bifurcation, with higher productivity being associated with a northward-shifted
positioning of the NPC \citep{Sydeman2011a}. One possible explanation for this
result is that shifts in the location of the NPC may influence the advection of
zooplankton communities into coastal ecosystems, which are a key food resource
for juvenile salmon \citep{Armstrong2008a, Beauchamp2007a, Brodeur2007a}. For
example, \citet{Keister2011a} and \citet{Bi2011b} indicated that changes in
along-shore transport can strongly influence zooplankton communities in the
Northern California Current with the transport of cooler water from northern
areas into the Northern California Current being associated with a more
lipid-rich copepod community.

Alternatively, the location of the bifurcation may also have indirect effects on
high-trophic-level species by altering physical or biological oceanographic
conditions such as inorganic nutrient concentrations, water column stability, or
thermal regimes \citep{Di-Lorenzo2009a, Sydeman2011a, Keister2011a}. These
indirect effects may be particularly important for salmon stocks that enter the
ocean in areas sheltered from coastal currents. For example, within the WC
ecosystem, the Salish Sea (which includes the Strait of Georgia east of
Vancouver Island, B.C. and Puget Sound, Washington) is largely isolated from
coastal ocean currents and the oceanography of this region is strongly
influenced by freshwater discharge from the Fraser River, which brings
land-derived nutrients into coastal waters and can influence water column
stability \citep{Hickey2008a}. Despite this difference in oceanographic
conditions in the Salish Sea compared to other parts of the West Coast
ecosystem, we found that the single-stock effects of the BI and NPGO did not
differ substantially between stocks that first enter salt water in the Salish
Sea and stocks that enter the ocean elsewhere within the West Coast ecosystem.
Thus, the effects of conditions encountered by Salish Sea stocks outside of the
Salish Sea appear to dominate the effects of oceanographic conditions that are
unique to the Salish Sea.

Horizontal transport pathways that are controlled by changes in the NPC do not
appear to substantially contribute to variability in salmon productivity for
stocks in what we called the GOA ecosystem. For instance, we found no support
for consistent effects of either NPC index on productivity of salmon stocks in
this region. This finding is in contrast to the strong and consistent positive
effects of the NPC indices on productivity in the WC ecosystem, which further
suggests that different environmental pathways may control productivity of
higher-trophic-level species in these two ecosystems. This hypothesis, that
different mechanisms may control salmon productivity in different ecosystems, is
supported by several previous studies that have also indicated differences in
the effects of environmental variables on salmon productivity between the WC and
GOA ecosystems. For example, \citet{Mueter2002a} showed opposite effects of sea
surface temperature on pink, chum, and sockeye salmon in northern and southern
areas, with the dividing line occurring around southern Southeast Alaska.
Similarly, \citet{Malick2015a} indicated that the phenology of the spring bloom
of phytoplankton has effects of opposite sign on pink salmon stocks in Alaska
compared to stocks in British Columbia. Both \citet{Mueter2002a} and
\citet{Malick2015a} suggested that the different effects may be driven by
differences in oceanography between the WC and GOA ecosystems, which is further
supported by our results. Differences in the effects of the NPC between northern
and southern areas further suggests that prey availability for salmon in the GOA
may either not be affected by variability in horizontal transport or prey
availability may not be limiting.

In the Bering Sea, we found less consistent effects of the BI and NPGO on salmon
productivity than in the West Coast ecosystem. For example, we found that (1)
neither index had a strong effect on sockeye salmon productivity, (2) the NPGO
was positively related to chum salmon productivity, and (3) the estimated
effects of both indices on pink salmon productivity were highly uncertain.
Because the BS ecosystem is geographically isolated from the NPC, it is unlikely
that variability associated with the NPC has a direct effect on salmon
productivity in this region. Instead, the BI and NPGO likely represent indirect
indicators of broad atmospheric or oceanographic patterns, such as changes in
the North Pacific Oscillation, that link conditions in the Bering Sea to the
mid-latitude conditions driving variability in the NPC \citep{DiLorenzo2013a}.

Finally, our results demonstrate that ocean current patterns and horizontal
transport pathways can strongly influence Pacific salmon stocks, with the
strongest effects being observed for stocks in Washington and British Columbia.
This conclusion, combined with previous research, indicates that in some areas
multiple environmental pathways may underlie changes in salmon productivity,
where one set of pathways is mediated by vertical ocean transport (e.g.,
upwelling) and another set is mediated by horizontal ocean transport
\citep{Ottersen2010a, Malick2015b, DiLorenzo2013b}. Furthermore, our results
provide some evidence that the relative importance of horizontal transport
pathways may differ between northern and southern areas for the species we
considered. Taken together, this suggests that quantifying the relative
importance and cumulative effects of multiple environmental pathways is
important for understanding how future environmental change will influence
production of higher-trophic-level species.



\section{Acknowledgments}

We are thankful to the many scientists from the Alaska Department of Fish and
Game, Fisheries and Oceans Canada, and Washington Department of Fish \& Wildlife
who collected and provided us with the salmon data sets. Funding for this
research was provided by Simon Fraser University. We also thank Marc Trudel and
an anonymous referee for their useful comments on our manuscript.



\section{Supporting materials}

\subsection{Supporting figures and tables}


\begin{table}[!ht]
  \small \centering \libertineLF
  \caption[Model selection quantities for models fit with the bifurcation index
           calculated using an expanded grid of drifters]{Model selection
           quantities from the sensitivity analysis that used the bifurcation
           index calculated using an expanded grid of drifters. \# gives the
           model number as defined in Table \ref{tab:npc:1} of the main text; Np
           gives the nominal number of parameters; pD gives the effective number
           of parameters; and $\Delta$WAIC gives the WAIC value for each model
           relative to the model with the minimum WAIC value.}
  % latex table generated in R 3.3.1 by xtable 1.8-2 package
% Wed Nov  2 13:09:48 2016
\begin{tabular}{llllllllll}
  \hline
  \multicolumn{1}{r}{} &
            \multicolumn{3}{c}{Pink} &
            \multicolumn{3}{c}{Chum} &
            \multicolumn{3}{c}{Sockeye} \\ 
  \cmidrule(lr){2-4} \cmidrule(lr){5-7} \cmidrule(lr){8-10} 
  \multicolumn{1}{l}{\#} & 
    \multicolumn{1}{l}{Np} & 
    \multicolumn{1}{l}{pD} & 
    \multicolumn{1}{l}{$\Delta$WAIC} & 
    \multicolumn{1}{l}{Np} & 
    \multicolumn{1}{c}{pD} & 
    \multicolumn{1}{c}{$\Delta$WAIC} & 
    \multicolumn{1}{c}{Np} & 
    \multicolumn{1}{c}{pD} & 
    \multicolumn{1}{l}{$\Delta$WAIC} \\ 
  \cmidrule{1-10} 
1 & 95 & 71 & 263.4 & 109 & 89 & 298.0 & 131 & 104 & 592.8 \\ 
  2 & 140 & 115 & 55.1 & 161 & 133 & 186.0 & 194 & 164 & 291.5 \\ 
  3 & 96 & 71 & 257.6 & 110 & 85 & 162.6 & 132 & 102 & 331.7 \\ 
  4 & 141 & 113 & 46.1 & 162 & 134 & 53.9 & 195 & 164 & 9.7 \\ 
  5 & 186 & 152 & 104.5 & 214 & 175 & 28.4 & 258 & 218 & 19.8 \\ 
  6 & 199 & 98 & 233.0 & 227 & 123 & 268.1 & 271 & 149 & 553.2 \\ 
  7 & 244 & 136 & 11.9 & 279 & 159 & 157.6 & 334 & 206 & 249.0 \\ 
  8 & 200 & 98 & 222.0 & 228 & 118 & 149.4 & 272 & 137 & 319.8 \\ 
  9 & 245 & 133 & \textbf{0.0} & 280 & 159 & 30.8 & 335 & 189 & \textbf{0.0} \\ 
  10 & 290 & 175 & 58.5 & 332 & 202 & \textbf{0.0} & 398 & 251 & 3.7 \\ 
   \hline
\end{tabular}

  \label{tab:npc:s1}
\end{table}


\begin{table}[!ht]
  \small \centering \libertineLF
  \caption[Model selection quantities for models fit with the bifurcation index
           calculated using the alternative method]{Model selection quantities
           from the sensitivity analysis that used the bifurcation index
           calculated using the methods outlined in \citet{Watters2008a}. \#
           gives the model number as defined in Table \ref{tab:npc:1} of the
           main text; Np gives the nominal number of parameters; pD gives the
           effective number of parameters; and $\Delta$WAIC gives the WAIC value
           for each model relative to the model with the minimum WAIC value.}
  % latex table generated in R 3.3.1 by xtable 1.8-2 package
% Tue Nov  1 15:27:38 2016
  \multicolumn{1}{r}{} &
            \multicolumn{3}{c}{Pink} &
            \multicolumn{3}{c}{Chum} &
            \multicolumn{3}{c}{Sockeye} \\ 
  \cmidrule(lr){2-4} \cmidrule(lr){5-7} \cmidrule(lr){8-10} 
  \multicolumn{1}{l}{\#} & 
    \multicolumn{1}{l}{Np} & 
    \multicolumn{1}{l}{pD} & 
    \multicolumn{1}{l}{$\Delta$WAIC} & 
    \multicolumn{1}{l}{Np} & 
    \multicolumn{1}{c}{pD} & 
    \multicolumn{1}{c}{$\Delta$WAIC} & 
    \multicolumn{1}{c}{Np} & 
    \multicolumn{1}{c}{pD} & 
    \multicolumn{1}{l}{$\Delta$WAIC} \\ 
\cmidrule{1-10} \endhead 
1 & 95 & 71 & 252.3 & 109 & 89 & 292.4 & 131 & 104 & 583.1 \\ 
  2 & 140 & 115 & 44.0 & 161 & 133 & 180.5 & 194 & 164 & 281.8 \\ 
  3 & 96 & 71 & 246.4 & 110 & 85 & 157.1 & 132 & 102 & 322.0 \\ 
  4 & 141 & 113 & 34.9 & 162 & 134 & 48.3 & 195 & 164 & \textbf{0.0} \\ 
  5 & 186 & 152 & 93.3 & 214 & 175 & 22.9 & 258 & 218 & 10.1 \\ 
  6 & 199 & 96 & 236.2 & 227 & 119 & 268.6 & 271 & 147 & 555.7 \\ 
  7 & 244 & 135 & 10.2 & 279 & 154 & 156.8 & 334 & 205 & 246.3 \\ 
  8 & 200 & 97 & 228.1 & 228 & 114 & 150.1 & 272 & 137 & 328.7 \\ 
  9 & 245 & 133 & \textbf{0.0} & 280 & 154 & 29.7 & 335 & 190 & 2.5 \\ 
  10 & 290 & 174 & 56.0 & 332 & 197 & \textbf{0.0} & 398 & 251 & 4.4 \\ 
  
  \label{tab:npc:s2}
\end{table}
\clearpage % force tables to come before figures


\begin{figure}[htbp]
  \centering \includegraphics[scale=0.9]{3_npcurrent/figures/S_drifter_grid.pdf}
  \caption[Starting location for the OSCURS model drifters used to calculate the
           bifurcation index]{Starting location for the OSCURS model drifters
           used to calculate the bifurcation index. Black dots show the
           locations where the drifters were seeded for the main bifurcation
           index and grey plus signs indicate the additional drifter starting
           locations for the bifurcation index calculated using the expanded
           grid of drifters, which was part of a sensitivity analysis examining
           changes in assumptions underlying the bifurcation index.}
  \label{fig:npc:s1}
\end{figure}


\begin{figure}[htbp]
  \centering \includegraphics[scale=0.20]{3_npcurrent/figures/drifter_tracks1.jpg}
  \caption*{Figure \ref{fig:npc:s2}: Continued on next page \ldots}
\end{figure}

\begin{figure}[htbp]
  \centering \includegraphics[scale=0.20]{3_npcurrent/figures/drifter_tracks2.jpg}
  \caption[OSCURS model drifter trajectories for years 1967--2010]{OSCURS model
           drifter trajectories for years 1967--2010. Each panel shows the
           drifter tracks for a single year where the drifters were simulated
           from February 1 to June 30. Red tracks indicate drifters that ended
           south of their starting location and blue tracks indicate drifters
           that ended north of their starting location. The lower left corner in
           each panel gives the year and the bifurcation index (BI) for that
           year. The BI values are standardized to a mean of 0 and a standard
           deviation of 1 (i.e., standard deviation units, SDUs).}
  \label{fig:npc:s2}
\end{figure}

\begin{figure}[htbp]
  \centering \includegraphics[scale=0.9]{3_npcurrent/figures/S_gamma_single.pdf}
  \caption[Single-stock model coefficients for the BI and NPGO]{Single-stock
           model coefficients for the BI and NPGO (i.e., $\gamma_i$). Gamma
           coefficients were estimated by fitting generalized Ricker models
           (i.e., eq. \ref{eq:npc:2}) to each of the 163 salmon stocks
           separately using maximum likelihood. Coefficients (in standard
           deviation units) are shown for pink salmon (panel a), chum salmon
           (panel b), and sockeye salmon (panel c). Within each panel,
           stock-specific estimates are grouped by ecosystem and stocks are
           ordered south (left) to north (right) where the stock number (x-axis)
           corresponds to the numbers in Fig. \ref{fig:npc:1} of the main text.
           Solid circles (black) indicate the maximum likelihood estimate for
           the BI effect, $\gamma_{BI,i}$, and solid squares (red) indicate the
           NPGO effect, $\gamma_{NPGO,i}$. Points inside the grey boxes indicate
           salmon stocks that enter the ocean in the Salish Sea.}
  \label{fig:npc:s3}
\end{figure}
