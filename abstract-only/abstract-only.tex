% Thesis Title + Abstract
% Michael Malick
% 2017-02-23


\documentclass[11pt]{report}
\usepackage{graphicx}
\usepackage{setspace}
\usepackage[showframe=false,
            includefoot,       % ensure page numbers do not extend into margins
            includehead,       % ensure page numbers do not extend into margins
            headheight=13.6pt, % need for fancyhdr
            top=1.0in,
            bottom=1.0in,
            left=1.25in,
            right=1.25in ]{geometry}

%% Libertine font
%% http://get-software.net/fonts/newtx/doc/newtxdoc.pdf (pg. 7)
\usepackage[lining,semibold]{libertine} % a bit lighter than Times--no osf in math
\usepackage[T1]{fontenc} % best for Western European languages
\usepackage{textcomp} % required to get special symbols
\usepackage[varqu,varl]{inconsolata}% a typewriter font must be defined
\usepackage{amsthm}% must be loaded before newtxmath
\usepackage[libertine,cmintegrals,bigdelims,vvarbb]{newtxmath}
\usepackage[scr=rsfso]{mathalfa}
\usepackage{bm}% load after all math to give access to bold math
%% After loading math package, switch to osf in text.
\useosf % for osf in normal text



% 1.5 line spacing
\onehalfspacing

\begin{document}
\pagenumbering{gobble}

\begin{center}
  \subsection*{Multi-scale environmental forcing of Pacific salmon population
               dynamics}

  Michael J. Malick

\end{center}

\subsection*{Abstract}

Understanding how environmental forcing governs the productivity of marine and
anadromous fish populations is a central, yet elusive, problem in fisheries
science. In this thesis, I use a cross-system comparative approach to
investigate how environmental forcing pathways could link climatic and ocean
processes to dynamics of Pacific salmon (\textit{Oncorhynchus} spp.) populations
in the Northeast Pacific Ocean. I begin by showing that phytoplankton phenology
and ocean current patterns are both strongly associated with inter-annual
changes in salmon productivity, suggesting that two alternative environmental
pathways may contribute to changes in salmon productivity:~one mediated by
vertical ocean transport and subsequent phytoplankton dynamics and the other
mediated by horizontal ocean transport and subsequent advection of plankton into
coastal areas. The relative importance of these pathways, however, may vary over
large spatial scales because the magnitude and direction of the estimated
environmental effects on productivity were conditional on the latitude of
juvenile salmon ocean entry. I then use a probabilistic network modeling
approach to show that changes in climatic and ocean processes can impact salmon
productivity via multiple concurrent environmental pathways, including multiple
pathways originating from the same climatic process. Finally, I use policy
analysis to demonstrate why efforts to integrate highly migratory species, such
as Pacific salmon, into ecosystem-based management policies need to explicitly
account for mismatches between the scale of ecosystem services provided by these
species and the scale at which human activities and natural processes impact
those services. Collectively, my thesis provides empirical evidence that
accounting for spatial heterogeneity and the relative importance of
simultaneously operating environmental pathways may be critical to developing
effective management and conservation strategies that are robust to future
environmental change.

\end{document}
