% General Introduction
% Michael Malick
% 2016-12-11

\chapter[General introduction]{General introduction}
\label{ch:intro}

Environmental forcing can have profound impacts on the provisioning of ecosystem
services generated by marine and coastal ecosystems. Yet, deep uncertainties
about the coupling among climate systems, physical and biological ocean
processes, and productivity\footnotemark[1] of higher-trophic-level species
limits our ability to anticipate or quickly detect impacts of changing
environmental conditions on commercially valuable species. These uncertainties
contribute to risks that have implications for conservation, harvest management,
and users of living marine resources. Effectively reducing uncertainties about
the links among different ecosystem components requires a quantitative
understanding about how perturbations in large-scale climatic and atmospheric
conditions propagate to regional and local scale changes in the population
dynamics of exploited species. In this thesis, I aim to add to that quantitative
understanding by applying a cross-system comparative approach to examine
environmental forcing pathways linking climatic and ocean processes to dynamics
of Pacific salmon (\emph{Oncorhynchus} spp.) populations in the Northeast
Pacific Ocean.

\footnotetext[1]{Throughout this thesis, the term productivity refers to the per
  capita growth rate for a population. For Pacific salmon, productivity is
  estimated as the number of recruits produced per spawner.}


\section{Large-scale environmental change}

Environmental change in marine and coastal ecosystems can arise from
anthropogenic sources or natural environmental stochasticity and can manifest as
gradual or abrupt changes in mean conditions or changes in the frequency or
distribution of extreme events \citep{Jentsch2007}. For example, gradual changes
in climate systems over the past five decades due to increased carbon dioxide
concentrations in the atmosphere have resulted in warmer mean atmosphere and
ocean temperatures, decreased snow and ice pack, rising sea levels, changes in
precipitation patterns, increased ocean acidification, and increased frequency
of extreme temperature events \citep{IPCC2013a}. The inter-decadal rate of
change for many of these abiotic ecosystem components is unprecedented with
equally rapid changes also being observed for biological processes including
shifts in phenology, species distributions, and fish stock productivity
\citep{IPCC2013a, Taylor2008a, Pinsky2013, Peterman2012}.

Concurrent with climate and ecosystem changes from anthropogenic forcing are
changes resulting from natural climate variability. In the Northeast Pacific,
large-scale climate patterns, e.g., the Pacific Decadal Oscillation and North
Pacific Gyre Oscillation, at least partially control the dynamics of marine and
coastal ecosystems. Fluctuations in these patterns, often referred to as regime
shifts, can substantially alter the structure and function of ecosystems that
comprise the Northeast Pacific \citep{Chavez2003a}. For instance, a rapid
ecological shift occurred in the Northeast Pacific in response to a climatic
regime shift in 1976/1977 (i.e., the Pacific Decadal Oscillation shifted from a
``cool regime'' to a ``warm regime''), which resulted in a taxonomic
reorganization in the Northeast Pacific where the abundances of wild adult
sockeye salmon (\emph{O. nerka}) and pink salmon (\emph{O. gorbuscha}) increased
by more than 65\% \citep{Ruggerone2010a, Anderson1999a, Mueter2000a}.

For Pacific salmon, effects of environmental change due to perturbations in
large-scale climatic conditions are mainly hypothesized to influence survival of
pre-recruit life stages. In particular, the first year of marine residency for
Pacific salmon is considered a critical period, i.e., mortality during this
life-stage can have a disproportionately large affect on overall stock
productivity compared to other life-stages \citep{Parker1968a, Peterman1985a,
Beamish2001a, Wertheimer2007a}. Although both bottom-up\footnotemark[2] and
top-down\footnotemark[3] forcing likely contribute to mortality during this
critical period, two pieces of evidence suggest that processes controlling food
resource availability are particularly important. First, juvenile salmon
mortality during the early marine life-stage is size selective, with larger
juveniles tending to survive to adult life-stages in higher proportions than
smaller juvenile salmon \citep{Parker1971a, Holtby1990a, McGurk1996a,
Moss2005a}. Second, growth rates during the early marine life-stage are strongly
and positively associated with overall marine survival rates \citep{Cross2008a,
Duffy2011, Farley2007b}. Together, this evidence suggests that large-scale
climatic perturbations likely have a strong impact on Pacific salmon year class
strength through bottom-up forcing pathways \citep{Perry1996a, Armstrong2005a}.

\footnotetext[2]{The term `bottom-up forcing' is used throughout this thesis to
  describe regulation of ecosystem structure and function through processes that
  affect the base of the food chain, such as nutrient supply and primary
  production.}

\footnotetext[3]{The term `top-up forcing' is used throughout this thesis to
  describe regulation of ecosystem structure and function occurring through
  predation.}


\section{Environmental forcing pathways}

A prevailing bottom-up forcing pathway in marine ecosystems posits that vertical
transport processes mediate the effects of climate variability on phytoplankton
dynamics in coastal ecosystems and subsequently, food resource availability for
juvenile Pacific salmon (Fig. \ref{fig:intro:1}) \citep{DiLorenzo2013b,
Rykaczewski2008a, Ware1991a}. In particular, atmospheric and ocean processes
controlling water column stability and the near surface nutrient supply are
frequently cited as key elements driving phytoplankton dynamics in coastal
Northeast Pacific ecosystems \citep{Henson2007a, Gargett1997a}. For example, in
coastal upwelling areas, winds drive surface waters offshore through Ekman
dynamics, causing nutrient rich subsurface water to upwell into the euphotic
zone, providing necessary nutrients for primary production \citep{Huyer1983}. In
turn, this primary production provides grazing opportunities for copepods and
other zooplankton, which are a critical food resource for juvenile Pacific
salmon during their early marine residency \citep{Armstrong2008a,
Beauchamp2007a, Brodeur2007a}. Over the past two decades, considerable evidence
has indicated strong connections between climate variability, vertical ocean
transport processes, and phytoplankton dynamics \citep{Chenillat2012,
Polovina1995a, Henson2007a, Henson2007b, Stabeno2004a, Weingartner2002a}.
However, relationships between lower-trophic-level process (e.g., phytoplankton
dynamics in coastal ecosystems) and productivity of Pacific salmon populations
largely remain untested assumptions. In chapter \ref{ch:bloom}, I investigate
the vertical transport hypothesis by asking whether the phenology or intensity
of the spring phytoplankton bloom can explain variability in productivity of 27
North American pink salmon stocks.

\begin{figure}[htbp]
  \centering
  \includegraphics[scale=0.4]{1_introduction/figures/vertical-horizontal.pdf}
  \caption[Schematic of two environmental forcing pathways]{Schematic of two
           environmental forcing pathways linking large-scale climate patterns,
           ocean processes, and higher-trophic-level species.}
  \label{fig:intro:1}
\end{figure}

Recently, evidence for an alternative bottom-up forcing pathway has emerged,
suggesting that horizontal ocean transport may be equally important as vertical
transport in mediating the effects of climate variability on
higher-trophic-level species \citep{DiLorenzo2013b}. This horizontal transport
hypothesis proposes that food resources available to juvenile salmon in coastal
ecosystems is driven by climate-induced changes in horizontal transport
processes, e.g., ocean currents or eddies, that cause zooplankton or other
weakly/passive drifters to be advected into or out of coastal areas (Fig.
\ref{fig:intro:1}). For example, off the central Oregon Coast, research has
indicated that the negative phase of the Pacific Decadal Oscillation is
associated with increased advection of large-bodied lipid-rich zooplankton into
the region from northern areas, which in turn is associated with increased
marine survival of coho salmon (\emph{O. kisutch}) \citep{Keister2011a,
Bi2011a}. Beyond the Northern California Current area, however, the effects of
variability in horizontal ocean transport on Pacific salmon productivity are
largely untested. In chapter \ref{ch:npc}, I investigate the horizontal
transport hypothesis by examining the effects of two modes of variability in
horizontal ocean transport in the Northeast Pacific on productivity of 163 North
American pink, chum (\emph{O. keta}), and sockeye salmon stocks.

Although the vertical and horizontal transport pathways are individually
appealing to explain how climate forcing downscales to affect regional and local
scale dynamics of higher-trophic-level species, these hypotheses are not
mutually exclusive and may have additive or multiplicative effects on salmon
productivity. In particular, regional-scale vertical and horizontal transport
processes are both hypothesized to mediate the effects of large-scale climate
variability on lower- and higher-trophic-level species. Thus, perturbations to
climatic systems from anthropogenic or natural sources may simultaneously
influence regional-scale vertical and horizontal transport pathways. Indeed, the
Pacific Decadal Oscillation has been shown to influence ocean current patterns
in the Northern California Current ecosystem and affect the magnitude of
upwelling-favorable winds in the region \citep{Keister2011a, Chhak2007}.
Estimating the cumulative effects and relative importance of these
simultaneously operating pathways is likely a necessary component of
understanding how environmental change impacts higher-trophic-level species. In
chapter \ref{ch:bn}, I use a novel quantitative method, probabilistic
networks, to estimate the cumulative effects of these different pathways on
productivity of coho salmon in the Northern California Current.


\section{Managing for environmental change}

A better understanding of how environmental forcing impacts salmon populations
is a necessary but not sufficient condition for maintaining viable and
productive salmon stocks. We also need to develop a parallel understanding of
how these impacts interact with other anthropogenic disturbances, such as
commercial harvesting, and how this information can be incorporated into
management decisions \citep{Link2002a}. Increasingly, management of living
marine resources is moving toward ecosystem-based approaches to management that
shift the focus of management from a single-species to maintaining critical
components of ecosystem structure and function \citep{Grumbine1994,
Murawski2007a, Long2015}. A necessary element of this shift toward
ecosystem-based management is defining boundaries that delimit the spatial
extent of the decision-making process \citep{Engler2015, Yaffee1999}. However,
for highly migratory marine and anadromous fish species, impacts from human or
natural sources can occur across a continuum of spatial scales that frequently
extend beyond the boundaries of the ecosystem-based management area
\citep{Dallimer2015}. For example, management actions in locations that are
geographically distance from the ecosystem-based management area, such as
decisions to increase commercial harvests, may strongly impact the supply of
ecosystem services provided by a migratory species within the bounds of the
ecosystem-based management area. In chapter \ref{ch:ebm}, I examine challenges
associated with integrating highly migratory Pacific salmon into regional and
local scale ecosystem-based management policies that arise from mismatches
between the scale of management and the biology of Pacific salmon and discuss
potential strategies to overcome the identified challenges.


\section{Statement of interdisciplinarity}

The research presented in this thesis includes two levels of
interdisciplinarity. First is the incorporation of research ideas, perspectives,
and approaches from both oceanography and fisheries. Although both fields are
firmly rooted in the natural sciences, the research approaches and the types of
questions important to researchers in both fields have diverged over time
\citep{Platt2007a}. In chapters \ref{ch:bloom}--\ref{ch:bn}, I attempt to
bring together some of the knowledge and research questions important to both
fisheries scientists and oceanographers. The second level of interdisciplinarity
involves a bridge between the natural and social sciences. One-quarter of the
research presented in this thesis is focused on this bridging by taking a policy
perspective in order to answer an important fisheries question.


\section{Contributions}

I am the sole author of the general introduction and general conclusions and
these chapters are written in the first-person singular. Chapters
\ref{ch:bloom}--\ref{ch:ebm} are derived from either published manuscripts or
submitted manuscripts with co-authors and these chapters are written in the
first-person plural. For each of the chapters deriving from multi-authored
manuscripts (chapters \ref{ch:bloom}--\ref{ch:ebm}), I am the first-author of
the work and performed the data analysis and wrote the first draft of the text.
These chapters, however, benefited greatly from discussions, editing, and
comments from the co-author. The original published sources of these chapters
are provided at the beginning of each chapter. The initial ideas for chapter
\ref{ch:bloom} were developed by myself, Randall Peterman, Franz Mueter and Sean
Cox. Chapter \ref{ch:npc} builds on ideas originally presented in an unpublished
manuscript by Randall Peterman, Franz Mueter, and Brigitte Dorner. The main idea
for chapter \ref{ch:bn} came out of discussions between myself and Randall
Peterman following a presentation on using Bayesian networks for ecological
research by Catherine Michielsens. The ideas presented in chapter \ref{ch:ebm}
were developed by myself, Murray Rutherford, and Sean Cox.

