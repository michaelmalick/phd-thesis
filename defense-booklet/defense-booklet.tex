% Thesis Defense Booklet
% Michael Malick
% 2017-04-20


\documentclass[11pt]{report}
\usepackage{graphicx}
\usepackage{setspace}
\usepackage[usenames,dvipsnames]{color}
\usepackage[colorlinks=true,urlcolor=MidnightBlue]{hyperref}
\usepackage{url}
\urlstyle{rm}
\usepackage[showframe=false,
            includefoot,       % ensure page numbers do not extend into margins
            includehead,       % ensure page numbers do not extend into margins
            headheight=13.6pt, % need for fancyhdr
            top=1.0in,
            bottom=1.0in,
            left=1.15in,
            right=1.15in ]{geometry}

\usepackage{enumitem}
\setlist[description]{font=\normalfont\small,leftmargin=50pt,style=nextline, topsep=1em,itemsep=1pt}

%% Libertine font
%% http://get-software.net/fonts/newtx/doc/newtxdoc.pdf (pg. 7)
\usepackage[lining,semibold]{libertine} % a bit lighter than Times--no osf in math
\usepackage[T1]{fontenc} % best for Western European languages
\usepackage{textcomp} % required to get special symbols
\usepackage[varqu,varl]{inconsolata}% a typewriter font must be defined
\usepackage{amsthm}% must be loaded before newtxmath
\usepackage[libertine,cmintegrals,bigdelims,vvarbb]{newtxmath}
\usepackage[scr=rsfso]{mathalfa}
\usepackage{bm}% load after all math to give access to bold math
%% After loading math package, switch to osf in text.
\useosf % for osf in normal text



% 1.5 line spacing
\onehalfspacing

\begin{document}
\pagenumbering{gobble}

\begin{center}
  \section*{Multi-scale environmental forcing of Pacific salmon population
               dynamics}

  Michael J. Malick

  Ph.D. Thesis Defense

  May 16, 2017, 12:30 \textsc{pm}

  SFU Burnaby, TASC 1, Room 8219

\end{center}

\subsection*{Abstract}

Understanding how environmental forcing governs the productivity of marine and
anadromous fish populations is a central, yet elusive, problem in fisheries
science. In this thesis, I use a cross-system comparative approach to
investigate how environmental forcing pathways could link climatic and ocean
processes to dynamics of Pacific salmon (\textit{Oncorhynchus} spp.) populations
in the Northeast Pacific Ocean. I begin by showing that phytoplankton phenology
and ocean current patterns are both strongly associated with inter-annual
changes in salmon productivity, suggesting that two alternative environmental
pathways may contribute to changes in salmon productivity:~one mediated by
vertical ocean transport and subsequent phytoplankton dynamics and the other
mediated by horizontal ocean transport and subsequent advection of plankton into
coastal areas. The relative importance of these pathways, however, may vary over
large spatial scales because the magnitude and direction of the estimated
environmental effects on productivity were conditional on the latitude of
juvenile salmon ocean entry. I then use a probabilistic network modeling
approach to show that changes in climatic and ocean processes can impact salmon
productivity via multiple concurrent environmental pathways, including multiple
pathways originating from the same climatic process. Finally, I use policy
analysis to demonstrate why efforts to integrate highly migratory species, such
as Pacific salmon, into ecosystem-based management policies need to explicitly
account for mismatches between the scale of ecosystem services provided by these
species and the scale at which human activities and natural processes impact
those services. Collectively, my thesis provides empirical evidence that
accounting for spatial heterogeneity and the relative importance of
simultaneously operating environmental pathways may be critical to developing
effective management and conservation strategies that are robust to future
environmental change.

\subsection*{Summary of Program of Study}
\begin{description}
  \item[Current] Ph.D. Candidate, School of Resource and Environmental
    Management, Simon Fraser University, Burnaby, BC

  \item[2008] M.Sc., School of Fisheries and Ocean Sciences, University of Alaska
    Fairbanks, Juneau, AK

  \item[2006] B.Sc., Mansfield University, Mansfield, PA (summa cum laude)
\end{description}


\subsection*{Selected Publications}
\begin{description}
  \item[2017] Malick, M.J., S.P. Cox, F.J. Mueter, B. Dorner, and R.M. Peterman.
    2017. Effects of the North Pacific Current on the productivity of 163
    Pacific salmon stocks. Fisheries Oceanography 26:268--281
    \url{http://doi.org/10.1111/fog.12190}

  \item[2016] Malick, M.J. and S.P. Cox. 2016. Regional-scale declines in
    productivity of pink and chum salmon stocks in western North America. PLoS
    ONE 11:e0146009 \url{http://doi.org/10.1371/journal.pone.0146009}

  \item[2015] Malick, M.J., S.P. Cox, R.M. Peterman, T.C. Wainwright, and W.T.
    Peterson. 2015. Accounting for multiple pathways in the connections among
    climate variability, ocean processes, and coho salmon recruitment in the
    Northern California Current. Canadian Journal of Fisheries and Aquatic
    Sciences 72:1552--1564 \url{http://doi.org/10.1139/cjfas-2014-0509}

  \item[2015] Malick, M.J., S.P. Cox, F.J. Mueter, and R.M. Peterman. 2015.
    Linking phytoplankton phenology to salmon productivity along a north-south
    gradient in the Northeast Pacific Ocean. Canadian Journal of Fisheries and
    Aquatic Sciences 72:697--708 \url{http://doi.org/10.1139/cjfas-2014-0298}

\end{description}


\subsection*{Selected Presentations}
\begin{description}
  \item[2016] Malick, M.J., S.P. Cox, F.J. Mueter, B. Dorner, and R.M. Peterman.
    Effects of the North Pacific Current on productivity of 163 Pacific salmon
    stocks. Poster. PICES Annual Meeting, San Diego, CA. November 15, 2016,
    Awarded Best Poster Presentation in Fisheries Science Section
    (\href{https://s3-us-west-2.amazonaws.com/michaelmalick-com-public/slides/malick_etal_pices2016_poster.pdf}{Poster})

  \item[2016] Malick, M.J. Environmental drivers of spatial and temporal
    variability in Pacific salmon productivity. Resource and Environmental
    Management Departmental Seminar, Simon Fraser University, Burnaby, BC. March
    7, 2016
    (\href{https://s3-us-west-2.amazonaws.com/michaelmalick-com-public/slides/malick_sfu_seminar_2016.pdf}{Slides})

  \item[2015] Malick, M.J., S.P. Cox, F.J. Mueter, and R.M. Peterman. Linking
    phytoplankton phenology to pink salmon productivity along a north-south
    gradient. American Fisheries Society Annual Meeting, Portland, OR. August
    17, 2015, Awarded Best Student Presentation, Honorable Mention
    (\href{https://s3-us-west-2.amazonaws.com/michaelmalick-com-public/slides/malick_etal_afs2015_slides.pdf}{Slides},
    \href{https://s3-us-west-2.amazonaws.com/michaelmalick-com-public/slides/malick_etal_afs2015_extended_abstract.pdf}{Extended Abstract})

\end{description}


\subsection*{Statement of Interdisciplinarity}

The research presented in my thesis includes two levels of interdisciplinarity.
First is the incorporation of research ideas, perspectives, and approaches from
both oceanography and fisheries. Although both fields are firmly rooted in the
natural sciences, the research approaches and the types of questions important
to researchers in each field have diverged over time. In chapters 2--4 of my
thesis, I attempt to bring together some of the knowledge and research questions
important to both fisheries scientists and oceanographers. The second level of
interdisciplinarity involves a bridge between the natural and social sciences.
One-quarter of the research presented in my thesis is focused on this bridging
by taking a policy perspective to examine potential strategies to more
effectively integrate multi-scale information about natural and anthropogenic
disturbances into ecosystem-based management programs.

\end{document}
