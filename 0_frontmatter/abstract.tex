%%%%%%%%%%%%%%%%%%%%%%%%%%%%%%%%%%%%%%%%%%%%%%%%%%%%%%%%%%%%%%%%%%%%%%%%%%%%%%%%
% PHD DISSERTATION ABSTRACT
%
%
% Michael Malick
% 2016-12-02
%
%%%%%%%%%%%%%%%%%%%%%%%%%%%%%%%%%%%%%%%%%%%%%%%%%%%%%%%%%%%%%%%%%%%%%%%%%%%%%%%%


\chapter*{Abstract}
\addcontentsline{toc}{chapter}{Abstract}


A central problem in fisheries science is understanding how environmental
forcing influences demographic rates of marine and anadromous fish populations.
Unprecedented changes in climate systems over the past few decades and
subsequent changes in marine ecosystems including shifts in phenology, species
distributions, and fish stock productivity, highlight in


In this thesis, I examine environmental forcing pathways linking climatic and
ocean processes to dynamics of Pacific salmon (\textit{Oncorhynchus} spp.)
populations in the Northeast Pacific Ocean. I begin by using a cross-system
comparative approach to assess the evidence for population-level responses to
inter-annual changes in phytoplankton dynamics. I show that the phenology of the
spring phytoplankton bloom (and to a lesser extent phytoplankton biomass) is
strongly associated with salmon productivity, although the direction of the
effect was dependent on the geographic location of juvenile salmon ocean entry.
Next, I again use a cross-system comparative approach to examine an alternative
hypothesis that horizontal ocean transport is a critical driver of Pacific
salmon productivity. I show that horizontal ocean transport can strongly affect
salmon productivity for stocks in Washington and British Columbia, but not for
more northern stocks in Alaska. Next, I use a novel quantitative method,
probabilistic networks, to examine the cumulative effects and relative
importance of multiple environmental pathways on salmon dynamics. I show that
multiple environmental pathways can simultaneously impact salmon population
dynamics, including multiple pathways that originate from the same climatic
process. Finally, I use a policy perspective to show that mismatches in the
scale at which ecosystem services are provided by Pacific salmon and the scale
at which human activities and natural processes impact those services need to be
explicitly accounted for ecosystem-based management policies. My thesis
demonstrates that climatic and ocean processes can impact higher-trophic-level
species via simultaneously operating environmental pathways---including pathways
mediated by either vertical or horizontal ocean transport---and that the
relative importance of these pathways can be non-stationary through space.
