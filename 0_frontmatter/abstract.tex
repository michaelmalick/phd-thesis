%%%%%%%%%%%%%%%%%%%%%%%%%%%%%%%%%%%%%%%%%%%%%%%%%%%%%%%%%%%%%%%%%%%%%%%%%%%%%%%%
% PHD DISSERTATION ABSTRACT
%
%
% Michael Malick
% 2016-12-11
%
%%%%%%%%%%%%%%%%%%%%%%%%%%%%%%%%%%%%%%%%%%%%%%%%%%%%%%%%%%%%%%%%%%%%%%%%%%%%%%%%


\chapter*{Abstract}
\addcontentsline{toc}{chapter}{Abstract}

Environmental change resulting from natural or anthropogenic forcing can have
profound impacts on the structure and function of marine and coastal ecosystems.
Yet, determining environmental processes governing population dynamics of marine
and anadromous fish species remains an elusive problem that has practical
management implications. In this thesis, I use a cross-system comparative
approach to examine environmental forcing pathways linking climatic and ocean
processes to dynamics of Pacific salmon (\textit{Oncorhynchus} spp.) populations
in the Northeast Pacific Ocean. I begin by assessing the evidence for
population-level responses to inter-annual changes in two meso-scale ocean
processes, phytoplankton dynamics and ocean currents, which represent critical
links within two alternative sets of environmental forcing pathways. In the
first set of pathways, vertical ocean transport and subsequent phytoplankton
dynamics are hypothesized to mediate the effects of climate variability on
higher-trophic-level species, whereas in the second set of pathways, climate
effects are hypothesized to be mediated by horizontal ocean transport and
subsequent advection of plankton into coastal areas. I show that both
phytoplankton dynamics and ocean current patterns are strongly associated with
changes in salmon productivity, indicating that both sets of hypothesized
pathways may drive salmon dynamics. However, the magnitude and direction of the
effects were conditional on the latitude of juvenile salmon ocean entry,
suggesting that the relative importance of different environmental pathways may
be region dependent. Next, I use a novel quantitative method, probabilistic
networks, to examine the joint effect and relative strength of 17 potential
environmental pathways linking large-scale climate processes to Pacific salmon
dynamics. I show that multiple environmental pathways can simultaneously impact
salmon population dynamics, including multiple pathways originating from the
same climatic process. Finally, I use a policy perspective to examine how
challenges arising from a highly migratory life history can impede efforts to
integrate Pacific salmon into ecosystem-based management policies. My findings
indicate that ecosystem-based management policies should explicitly account for
mismatches in the scale at which ecosystem services are provided by highly
migratory species and the scale at which human activities and natural processes
impact those services to achieve more effective integration. Collectively, my
thesis demonstrates that climatic and ocean processes can impact
higher-trophic-level species via multiple simultaneously operating environmental
pathways and accounting for spatial heterogeneity in the relative importance of
these pathways may be critical to developing effective management strategies
that are robust to future environmental change.
\newline

\noindent \textbf{Keywords:} Pacific salmon; population dynamics; environmental
change; spatial non-stationarity; ecosystem-based management; productivity

