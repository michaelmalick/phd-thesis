Date: 2016-10-27

\chapter{General Introduction}\label{general-introduction}

Environmental change can have profound impacts on the provisioning of
ecosystem services generated by marine and coastal ecosystems. Yet, deep
uncertainties about the coupling among climatic change, physical and
biological ocean processes, and the productivity of higher-trophic-level
species limits our ability to anticipate or quickly detect impacts from
environmental change on commercially valuable species. These
uncertainties contribute to risks, which have implications for
conservation, harvest management, and users of the resource. Fundamental
to minimizing uncertainties about the effects of environmental change on
higher-trophic-level species is developing a quantitative understanding
about how perturbations in large-scale climatic and atmospheric
conditions propagate to regional and local scale changes in the
population dynamics of exploited species. In this thesis, I aim to add
to that quantitative understanding by using a cross-system comparative
approach to examine ecological pathways linking climate and ocean
processes to the population ecology of North American Pacific salmon
(\emph{Oncorhynchus} spp.).

\section{Cross-scale environmental
change}\label{cross-scale-environmental-change}

Environmental change in marine and coastal ecosystems can arise from
anthropogenic sources or natural environmental stochasticity and can
manifest as gradual or abrupt changes in mean conditions or changes in
the frequency or distribution of extreme events. For example, gradual
changes in climate systems over the past five decades due to increased
carbon dioxide concentrations in the atmosphere have resulted in warmer
mean atmosphere and ocean temperatures, decreased snow and ice pack,
rising sea levels, changes in precipitation patterns, increased ocean
acidification, and increased frequency of extreme temperature events
{[}@IPCC2013a{]}. The inter-decadal rate of change for many of these
abiotic ecosystem components is unprecedented with equally rapid changes
also being observed for biological processes including shifts in
phenology, species distributions, and fish stock productivity
{[}@IPCC2013a{]}.

Concurrent with climate and ecosystem changes from anthropogenic sources
are changes resulting from natural climate variability. In the North
Pacific, large-scale climate patterns, e.g., the Pacific Decadal
Oscillation and North Pacific Gyre Oscillation, at least partially
control the dynamics of marine and coastal ecosystems. Fluctuations in
these patterns, often referred to as regime shifts, can substantially
alter the structure and function of ecosystems that comprise the North
Pacific {[}@Chavez2003a{]}. For instance, following a major reversal of
the Pacific Decadal Oscillation in 1976/77 the species composition of
the North Pacific shifted from a crustacean-dominated system to a gadid
and flatfish-dominated system {[}@Anderson1999a; @Mueter2000a{]}. Also,
the abundances of wild adult sockeye salmon (\emph{O. nerka}) and pink
salmon (\emph{O. gorbuscha}) in the North Pacific increased by more than
65\% after the regime shift {[}@Ruggerone2010a{]}.

For Pacific salmon, effects of environmental change due to perturbations
in climatic conditions are largely hypothesized to influence survival of
pre-recruit life stages. In particular, the first year of ocean
residency for Pacific salmon is considered a critical period, i.e.,
mortality during this life-stage can have a disproportionately large
affect on overall stock productivity compared to other life-stages
{[}@Parker1968a; @Peterman1985a; @Beamish2001a; @Wertheimer2007a{]}.
Although both bottom-up\footnote{The term `bottom-up forcing' is used
  throughout the thesis to describe regulation of ecosystem structure
  and function through processes that affect the base of the food chain,
  such as nutrient supply and primary production.} and
top-down\footnote{The term `top-up forcing' is used throughout the
  thesis to describe regulation of ecosystem structure and function
  occuring through predation.} forcing likely contribute to mortality
during this critical period, two pieces of evidence suggest that
processes controlling food resource availability are particularly
important. First, juvenile salmon mortality during the early marine life
stage is size selective, with larger juveniles tending to survive to
adult life stages in higher proportions than smaller juvenile salmon
{[}@Parker1971a; @Holtby1990a; @McGurk1996a; @Moss2005a{]}. Second,
growth rates during the early marine life-stage are strongly and
positively associated with overall marine survival rates {[}@Cross2008a;
@Duffy2011; @Farley2007b{]}. Together, this suggests that large-scale
climatic perturbations likely have a strong impact on Pacific salmon
year class strength through bottom-up forcing pathways {[}@Perry1996a;
@Armstrong2005a{]}.

\section{Ecological pathways}\label{ecological-pathways}

A prevailing bottom-up forcing pathway in marine ecosystems posits that
vertical transport processes mediate the effects of climate variability
on phytoplankton dynamics in coastal ecosystems and subsequently, food
resource availability for juvenile Pacific salmon {[}@DiLorenzo2013b;
@Rykaczewski2008a; @Ware1991a{]}. In particular, atmospheric and ocean
processes controlling water column stability and the near surface
nutrient supply are frequently cited as key elements driving
phytoplankton dynamics in coastal North Pacific ecosystems
{[}@Henson2007a; @Gargett1997a{]}. For example, in coastal upwelling
areas, winds drive surface waters offshore through Ekman dynamics,
causing nutrient rich subsurface water to upwell into the euphotic zone,
providing necessary nutrients for primary production {[}@Huyer1983{]}.
In turn, this primary production provides grazing opportunities for
copepods and other zooplankton, which are a critical food resource for
juvenile Pacific salmon {[}@Armstrong2008a; @Beauchamp2007a;
@Brodeur2007a{]}. Over the past two decades, considerable evidence has
indicated strong connections between climate variability, vertical ocean
transport processes, and phytoplankton dynamics {[}@Chenillat2012;
@Polovina1995a; @Henson2007a; @Henson2007b; @Stabeno2004a;
@Weingartner2002a{]}. However, relationships between lower-trophic-level
process, such as phytoplankton dynamics in coastal ecosystems, and
productivity of Pacific salmon populations largely remains an untested
assumption. In chapter 1, I investigate this assumption of the vertical
transport hypothesis by asking whether the phenology or intensity of the
spring phytoplankton bloom can explain variability in salmon
productivity. In particular, I use satellite derived chl-a estimates to
examine the effects of spring bloom initiation date and phytoplankton
biomass on productivity of 27 pink salmon populations.

Recently, an alternative bottom-up forcing pathway has emerged that
suggests that horizontal ocean transport may be equally important as
vertical transport in mediating the effects of climate variability on
higher-trophic-level species {[}@DiLorenzo2013b{]}. The horizontal
transport hypothesis suggests that food resources available to juvenile
salmon is driven by climate-induced changes in horizontal transport
processes, e.g., ocean currents or eddies, that cause zooplankton or
other weakly/passive drifters to be advected into or out of coastal
ecosystems. For example, off the central Oregon Coast, research has
indicated that the negative phase of the Pacific Decadal Oscillation is
associated with increased advection of large-bodied lipid-rich
zooplankton into the region from northern areas, which in turn is
associated with increased marine survival of coho salmon
{[}@Keister2011a; @Bi2011a{]}. Beyond the Northern California Current
area, however, the effects of variability in horizontal ocean transport
on Pacific salmon productivity are largely untested. In chapter 2, I
examine the effects of two modes of variability in horizontal ocean
transport in the North Pacific on productivity of 163 North American
pink, chum, and sockeye salmon stocks.

Although the vertical and horizontal transport pathways are individually
appealing to explain how climate forcing downscales to affect local
dynamics of higher-trophic-level species, these hypotheses are not
mutually exclusive and may have additive or multiplicative effects on
salmon productivity. In particular, regional-scale vertical and
horizontal transport processes both mediate the effects of large-scale
climate variability on lower- and higher-trophic level species. Thus,
large-scale climate patterns, such as the Pacific Decadal Oscillation,
may simultaneously influence regional-scale vertical and horizontal
transport pathways. Indeed, the Pacific Decadal Oscillation has been
shown to influence both ocean current patterns in the Northern
California ecosystem and affect the magnitude of upwelling-favorable
winds in the region. The coupling among different ecological pathways
thus creates a complex ecological network that links climate patterns,
ocean conditions, and salmon population dynamics and estimating the
cumulative effects and relative importance of the pathways within this
network is a necessary component of understanding how environmental
change impacts higher-trophic-level species. In chapter 3, I estimate
the cumulative effects of these pathways on productivity of coho salmon
in the Northern California Current using Bayesian network.

\section{Managing for environmental
change}\label{managing-for-environmental-change}

A better understanding of how salmon and the environment interact is a
necessary but not sufficient condition for maintaining viable and
productive salmon stocks. We also need to develop a parallel
understanding of how this ecological information can be incorporated
into management decision making {[}@Link2002a{]}. Due to complex and
non-linear interactions of ecosystem components, management of living
marine resources is increasingly moving towards a more holistic
ecosystem-based approach to management. Ecosystem-based management (EBM)
approaches are often recommended to overcome perceived failures of
traditional single-sector resource management by explicitly taking into
account cumulative impacts of human activities on ecosystem structure
and function {[}@Grumbine1994; @Murawski2007a; @Long2015{]}.
Implementing an EBM approach that is holistic and integrative across
sectors requires defining boundaries that delimit the spatial extent of
the decision-making process {[}@Engler2015; @Yaffee1999{]}. In other
words, EBM is a type of place-based (or area-based) management that
ideally has ecologically defined boundaries. Ecosystem boundaries,
whether defined for management purposes or research purposes, are
porous, allowing organisms, materials, and energy to move across the
defined boundaries {[}@ONeill2001{]}. In particular, migratory marine
and anadromous fish species often move across management and
jurisdictional boundaries throughout their life cycles, potentially
leading to mismatches between the scale of management and the biology of
a fish stock {[}@Cash2006a{]}. Because migratory species may spend a
considerable portion of time outside the human defined boundaries of an
ecosystem, factors external to the management area may strongly
contribute to the cumulative impacts of human activities for these
species {[}@Dallimer2015{]}. In chapter 4, I investigate challenges
associated with integrating highly migratory Pacific salmon into
ecosystem-based management policies and outline potential solutions to
these challenges.

\section{Contributions}\label{contributions}

I am the sole author of chapters xxx and xxx (general introduction and
discussion, respectively) and these chapters are written in the
first-person singular. Chapters xxx-xxx are derived from either
published manuscripts or submitted manuscripts with co-authors and these
chapters are written in the first-person plural. For each of the
chapters deriving from multi-authored manuscripts (chapters xxx-xxx), I
am the first-author of the work and performed the data analysis and
wrote the first draft of the text. These chapters, however, benefited
greatly from discussions, editing, and comments from the co-author. The
original published sources of these chapters are provided at the
beginning of each chapter. The initial ideas for chapter (bloom) were
developed by myself, Randall Peterman, Franz Mueter and Sean Cox.
Chapter (npc) builds on ideas originally presented in an unpublished
manuscript by Randall Peterman, Franz Mueter, and Brigitte Dorner. The
idea for chapter (bn) came out of discussions between myself and Randall
Peterman following a presentation on Bayesian networks by Catherine
Michielsens. The ideas presented in chapter (policy) were initially
developed by myself, Murray Rutherford, and Sean Cox.

\chapter{References}\label{references}
