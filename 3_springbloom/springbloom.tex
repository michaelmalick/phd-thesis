% Thesis Spring Bloom Chapter
% Michael Malick
% 2015-10-12

\chapter[Linking phytoplankton phenology to salmon productivity]{Linking
  phytoplankton phenology to salmon productivity along a north-south gradient in
  the Northeast Pacific Ocean\footnotemark[1]}

\footnotetext[1]{A version of this chapter appears as 
  Malick, M.J., S.P. Cox, F.J. Mueter, R.M. Peterman. 2015. Linking
  phytoplankton phenology to salmon productivity along a north-south gradient in
  the Northeast Pacific Ocean. Canadian Journal of Fisheries and Aquatic
  Sciences 72:697-708. \url{http://doi.org/10.1139/cjfas-2014-0298}.}
  

\section{Abstract}

We investigated spatial and temporal components of phytoplankton dynamics in the
Northeast Pacific Ocean to better understand the mechanisms linking biological
oceanographic conditions to productivity of 27 pink salmon (\emph{Oncorhynchus
gorbuscha}) stocks. Specifically, we used spatial covariance functions in
combination with multi-stock spawner-recruit analyses to model relationships
among satellite-derived chlorophyll-a concentrations, initiation date of the
spring phytoplankton bloom, and salmon productivity. For all variables, positive
spatial covariation was strongest at the regional scale (0-800 km) with no
covariation beyond 1500 km. Spring bloom timing was significantly correlated
with salmon productivity for both northern (Alaska) and southern (British
Columbia) populations, although the correlations were opposite in sign. An early
spring bloom was associated with higher productivity for northern populations
and lower productivity for southern populations. Furthermore, the spring bloom
initiation date was always a better predictor of salmon productivity than mean
chlorophyll-a concentration. Our results suggest that changes in spring bloom
timing resulting from natural climate variability or anthropogenic climate
change could potentially cause latitudinal shifts in salmon productivity.



\section{Introduction}

The dynamics of marine fish populations are often characterized by large
inter-annual and inter-decadal variability in abundances. For Pacific salmon
(\emph{Oncorhynchus} spp.), the first year of ocean residence is widely viewed
as a critical period that can strongly influence stock abundance
\citep{Peterman1985a, Parker1968a, Wertheimer2007a}. During this period,
climatic and oceanographic conditions are believed to strongly affect salmon
productivity (i.e., the number of adult recruits produced per spawner), yet the
ecological pathways connecting environmental variability to upper trophic levels
of marine food webs are not well understood \citep{Drinkwater2010a,
Ottersen2010a}. Evidence suggests that salmon mortality during the early marine
life stage is inversely related to body size, indicating that bottom-up forcing
mechanisms that affect prey resources may be an important part of these
ecological pathways \citep{McGurk1996a, Duffy2011, Farley2007b}.

Several bottom-up forcing mechanisms have been proposed to explain productivity
variation in marine fish stocks, including salmon \citep{Cushing1990a,
Gargett1997a}. For example, the ``optimal stability window'' hypothesis suggests
that changes in water column stability may be a critical component linking
changes in large-scale climate patterns and salmon productivity
\citep{Gargett1997a}. However, this hypothesis assumes a strong link between
phytoplankton dynamics (e.g., productivity or total biomass) and salmon
productivity, which is largely untested beyond a few local-scale studies
\citep{Mathews1989a, Chittenden2010a}.  Accounting for both spatial and temporal
variability of lower-trophic-level processes is a key challenge to testing the
optimal stability window hypothesis on large spatial scales.

In the coastal Northeast Pacific, seasonal biomass of phytoplankton follows a
well-known pattern defined primarily by the spring bloom \citep{Henson2007a,
Waite2013}, which is mainly driven by large-scale climate patterns combined with
regional and local-scale physical environmental conditions \citep{Sverdrup1953a,
Ware1991a, Polovina1995a, Henson2007a}. In the coastal Gulf of Alaska, the
spring bloom initiation date is strongly correlated with the onset of water
column stability, which is at least partially controlled by the strength of the
Aleutian Low Pressure system \citep{Henson2007a}. In that region, an earlier
spring bloom is also associated with a more intense bloom, suggesting that both
the phenology of the spring bloom and overall production during the bloom may be
important components of bottom-up forcing pathways. Indeed, features of the
spring bloom such as initiation date and total phytoplankton biomass are
correlated with yield and productivity of certain marine fish populations
\citep{Platt2003a, Ware2005a, Koeller2009}.

In this paper, we asked whether the spring bloom initiation date and average
chlorophyll-a (chl-a) concentrations (a surrogate for phytoplankton biomass) can
explain spatial and inter-annual variability in pink salmon (\emph{O.
gorbuscha}) productivity, which we estimated using spawner-recruit data for 27
stocks. Establishing a plausible mechanistic link between the spring
phytoplankton bloom and salmon production first requires evidence that the two
processes operate at similar spatial scales; that is, spatial covariation of
lower-trophic level processes should approximately match the spatial scale of
covariation observed in the salmon productivity data that they are being used to
explain \citep{Bjornstad1999a, Koenig1999a}. In the Northeast Pacific Ocean,
productivity of salmon stocks exhibit spatial synchrony at the scale of 100 to
1000 km \citep{Mueter2002b} with positive correlations being greatest at
distances less than 500 km \citep{Pyper2005a}. Therefore, we hypothesized that
features of the spring phytoplankton bloom operate at similar regional spatial
scales as salmon productivity.

We used spatial covariance analyses to determine the spatial extent of synchrony
in the timing of the spring phytoplankton bloom and mean chl-a concentrations
along the Northeast Pacific coast, as well as to determine the scale of spatial
averaging that should be used on data for these variables. We then used a
hierarchical, multi-stock statistical modeling approach to estimate
relationships between pink salmon productivity and inter-annual variability in
spring bloom initiation date, as well as mean chl-a concentration during spring
and late summer.  Compared to single-stock analyses, our multi-stock modeling
approach can help reduce uncertainty associated with the biological processes
that underpin the dynamics of salmon populations and reduce the chance of
finding spurious relationships by using different salmon stocks as replicates
within the analysis \citep{Myers1998c, Myers1999a}.



\section{Methods}

\subsection{Pink salmon data}

We estimated annual indices of productivity (in units of adult recruits per
spawner) for 27 wild pink salmon stocks from British Columbia (B.C.) and Alaska
(AK) using data on spawner abundance and total recruitment (catch plus
escapement). The 27 spawner-recruit data sets (Table \ref{tab:bloom:1})
represent aggregations of escapement and catch of adjacent salmon stocks. The
aggregation helped ensure that catches were attributed to the correct spawning
stocks and was primarily based on jurisdictional management units, although in
some cases aggregation occurred at a larger scale because of difficulty
allocating catch into individual management units (e.g., Prince William Sound).
Hatchery returns were excluded from all estimates of catch and escapement.
Estimation methods for spawner abundances varied among stocks, but in general,
southern stocks (B.C.) were estimated using expansions of foot surveys, while
northern stocks (AK) were estimated using expansions of aerial surveys (personal
communication from data sources listed in footnotes of Table \ref{tab:bloom:1}).
Annual recruitment varied widely among stocks with long-term averages ranging
from 0.12 million pink salmon for Chignik Bay to 33.31 million for southern
southeast Alaska (Table \ref{tab:bloom:1}).

\subsection{Chlorophyll-a data}

We used satellite-derived chl-a concentration estimates (measured as mg
m\textsuperscript{-3}) from the Sea-viewing Wide Field-of-view Sensor (SeaWiFS)
and the Moderate Resolution Imaging Spectroradiometer (MODIS-Aqua) from the
Goddard Space Flight Center (\url{http://oceancolor.gsfc.nasa.gov}). Level-3
processed data were downloaded for 1998-2010 (but only 2003-2010 had complete
MODIS data) in their original 9 km\textsuperscript{2}, 1-day-resolution. We
converted the data to a 1$^{\circ}$x1$^{\circ}$ resolution and subsetted the
resulting grid to 46$^{\circ}$--61$^{\circ}$N and 167$^{\circ}$--125$^{\circ}$W,
including only grid cells adjacent to the coast (Fig.  1). We excluded grid
cells in the Bering Sea because all salmon stocks in our data set enter the
ocean in the Gulf of Alaska.

All analyses were performed using 8-day composite chl-a data because these had
less missing data across all years compared to 1-day and 5-day composites
(Supporting materials Fig. \ref{fig:bloom:s1}). In addition, the SeaWiFS data
set had numerous large gaps during the spring and summer for years 2008-2010,
which made these years of SeaWiFS data unsuitable for our study (Supporting
materials Fig. \ref{fig:bloom:s1}). We evaluated the feasibility of
concatenating the SeaWiFS and MODIS data sets into a single continuous data set
that would provide an additional three years of data (compared to using SeaWiFS
data alone) by comparing the two data sets over the first five overlapping years
(2003-2007; Supporting materials \ref{supp:bloom:A}). We estimated correlations,
root mean squared log\textsubscript{10} error (RMSE), and log\textsubscript{10}
bias to quantify differences between the two chl-a data sets. The SeaWiFS and
MODIS data sets were highly correlated (average of 0.87 across all grid cells).
In addition, over all years and grid cells RMSE (0.16) and log bias (0.012)
(Supporting materials Fig. \ref{fig:bloom:s2}) were consistent with other
studies comparing SeaWiFS and MODIS data products over a similar study region
\citep{Waite2013}. Based on these minimal differences, we concatenated the
SeaWiFS (1998-2002) and MODIS (2003-2010) data sets without further processing.

Inter-annual variability in phytoplankton standing stock and phytoplankton
phenology were quantified using mean monthly chl-a concentration and the
spring-bloom initiation date, respectively, which we derived from the 8-day
composite chl-a estimates for each grid cell (Fig. \ref{fig:bloom:1}). We
linearly interpolated data points in chl-a time series for each grid cell
between gaps less than 3 data points (4.1\% of all chl-a data points were
interpolated). This procedure was done prior to estimating annual spring bloom
initiation date and mean chl-a concentrations. We estimated the spring bloom
initiation date as the first 8-day period in a given year when the chl-a
concentration was more than 5\% above the median chl-a concentration of the
entire multi-year data set for a particular grid cell \citep{Siegel2002a,
Henson2007a}. In addition, we log\textsubscript{10}-transformed the chl-a
averages to help normalize the chl-a values.


\subsection{Spatial covariation analysis}

We constructed cross-correlation matrices to quantify spatial covariance
patterns for (1) pink salmon stock productivities, (2) spring-bloom initiation
dates, and (3) monthly mean chl-a concentrations. For salmon stocks, pairwise
correlation coefficients were computed between time series of productivity for
each of the 27 stocks. For the spring bloom initiation date, correlation
coefficients were computed between each pair of grid cells using time series of
the estimated annual spring bloom initiation date for years 1998-2010. For chl-a
concentration, we calculated correlations across grid cells using time series of
the monthly mean chl-a concentration. To account for potential changes in
spatial patterns across seasons, we calculated correlations for chl-a
concentration for each month (February-October) separately.

We estimated annual salmon productivity using residuals from a Ricker
spawner-recruit model, which removed potential within-stock density-dependent
effects \citep{Pyper2001a, Mueter2002a, Ricker1954a}.  The Ricker model for each
stock was of the form,

\begin{equation}
log_e(R_{i,t} / S_{i,t}) = \alpha_i + \beta_i S_{i,t} + \epsilon_{i,t},
\label{eq:bloom:1}
\end{equation}

\noindent where \(R_{i,t}\) is total pink salmon recruits for the \(i^{th}\)
stock in brood year \(t\), \(S_{i,t}\) is the spawning stock two years earlier,
\(\alpha_i\) is the maximum log\textsubscript{e} recruits-per-spawner,
\(\beta_i\) is the coefficient of density-dependence, and \(\epsilon_{i,t}\) is
the residual.

We fit the Ricker models (eq. \ref{eq:bloom:1}) to two partitions of the data --
one including all available brood years and the second including only brood
years 1997-2009 (Table \ref{tab:bloom:1}). The latter partition corresponds to
the years available for the bloom initiation date and chl-a variables. Because
juvenile pink salmon enter the ocean the year following spawning (i.e., brood
year + 1), we offset the phytoplankton variables one year to correspond with the
ocean entry year for pink salmon (e.g., 2006 brood year was lined up with 2007
phytoplankton variables).

To test whether spatial covariation was present in each of the variables, we
first performed Mantel tests using matrices of the cross-correlations and a
matrix of great-circle distance (computed using the haversine formula) between
all pairs of grid cells or stocks \citep{Legendre1998a, Koenig1999a}.
Statistical significance of Mantel statistics were determined using
randomization tests with 1 000 permutations. We then determined the spatial
scale of covariation for each significant Mantel test by fitting a smooth
non-parametric covariance function \citep{Bjornstad2001a} between the
correlation coefficients for a given variable and the distance separating
correlated grid cells or ocean-entry points of salmon stocks. Confidence
intervals (CI) for each covariance function were computed by bootstrapping the
estimation procedure 1 000 times.

Covariance functions were summarized using two distance metrics: (1) the
y-intercept of the covariance function, which provides an estimate of the
correlation at zero distance (CZD), and (2) 50\% correlation scale (D50). The
CZD was estimated by extrapolating the fitted covariance function to zero
distance to find the y-axis intercept. The D50 was estimated as the distance at
which the covariance function falls to 50\% of its observed maximum value, which
provides a useful metric of how much the correlation declines with increasing
distance between salmon stocks or grid cells \citep{Mueter2002b}.


\subsection{Salmon productivity models}

We used a combination of single-stock linear models and multi-stock linear
mixed-effects models to investigate relationships among temporal averages of
mean chl-a concentrations (spring and late summer), spring bloom initiation
date, and pink salmon productivity. The single-stock model analysis had two
purposes. First, we used the values of fitted coefficients for different stocks
to inform construction of the multi-stock models and to help evaluate the
multi-stock model assumptions. Second, we used the single-stock analysis along
with intervention analyses to break up the data sets spatially to provide the
best fits of the multi-stock models. For both single-stock and multi-stock
models, only pink salmon brood years 1997-2009 were used.

The bloom initiation date and both chl-a covariates included in the models
represented spatial and temporal (for chl-a) averages of conditions experienced
by juvenile pink salmon during their early marine life phase. The bloom
initiation covariate was calculated for each salmon stock as the average of grid
cell specific anomalies (i.e., a grid cell's value minus the long term mean for
that grid cell) over all grid cells whose centers were within 250 km of the
stock's ocean entry point. For chl-a, we calculated April-May averages to
capture variability in phytoplankton biomass during the spring bloom and
July-September averages to index chl-a variability during the late summer, which
is believed to be a critical period for juvenile salmon survival
\citep{Beamish2001a, Moss2005a}. For both time periods, we first averaged chl-a
values over the specified months for each grid cell and then averaged over all
grid cells within 250 km of the stock's ocean entry point.


\subsubsection{Single-stock models}

The single-stock Ricker models took the form \citep{Adkison1996b},

\begin{equation}
log_e(R_{i,t} / S_{i,t}) = \alpha_i + \beta_i S_{i,t} + \gamma_i
X_{i,t+1} + \epsilon_{i,t}, \label{eq:bloom:2}
\end{equation}

\noindent where \(S_{i,t}\) is spawner abundance of pink salmon in brood year
\(t\) for the \(i^{th}\) stock, \(R_{i,t}\) is the total recruitment resulting
from \(S_{i,t}\), \(\alpha_i\) indicates stock productivity at low spawner
abundances, \(\beta_i\) indicates the magnitude of density-dependence,
\(X_{i,t+1}\) is a stock-specific measure of either the spring bloom initiation
date or mean chl-a concentration (the latter for either the spring or late
summer), \(\gamma_i\) is the coefficient for either the stock-specific bloom
initiation date or mean chl-a, and \(\epsilon_{i,t} \sim N(0, \sigma^2)\) is an
independent and identically distributed residual term.

Environmental variables such as sea surface temperature could have opposite
effects on northern and southern pink salmon stocks \citep{Mueter2002a,
Su2004a}; therefore we used an intervention model with two means
\citep{Chatfield2004, Mueter2002a} to test for differences in the effect of the
bloom initiation date and chl-a concentration between northern and southern
stocks \citep{Chatfield2004, Mueter2002a}. The intervention models were fit to
either the estimated chl-a or spring bloom coefficients from the single-stock
models (i.e., \(\gamma\) in eq. \ref{eq:bloom:2}) where the coefficients were
arranged south to north based on ocean entry locations (i.e., by stock number in
Table \ref{tab:bloom:1}).


\subsubsection{Multi-stock models}

We used hierarchical, multi-stock models to estimate both regional and
stock-specific effects of spring and late summer chl-a and the bloom initiation
date on pink salmon productivity, while also accounting for heterogeneity in
density-dependence among stocks. The multi-stock mixed effects Ricker models
took the form \citep{Myers1999a, Mueter2002a},

\begin{equation}
log_e(R_{i,t}/S_{i,t}) = \alpha + a_i - \beta_iS_{i,t} + 
X_{i,t+1} (\gamma_{X} + g_{i}) + \epsilon_{i,t}, \label{eq:bloom:3}
\end{equation}

\noindent where the fixed intercept \(\alpha\) is the overall mean productivity
common to all stocks and \(a_i\) is the stock-specific deviation from that mean,
\(\beta_i\) is the fixed stock-specific density-dependent effect, \(X_{i,t+1}\)
represents either the spring bloom initiation date or mean chl-a concentration
(either spring or late summer average), \(\gamma_X\) is the overall mean effect
of either the spring bloom initiation date or mean chl-a concentration, \(g_i\)
is the stock-specific deviation from that overall mean for a particular chl-a
variable, and \(\epsilon_{i,t}\) is an independent and identically distributed
residual term (i.e., \(\epsilon_{i,t} \sim N(0,\sigma^2)\)).  The stock-specific
random effects \(a_i\) and \(g_i\) are assumed to follow a joint normal
distribution with means zero, variances \(\sigma^2_a\) and \(\sigma^2_g\), and
covariance \(\sigma^2_{ag}\).

Because the chl-a and bloom initiation variables were moderately correlated
(average correlations between stock-specific phytoplankton time series ranged
from -0.50 to 0.20), the multi-stock models were fit separately for the bloom
initiation date and both chl-a metrics. For the bloom initiation date and spring
chl-a variables, we also fit multi-stock models separately for a southern stock
group (stocks 1-9 in Table \ref{tab:bloom:1}) and a northern stock group (stocks
10-27 in Table \ref{tab:bloom:1}), because the single-stock analysis and
intervention models suggested consistent differences in the effects of these
variables between northern and southern stock groupings (see Results). For the
late summer chl-a variable, we fit a single model using all stocks because the
intervention models did not indicate a significant break between northern and
southern stock groupings.

In addition to the full models (eq. \ref{eq:bloom:3}) for both chl-a variables
and the bloom initiation date, we investigated two simpler nested models, (1)
eq. \ref{eq:bloom:3} but without the random chl-a or bloom effect (i.e.,
\(g_{i}\)), and (2) eq. \ref{eq:bloom:3} but without either the random or fixed
chl-a or bloom effect (i.e., \(g_{i}\) and \(\gamma_{X}\), which was the null
model).  Random effect significance was determined using likelihood ratio (L)
tests among the nested models, whereas fixed effect significance was tested
using F-tests \citep{Pinheiro2000a}. All reported parameters were estimated
using restricted maximum likelihood methods; however, for model comparisons,
parameters were estimated using maximum likelihood methods to reduce bias
\citep{Pinheiro2000a}.

To compare the relative importance of the bloom initiation date and both chl-a
variables, we also calculated the small-sample Akaike Information Criterion
(AIC\textsubscript{C}) for all models \citep{Hurvich1989a, Burnham2002a}. For
the models that included late summer chl-a, which were fit using all 27 salmon
stocks, we calculated a single set of AIC\textsubscript{C} values (one for each
nested model).  For the models that included either the bloom initiation date or
spring chl-a variables, we calculated two sets of AIC\textsubscript{C} values.
First, to compare the relative importance of both variables within the northern
and southern areas, we calculated AIC\textsubscript{C} values for each model fit
to the northern and southern stock groups separately.  Second, to compare
variable importance with the late summer chl-a variable, we calculated an
AIC\textsubscript{C} value for the combined northern and southern models.
Because northern and southern models for the bloom initiation date and spring
chl-a variables were fit using identical salmon data as the late summer chl-a
models, we calculated a combined northern and southern AIC\textsubscript{C}
value for each variable by summing the log-likelihoods and the number of model
parameters. To more easily compare models, we also calculated the
\(\Delta\)AIC\textsubscript{C}, i.e., the difference between each individual
model's AIC\textsubscript{C} value and the minimum AIC\textsubscript{C} value
among models. Models within three AIC\textsubscript{C} units of the model with
the lowest AIC\textsubscript{C} value are considered equally plausible
\citep{Burnham2002a}.


\subsection{Sensitivity analysis}

We checked the sensitivity of our results to four assumptions. First, we
estimated sensitivity of the spatial analysis results to an alternative
Beverton-Holt stock-recruitment model \citep{Beverton1957a}
(log\textsubscript{e}(R/S) = log\textsubscript{e}(a) - log\textsubscript{e}(1 +
bS) + \(\epsilon\)). Second, we checked the sensitivity to the interpolation
procedure used on the chl-a time series by re-running each analysis using spring
bloom and chl-a values that did not include interpolated data points. Third, we
tested our assumption that the error terms of the multi-stock models were
temporally independent by refitting the models with first-order autocorrelated
errors (i.e., \(\epsilon_{i,t} = \phi\epsilon_{i,t-1} + \upsilon_t\), where
\(\upsilon_t \sim N(0, \sigma^2)\)) and using likelihood ratio tests to
determine the significance. Fourth, because our spawner-recruit data sets
include variability associated with both freshwater and marine life phases, we
checked the sensitivity of our results to the source of pink salmon data by
comparing each chl-a metric to pink salmon marine survival rates for three
Alaska hatchery stocks (Armin F.Koernig, Kitoi Bay, and Port Armstrong) using
Pearson correlation coefficients (see Supporting materials \ref{supp:bloom:B}
for details of the analysis).



\section{Results}

\subsection{Spatial analysis}

Both sets of pink salmon residuals, monthly mean chl-a, and bloom initiation
date all showed significant spatial covariation (P \textless{} 0.01 for all
Mantel tests). For both sets of pink salmon residuals (all brood years and
recent, satellite-covered years), the nonparametric covariance functions
indicated declining positive covariation with increasing distance between ocean
entry points of juvenile salmon, up to approximately 800 to 1000 km where the
functions approached zero correlation (Fig. \ref{fig:bloom:2}a and b). The
estimated D50 was slightly larger for productivity indices fitted using all
available brood years (D50 = 305 km; 95\% CI = 218-488 km) than for indices
fitted using only brood years 1997-2009 (D50 = 261 km; 95\% CI = 148-628 km),
although there was considerable overlap in confidence intervals (Figs.
\ref{fig:bloom:2} and \ref{fig:bloom:3}a). Correlations at zero distance (i.e.,
the y-intercept of the covariance function) for both sets of productivity
indices were considerably less than one (CZD = 0.51; 95\% CI = 0.41-0.62 and CZD
= 0.49; 95\% CI = 0.28-0.69 for all brood years and 1997-2009 respectively; Fig.
\ref{fig:bloom:3}b). Although the nonparametric covariance function for bloom
initiation date had a slightly larger D50 (D50 = 367 km; 95\% CI = 235-776 km)
than the two salmon productivity indices, there was considerable overlap in
confidence intervals with both productivity indices (Fig. \ref{fig:bloom:3}a).
Correlation at zero distance for the bloom initiation date was also considerably
less than one (CZD = 0.44; 95\% CI = 0.33-0.55; Fig. \ref{fig:bloom:3}b).

For monthly mean chl-a concentrations, covariation decayed steeply with
increasing distance over spatial scales of 0-500 km for all months (Fig.
\ref{fig:bloom:4}).  The D50 was highest (\textasciitilde{}380 to 430 km) during
the winter and spring (February-May) and declined to about 250 km during summer
and fall (June-October), which was similar to the estimated D50 for salmon
productivity (Fig. \ref{fig:bloom:5}). In addition, confidence intervals for the
chl-a D50 for all months overlapped the confidence intervals for salmon
productivity D50s (Fig.  \ref{fig:bloom:5}). The CZD for chl-a concentrations
ranged from 0.56 in June to 0.76 in April, which was slightly higher than the
estimated CZD for the bloom initiation date and salmon productivity.


\subsection{Single-stock models}

The single-stock Ricker models indicated that pink salmon productivity was
related to the spring bloom initiation date either positively (15 stocks) or
negatively (12 stocks) (Fig. \ref{fig:bloom:6}a). The distribution of model
coefficients (i.e., \(\gamma\) in eq. \ref{eq:bloom:2}) ranged from -0.55 to
0.25 and was asymmetric about zero with the majority of values between -0.2 and
0.25 (Fig.  \ref{fig:bloom:6}d).  Productivity of all 9 pink salmon stocks south
of northern southeast Alaska (i.e., stocks 1-9 in Table \ref{tab:bloom:1}) was
positively related to the spring bloom initiation date, whereas productivity of
northern stocks was mostly negatively related (12 of 18 stocks; Fig.
\ref{fig:bloom:6}a). The intervention model indicated a significant break (P
\textless{} 0.05) in the sign of these relationships near 55.7$^{\circ}$N, which
was between the southern southeast Alaska stock (stock 9) and the northern
southeast Alaska outside stock (stock 10; Fig.  \ref{fig:bloom:1}).

Pink salmon productivity was also both positively (14 stocks) and negatively (13
stocks) related to spring chl-a concentrations (Fig. \ref{fig:bloom:6}b) with
the coefficients ranging from -3.4 to 6.0 (Fig. \ref{fig:bloom:6}e). Like the
bloom initiation date, the intervention model indicated a significant break (P
\textless{} 0.05) between stocks 9 and 10 for the spring chl-a coefficients
(Fig. \ref{fig:bloom:6}b). Productivity for all but two stocks in the southern
group had a negative relationship with spring chl-a concentrations, whereas the
northern group had a mix of positive and negative relationships.

For the late summer chl-a variable, productivity was consistently negatively
related to chl-a, with only 4 of the 27 stocks having a positive relationship
(Fig. \ref{fig:bloom:6}c). The coefficients were approximately normally
distributed with the magnitudes ranging from -8.6 to 3.2 with a median value of
-1.9 (Fig.  \ref{fig:bloom:6}f). In contrast to the other two phytoplankton
variables, the intervention model did not indicate a significant break in the
sign of the coefficients between northern and southern stocks (Fig.
\ref{fig:bloom:6}c).

The productivity parameters (\(\alpha\)) for the bloom initiation date and both
chl-a models were approximately normally distributed (a requirement for the
multi-stock models), but the distribution of the density-dependent coefficients
(\(\beta\)) had a long left tail (Supporting materials Fig. \ref{fig:bloom:s3}).


\subsection{Multi-stock models}

Over all pink salmon stocks we considered, productivity was significantly
related to spring bloom initiation date (northern and southern models), spring
chl-a concentrations (southern model only), and late summer chl-a (\(\gamma\) in
Table \ref{tab:bloom:2}), although stock-specific differences were not
significant in any models. For the bloom initiation date, regional effects were
opposite in sign for the northern and southern stock groups (\(\gamma\) in rows
2 and 5 in Table \ref{tab:bloom:2}, Fig. \ref{fig:bloom:6}a), suggesting that
salmon productivity for the southern stock group is higher than average when the
bloom is later (positive coefficient), whereas productivity is higher than
average for the northern stock group when the bloom is early (negative
coefficient). This result contrasts with those for spring (southern model only)
and late summer chl-a, where the regional effect was negative, implying reduced
salmon productivity when chl-a concentrations are higher (\(\gamma\) in rows 6
and 10 in Table \ref{tab:bloom:2}). Furthermore, the spring bloom initiation
date was a better predictor of salmon productivity than mean chl-a concentration
for all subsets of data (i.e., northern stock group, southern stock group, and
all stocks) (AIC\textsubscript{C} values in Table \ref{tab:bloom:2}).

Estimates of the regional effect of the spring bloom initiation date on salmon
productivity were significantly different than zero for both northern and
southern multi-stock models (\(\gamma\) in rows 2 and 5 in Table
\ref{tab:bloom:2}), but the models were not significantly different than the
full model (eq. \ref{eq:bloom:3}), which included both the regional and
stock-specific effects (L = 0.03, P \textgreater{} 0.1). The estimated
region-wide effect of spring chl-a concentrations on salmon productivity of
southern stocks and late summer chl-a on productivity of all stocks were
significantly different from zero (\(\gamma\) in rows 6 and 10 in Table
\ref{tab:bloom:2}). However, for both models and chl-a variables, there was no
evidence of stock-specific effects based on likelihood ratio tests comparing the
full models to a model without the random chl-a effects (L = 0.001, P
\textgreater{} 0.1 for both spring and late summer chl-a). In addition, for the
northern stock group there was no support for either a regional or
stock-specific effect of spring chl-a (row 3 in Table \ref{tab:bloom:2}; L =
0.001, P \textgreater{} 0.1).

In both the northern and southern areas, the bloom initiation date had a
stronger effect on pink salmon productivity than spring chl-a concentrations as
shown by the \(\Delta\)AIC\textsubscript{C} of 5.4 between the best fit models
for chl-a and the bloom initiation for the northern stock group and 12.5 for the
southern stock group (Table \ref{tab:bloom:2}).  The bloom initiation date also
had the highest relative importance (\(\Delta\)AIC\textsubscript{C} = 0) when
the northern and southern models were combined with an AIC\textsubscript{C}
value considerable less than late summer chl-a, spring chl-a, and the null model
(rows 7-10 in Table \ref{tab:bloom:2}). Between the two chl-a variables, the
late summer chl-a average had a higher relative importance (i.e., lower
AIC\textsubscript{C} value) than the average spring chl-a concentration as
indicated by the 10 unit difference between AIC\textsubscript{C} values (rows 9
and 10 in Table \ref{tab:bloom:2}).


\subsection{Sensitivity analysis}

Our estimates of the spatial covariation in pink salmon productivity were not
sensitive to the form of stock-recruit model because residuals from the Ricker
and Beverton-Holt models were highly correlated (average correlation across
stocks = 0.97). The D50 and CZD values were nearly identical between models fit
using the Ricker and Beverton-Holt residuals. Similarly, the spatial analyses
were not sensitive to the interpolation of data points in the chl-a time series.
Difference between D50 values for the interpolated and non-interpolated spring
bloom series was 10 km, with almost complete overlap of the confidence
intervals. In addition, changes in D50 for monthly chl-a without interpolation
values ranged from 0 km to 10 km across months with almost complete overlap of
the confidence intervals. Coefficients of the multi-stock Ricker model were also
insensitive to the interpolation of missing data.

The results from the multi-stock models were not sensitive to our initial
assumption of uncorrelated errors. Specifically, the single-stock models did not
indicate strongly autocorrelated errors, and adding an autocorrelated error term
to the best-fit multi-stock models did not significantly improve the fits for
any of the models, which is consistent with other research on pink salmon
productivity \citep{Pyper2001a}. In addition, comparisons between the three
chl-a metrics and hatchery marine survival rates broadly agreed with the results
of the multi-stock models (Supporting materials \ref{supp:bloom:B} and Fig.
\ref{fig:bloom:s4}).



\section{Discussion}

We investigated two indices of phytoplankton dynamics, spring bloom initiation
date and mean chl-a concentration, to better understand the potential mechanisms
linking biological oceanographic conditions to Pacific salmon productivity. Our
results indicated that (1) spatial covariation patterns for the spring bloom
initiation date, average chl-a concentration, and pink salmon productivity were
similar, with strongest positive covariation at the regional scale (0-800 km),
(2) there were opposing effects of the spring bloom initiation date on northern
and southern pink salmon stock productivity with an early bloom initiation date
being associated with higher northern stock productivity and a late bloom being
associated with higher southern stock productivity, (3) phytoplankton biomass
during the late summer (July-September) was more strongly associated with salmon
productivity than phytoplankton biomass during the spring (April-May), and (4)
the spring bloom initiation date was a better predictor of salmon productivity
than mean chl-a concentration for both southern and northern stocks.

Spatial synchrony for all three variables was strongest at regional spatial
scales and declined rapidly with increasing distance. For the bloom initiation
date and chl-a concentration, this suggests that physical processes operating on
relatively small spatial scales (e.g., summer sea surface temperature and sea
surface salinity) drives the spatial variability, rather than larger-scale
atmospheric processes such as the Pacific Decadal Oscillation
\citep{Mueter2002b}. For pink salmon productivity, our results suggest that both
phytoplankton biomass and the bloom initiation date could be factors driving the
regional-scale covariation. Furthermore, the match in spatial synchrony between
both phytoplankton variables and salmon productivity supports the inclusion of
these variables in the single-stock and multi-stock models and also lends
support for the observed correlations between pink salmon productivity and both
phytoplankton variables.

Spatial correlation of all three variables was less than one at zero distance,
indicating the presence of a ``nugget effect'', which represents variability due
to sampling error or spatial dependence at smaller scales than those sampled
\citep{Cressie1993}. For pink salmon productivity, this could be caused by
errors enumerating spawner abundances. For chl-a and the bloom initiation date,
the nugget effect may be caused by measurement errors in the chl-a estimates and
spatial averaging, or from the presence of small-scale oceanographic features
such as tidal mixing or river plumes that can lead to large changes in the bloom
initiation date and chl-a concentrations over short distances (\textless{} 100
km) \citep{Henson2007a}. The latter process could reduce the explanatory power
of both phytoplankton variables for salmon productivity because the
phytoplankton variables may not index conditions that salmon actually experience
at small spatial scales.

There was a marked reduction in the D50 for monthly chl-a concentrations between
May and June. The winter and spring period (February-May) included months with
both the highest annual chl-a concentrations (April and May) and the lowest
(February and March), whereas the chl-a concentrations during June-October were
relatively constant (Supporting materials Fig. \ref{fig:bloom:s5}). These
periods correspond to times before the spring bloom (February and March), during
it (April and May), and after it (June-October). Our results also showed that
coherence in chl-a concentrations following the spring bloom is smaller than
prior to and during the bloom, which may result from different mechanisms
underlying the spatial synchrony at different periods. For example, spatial
synchrony before and during the bloom may be primarily driven by regional-scale
physical oceanographic conditions such as sea surface temperature or sea surface
salinity \citep{Henson2007a}. In contrast, the period after the bloom also tends
to correspond to the period of peak zooplankton abundances in the Northeast
Pacific, indicating that chl-a concentrations after the bloom may be more
influenced by top-down grazing pressure, as suggested by others
\citep{Chittenden2010a, Bornhold2000, Mackas2012}.

A plausible explanation for the opposite effects of the bloom initiation date on
productivity of northern and southern stocks is that the spring bloom initiation
date is a surrogate for other processes that have direct effects on salmon
productivity such as predator abundances or zooplankton distributions. The
dividing line between the northern and southern stocks occurred in southern
southeast Alaska, which falls in the transition zone between the northern
downwelling domain and southern upwelling domain \citep{Ware1989a}. In the
northern region, the spring bloom initiation date has been shown to be closely
linked to the timing of water column stability, which is primarily determined by
freshwater runoff in the spring \citep{Weingartner2005a, Henson2007a}. Moreover,
both stability and the bloom initiation date in the northern domain tend to
occur earlier in warmer, wetter years that are associated with a more intense
Aleutian Low, higher zooplankton biomass, and increased salmon productivity
\citep{Brodeur1992a, Mueter2002a}. In the southern domain, an earlier spring
bloom is also associated with increased water column stability, however,
stability in this region is primarily driven by increased thermal warming and
reduced upwelling-favorable winds, both of which are also associated with a
stronger Aleutian Low \citep{Polovina1995a, Henson2007a}. In contrast to the
northern domain, these conditions in the south for an early bloom initiation
have been shown to be associated with increased predator abundances, reduced
zooplankton biomass, and decreased salmon productivity \citep{Ware1995a,
Mackas2001a, Mueter2002a}.

The optimal stability window hypothesis \citep{Cury1989a, Gargett1997a} provides
another possible explanation for the opposite effects of the bloom initiation
date on productivity of northern and southern stocks.  This bottom-up forcing
mechanism posits that synchronous changes in water column stability in northern
and southern areas, which are driven by strength of cyclonic circulation in the
Gulf of Alaska \citep{Gargett1997a}, can lead to out-of-phase salmon survival
rates between the two areas. However, the degree to which our results support
the optimal stability window hypothesis depends on the extent to which (1) water
column stability and the spring bloom initiation date are linked, (2) water
column stability has opposite effects on primary production in northern and
southern regions, and (3) there is a strong positive relationship between
phytoplankton biomass and salmon productivity. Although the first two
relationships are beyond the scope of this research, our results indicate that
there is only a weak relationship between phytoplankton biomass during the
spring and salmon productivity and a stronger but negative relationship between
phytoplankton biomass during the late summer and salmon productivity.

The latitudinal shift in the effect of the spring bloom initiation date on
northern and southern stock productivity corresponds with previous studies that
showed opposite effects of sea surface temperature on the productivity of
northern and southern pink salmon, chum salmon (\emph{O.  keta}), and sockeye
salmon (\emph{O. nerka}) stocks \citep{Mueter2002a, Su2004a}. In particular,
\citet{Mueter2002a} and \citet{Su2004a} indicated that warm sea surface
temperatures were associated with higher pink salmon productivity for northern
stocks and lower productivity for southern stocks with the north/south break
occurring near the southern end of Southeast Alaska
(\textasciitilde{}54$^{\circ}$N), which closely matches the break point we
identified for the spring bloom initiation date
(\textasciitilde{}56$^{\circ}$N).  This consistency in the latitude of the
north/south break point across studies of different environmental variables
further supports the idea that the opposite effects are driven by differences in
ocean conditions between the northern and southern domains.

It is not clear why a lower late summer chl-a concentration would be associated
with greater salmon productivity, but a possible explanation relates to top-down
grazing pressure. Our chl-a variable represents variability in the phytoplankton
standing stock, which can be influenced by both phytoplankton productivity and
top-down grazing pressure.  Zooplankton grazers in the Northeast Pacific at
least partially control the standing stock of phytoplankton \citep{Strom2001,
Frost1987} and, in turn, are an important food source for juvenile pink salmon
\citep{Boldt2003a, Armstrong2005a, Beauchamp2007a}. If pink salmon do not
significantly control zooplankton abundance, then lower phytoplankton biomass
could represent higher zooplankton abundances available to support higher growth
and survival of pink salmon. This hypothesis is supported by observations in the
Strait of Georgia, British Columbia indicating that peak zooplankton biomass (in
particular \emph{Neocalanus} spp.) often coincides with phytoplankton biomass
minima \citep{Bornhold2000}. Furthermore, grazing by zooplankton may also
partially explain the weak positive effect of spring chl-a on northern stock
productivity and the negative effect on southern stock productivity if the
seasonal timing of peak zooplankton biomass follows a north-south gradient with
later peaks in more northern areas. For example, in the north the spring chl-a
variable may index phytoplankton biomass prior to increases in zooplankton
biomass, while in the south zooplankton biomass may have already started to
increase by the end of May \citep{Mackas2012}.

Mortality of pink salmon in the marine life phase is thought to primarily occur
in coastal environments during the first summer in the ocean \citep{Farley2007a,
Parker1968a, Wertheimer2007a}. In particular, research on pink salmon in
Southeast Alaska and Prince William Sound, AK has indicated that a considerable
portion of marine mortality occurs in inside waters prior to salmon migrating to
the Gulf of Alaska \citep{Orsi2013, Farley2007a}. While the majority of these
coastal environments were indexed by our satellite-derived estimates for the
spring bloom initiation date and chl-a averages (Fig. \ref{fig:bloom:1}), there
were a few areas that were poorly covered (e.g., inside Southeast Alaska and the
inner coast of Vancouver Island) due to missing satellite data. For stocks in
these areas, we assumed that oceanographic conditions on the outer coast were
representative of conditions experienced by juvenile salmon during their first
few months in the ocean. This assumption is supported by the correspondence of
our estimates of the spring bloom initiation date and several studies that
estimated the spring bloom start date using \emph{in situ} observations. For
example, the average estimate for the spring bloom initiation date for the inner
SEAK pink salmon stock group over the years 1998-2009 was the first week in
April, which matches \emph{in situ} observations for the bloom start date in
Auke Bay, AK \citep{Ziemann1991}. Likewise, the average bloom start date for the
southern BC stock was the second to third week in March, which matches \emph{in
situ} observations from the inner coast of Vancouver Island
\citep{Chittenden2010a}.  Because of this coherence between the satellite
estimates and \emph{in situ} observations, we believe our assumption about
outside waters being representative of coastal environments is valid for our
study region.

Although we focused on pink salmon productivity, the indicators of phytoplankton
dynamics we investigated may also be important factors controlling covariation
in other salmon species. For instance, productivity indices of sockeye salmon,
chum salmon, and coho salmon (\emph{O. kisutch}) also tend to covary at a
regional spatial scale with sockeye and coho salmon having the most similar
spatial scales of covariation to that of pink salmon \citep{Mueter2002b,
Teo2009a, Peterman2012}. In particular, our results may be most applicable to
sockeye salmon, which tend to feed at a similar trophic level as pink salmon
\citep{Johnson2009a}.

Our results suggest a link between the spring bloom initiation date and pink
salmon productivity; however, further research is needed to understand the
mechanisms underlying this relationship. For example, comparing the potential
match/mismatch between salmon out-migration timing and initiation of
phytoplankton and zooplankton blooms could help clarify how phenologies are
coupled across trophic levels. Similarly, research into the relationships
between primary productivity and salmon productivity, as opposed to
phytoplankton biomass, would help in understanding the importance of the optimal
stability window. Our spatial correlation results indicate that such research
should focus on regional-scale processes and avoid correlating large-scale
climate indices with salmon productivity \citep{Mueter2002a, Peterman2012}.

In conclusion, our results suggest that the phenology of bottom-up biological
oceanographic processes are more important for higher trophic level species such
as pink salmon than the standing stock of phytoplankton. This conclusion has
important implications as the climate warms. It is generally recognized that a
warming climate will lead to an earlier onset of spring conditions, including
earlier timing of peak zooplankton biomass and outmigration of pink salmon
\citep{Parmesan2003, Taylor2008a, Mackas2012}. Phenological mismatches could
occur across trophic levels if separate processes do not change in synchrony
\citep{Edwards2004a}, potentially leading to northward latitudinal shifts in
peak pink salmon productivity.



\section{Acknowledgments}

We are grateful to the many biologists from the Alaska Department of Fish and
Game and Fisheries and Oceans Canada who collected and provided us with the
numerous salmon time series analyzed here. We also thank Nathan Mantua and Todd
Mitchell for helpful discussions as well as Ed Farley and an anonymous reviewer
for their useful comments on our manuscript. Funding for this research was
provided by Simon Fraser University and a grant from the Canada Research Chairs
Program to R.M.  Peterman.



\section{Tables}

\begin{landscape}
% Table captions are located in the R files that produce the tables
% Pink salmon data table
% 2016-10-31

{\small
  \libertineLF

  \begin{ThreePartTable}
    \begin{TableNotes}
     {\footnotesize
        \item[a] Sources of data by stock number: 1: Pieter Van Will, Fisheries
          and Oceans Canada (DFO), Port Hardy, BC; 2-8: David Peacock, Fisheries
          and Oceans Canada (DFO), Prince Rupert, BC; 9-11: Steve Heinl, Alaska
          Department of Fish and Game (ADFG), Ketchikan, AK and
          \citet{Piston2011a}; 12: Steve Moffitt, Alaska Department of Fish and
          Game (ADFG), Cordova, AK; 13-15: Ted Otis, Alaska Department of Fish
          and Game (ADFG), Homer, AK; 16-20: Matt Foster, Alaska Department of
          Fish and Game (ADFG), Kodiak, AK; 21-25: Charles Russell, Alaska
          Department of Fish and Game (ADFG), Kodiak, AK; 26-27: Matt Foster,
          Alaska Department of Fish and Game (ADFG), Kodiak, AK.
        \item[b] Includes statistical areas 11-16; Excludes Fraser River
        \item[c] Includes districts 101-108
        \item[d] Includes districts 109-112, 114, 115
        \item[e] Includes district 113
        \item[f] Sum of Humpy Creek and Seldovia Bay data sets
        \item[g] Sum of Port Chatham, Port Dick, Rocky River, Windy Creek, and South Nuka data sets
        \item[h] Sum of Bruin River, Sunday Creek, and Brown's Peak Creek data sets
        \item[i] Sum of Southeast and Southcentral districts data sets
      }
    \end{TableNotes}

    \begin{longtable}{lllllllll}
      \caption[Summary of pink salmon stock-recruit data sets.]{Summary of pink
        salmon stock-recruit data sets. Brood years indicate the temporal range of
        spawning years; N is the number of non-missing years within that range; R
        is the annual average recruitment (catch plus escapement) in millions
        across all brood years; S is the average number of spawners in millions
        across all brood years; $\alpha$ and $\beta$ are maximum likelihood
        estimates of the intercept (i.e., stock productivity at low spawner
        abundances) and slope (i.e., density-dependent effect), respectively, of
        the single-stock Ricker models (eq. \ref{eq:bloom:1}).} \\ 
      \hline
      % latex table generated in R 3.3.1 by xtable 1.8-2 package
% Mon Oct 31 13:03:25 2016
Stock \#\textsuperscript{a} & Jurisdiction & Stock & Brood years & N & R & S & $\alpha$ & $\beta$ \\ 
  \hline 
\endfirsthead 
Table \thetable\ Continued \\ 
\hline 
Stock \#\textsuperscript{a} & Jurisdiction & Stock & Brood years & N & R & S & $\alpha$ & $\beta$ \\
\hline 
\endhead 
\hline 
{\footnotesize Continued on next page} 
\endfoot 
\endlastfoot 
1 & BC & Southern BC\textsuperscript{b} & 1953-2008 &  56 & 2.02 & 1.09 & 0.90 & -0.49 \\ 
  2 & BC & Statistical Area 9 & 1980-2008 &  29 & 0.46 & 0.41 & 0.13 & -0.50 \\ 
  3 & BC & Statistical Area 8 & 1980-2008 &  29 & 3.28 & 2.36 & 0.36 & -0.19 \\ 
  4 & BC & Statistical Area 7 & 1980-2008 &  29 & 0.53 & 0.38 & 0.50 & -1.05 \\ 
  5 & BC & Statistical Area 6 & 1980-2008 &  29 & 2.76 & 1.33 & 0.45 & -0.01 \\ 
  6 & BC & Statistical Area 5 & 1982-2008 &  27 & 0.45 & 0.31 & 0.79 & -1.28 \\ 
  7 & BC & Statistical Area 4 & 1982-2008 &  27 & 5.93 & 2.57 & 1.19 & -0.29 \\ 
  8 & BC & Statistical Area 3 & 1982-2008 &  27 & 1.50 & 0.80 & 1.31 & -0.78 \\ 
  9 & AK & Southern SEAK\textsuperscript{c} & 1960-2008 &  49 & 33.31 & 13.47 & 1.15 & -0.02 \\ 
  10 & AK & Northern SEAK Outside\textsuperscript{d} & 1960-2008 &  49 & 3.98 & 2.37 & 1.31 & -0.17 \\ 
  11 & AK & Northern SEAK Inside\textsuperscript{e} & 1960-2008 &  49 & 16.48 & 7.62 & 1.21 & -0.05 \\ 
  12 & AK & Prince William Sound & 1960-2009 &  50 & 10.12 & 4.40 & 0.95 & -0.06 \\ 
  13 & AK & Southern Cook Inlet\textsuperscript{f} & 1976-2009 &  34 & 0.13 & 0.09 & 0.98 & -10.67 \\ 
  14 & AK & Outer Cook Inlet\textsuperscript{g} & 1976-2009 &  34 & 0.53 & 0.23 & 1.36 & -2.30 \\ 
  15 & AK & Kamishak District\textsuperscript{h} & 1976-2009 &  34 & 0.38 & 0.32 & 1.15 & -2.82 \\ 
  16 & AK & Afognak District & 1978-2009 &  32 & 1.85 & 0.75 & 2.69 & -2.38 \\ 
  17 & AK & Westside Kodiak & 1978-2009 &  32 & 10.00 & 4.13 & 1.20 & -0.06 \\ 
  18 & AK & Alitak District & 1978-2009 &  32 & 3.14 & 1.58 & 1.65 & -0.61 \\ 
  19 & AK & Eastside Kodiak & 1978-2009 &  32 & 4.47 & 2.16 & 0.68 & -0.02 \\ 
  20 & AK & Mainland Kodiak & 1978-2009 &  32 & 1.84 & 1.36 & 1.16 & -0.77 \\ 
  21 & AK & Chignik Bay & 1962-2009 &  43 & 0.12 & 0.03 & 1.15 & -6.49 \\ 
  22 & AK & Central Chignik & 1962-2009 &  48 & 0.32 & 0.17 & 1.12 & -1.68 \\ 
  23 & AK & Eastern Chignik & 1962-2009 &  48 & 0.64 & 0.50 & 0.73 & -0.98 \\ 
  24 & AK & Western Chignik & 1962-2009 &  48 & 0.48 & 0.16 & 1.57 & -2.97 \\ 
  25 & AK & Perryville & 1962-2009 &  48 & 0.32 & 0.15 & 1.37 & -8.73 \\ 
  26 & AK & AK Peninsula\textsuperscript{i} & 1962-2009 &  48 & 4.94 & 1.76 & 1.55 & -0.29 \\ 
  27 & AK & Southwest Unimak & 1962-2009 &  48 & 2.05 & 0.83 & 1.37 & -0.64 \\ 
  
      \hline
      \insertTableNotes
      \label{tab:bloom:1}
    \end{longtable}

  \end{ThreePartTable}
}


% \noindent \textsuperscript{a} Sources of data by stock number: 1: Pieter Van
% Will, Fisheries and Oceans Canada (DFO), Port Hardy, BC; 2-8: David Peacock,
% Fisheries and Oceans Canada (DFO), Prince Rupert, BC; 9-11: Steve Heinl, Alaska
% Department of Fish and Game (ADFG), Ketchikan, AK and \citet{Piston2011a}; 12:
% Steve Moffitt, Alaska Department of Fish and Game (ADFG), Cordova, AK; 13-15:
% Ted Otis, Alaska Department of Fish and Game (ADFG), Homer, AK; 16-20: Matt
% Foster, Alaska Department of Fish and Game (ADFG), Kodiak, AK; 21-25: Charles
% Russell, Alaska Department of Fish and Game (ADFG), Kodiak, AK; 26-27: Matt
% Foster, Alaska Department of Fish and Game (ADFG), Kodiak, AK.

% \noindent \textsuperscript{b} Includes statistical areas 11-16; Excludes Fraser River

% \noindent \textsuperscript{c} Includes districts 101-108

% \noindent \textsuperscript{d} Includes districts 109-112, 114, 115

% \noindent \textsuperscript{e} Includes district 113

% \noindent \textsuperscript{f} Sum of Humpy Creek and Seldovia Bay data sets

% \noindent \textsuperscript{g} Sum of Port Chatham, Port Dick, Rocky River, Windy Creek, and South Nuka data sets

% \noindent \textsuperscript{h} Sum of Bruin River, Sunday Creek, and Brown's Peak Creek data sets

% \noindent \textsuperscript{i} Sum of Southeast and Southcentral districts data sets


\noindent \textsuperscript{a} Sources of data by stock number: 1: Pieter Van
Will, Fisheries and Oceans Canada (DFO), Port Hardy, BC; 2-8: David Peacock,
Fisheries and Oceans Canada (DFO), Prince Rupert, BC; 9-11: Steve Heinl, Alaska
Department of Fish and Game (ADFG), Ketchikan, AK and \citet{Piston2011a}; 12:
Steve Moffitt, Alaska Department of Fish and Game (ADFG), Cordova, AK; 13-15:
Ted Otis, Alaska Department of Fish and Game (ADFG), Homer, AK; 16-20: Matt
Foster, Alaska Department of Fish and Game (ADFG), Kodiak, AK; 21-25: Charles
Russell, Alaska Department of Fish and Game (ADFG), Kodiak, AK; 26-27: Matt
Foster, Alaska Department of Fish and Game (ADFG), Kodiak, AK.

\noindent \textsuperscript{b} Includes statistical areas 11-16; Excludes Fraser
River

\noindent \textsuperscript{c} Includes districts 101-108

\noindent \textsuperscript{d} Includes districts 109-112, 114, 115

\noindent \textsuperscript{e} Includes district 113

\noindent \textsuperscript{f} Sum of Humpy Creek and Seldovia Bay data sets

\noindent \textsuperscript{g} Sum of Port Chatham, Port Dick, Rocky River, Windy
Creek, and South Nuka data sets

\noindent \textsuperscript{h} Sum of Bruin River, Sunday Creek, and Brown's Peak
Creek data sets

\noindent \textsuperscript{i} Sum of Southeast and Southcentral districts data
sets

\newpage



% LME Coefficient Table
% 2016-10-31

{\small
  \libertineLF

  \begin{ThreePartTable}
    \begin{TableNotes}
      {\footnotesize
        \item[*] Significantly different from zero at P \textless{} 0.05. 
        \item[**] Significantly different from zero at P \textless{} 0.01. 
      }
    \end{TableNotes}

    \begin{longtable}{llllllllll}
      \caption[Summary of the best-fit multi-stock Ricker model
        coefficients.]{Summary of the best-fit multi-stock Ricker model
        coefficients (eq. \ref{eq:bloom:3}). The Subset column identifies the
        stocks included in the hierarchical model (North = stocks 1-9, South =
        stocks 10-27, and All = stocks 1-27). The Stocks column indicates the
        number of stocks used to fit the model; N is the total number of data
        points across all stocks used to fit the model; $\alpha$ is the
        intercept representing average productivity of pink salmon at low
        spawner abundance (fixed effect; see eq. \ref{eq:bloom:3}); $\gamma$ is
        the fixed effect corresponding to either the bloom initiation or mean
        chl-a concentration (see eq. \ref{eq:bloom:3}); and SE is the standard
        error for the fixed-effect coefficients. AIC\textsubscript{C} is the
        Akaike Information Criterion, corrected for small sample size.} \\ 
      \hline
      % latex table generated in R 3.3.1 by xtable 1.8-2 package
% Mon Oct 31 09:25:38 2016
Subset & Covariate & Stocks & N & $\alpha$ & SE\textsubscript{$\alpha$} & $\gamma$ & SE\textsubscript{$\gamma$} & AIC\textsubscript{C} & $\Delta$AIC\textsubscript{C} \\ 
  \hline
North & Null & 18 & 232.00 & 1.25** & 0.11 &  &  & 566.0 & 5.6 \\ 
   & Bloom Initiation & 18 & 232.00 & 1.25** & 0.11 & -0.12** & 0.04 & 560.3 & 0.0 \\ 
   & Mean Chl-a (Apr-May) & 18 & 232.00 & 1.07** & 0.16 & 0.51 & 0.32 & 565.7 & 5.4 \\ 
  South & Null & 9 & 108.00 & 0.71** & 0.15 &  &  & 266.6 & 13.8 \\ 
   & Bloom Initiation & 9 & 108.00 & 0.69** & 0.15 & 0.12** & 0.03 & 252.8 & 0.0 \\ 
   & Mean Chl-a (Apr-May) & 9 & 108.00 & 1.16** & 0.28 & -0.87* & 0.45 & 265.4 & 12.5 \\ 
  All & Null & 27 & 340.00 & 1.05** & 0.10 &  &  & 834.8 & 22.1 \\ 
   & Bloom Initiation & 27 & 340.00 &  &  &  &  & 812.7 & 0.0 \\ 
   & Mean Chl-a (Apr-May) & 27 & 340.00 &  &  &  &  & 830.7 & 17.9 \\ 
   & Mean Chl-a (July-Sept) & 27 & 340.00 & 1.65** & 0.18 & -1.69** & 0.43 & 820.3 & 7.5 \\ 
  
      \hline
      \insertTableNotes
      \label{tab:bloom:2}
    \end{longtable}

  \end{ThreePartTable}
}

% \noindent * Significantly different from zero at P \textless{} 0.05; **
% significant at P \textless{} 0.01.

\noindent * Significantly different from zero at P \textless{} 0.05; **
significant at P \textless{} 0.01.

\end{landscape}
\newpage



\section{Figures}

\begin{figure}[htbp]
  \centering \includegraphics[scale=0.6]{3_springbloom/figures/map.pdf}
  \caption[Study area indicating chl-a grid cells and ocean entry locations for
    pink salmon stocks.]{Study area indicating the grid cells used to compute the
    bloom initiation date and mean chl-a concentrations (green squares) and the
    locations of ocean entry points for the pink salmon stocks (solid black
    triangles). Solid line indicates break point identified by the intervention
    model (see text) between northern and southern stock groupings.}
  \label{fig:bloom:1}
\end{figure}

\begin{figure}[htbp]
  \centering \includegraphics[scale=0.9]{3_springbloom/figures/pinks_bloom_ncf.pdf}
  \caption[Correlograms of correlations among salmon productivity
    indices and spring bloom initiation date.]{Correlograms (pairwise
    correlations as a function of distance between location of data pairs) of
    correlations among salmon productivity indices
    across all brood years (top panel), salmon productivity for brood years
    1997-2009 (middle panel), and spring bloom initiation date (bottom panel).
    Solid curves represent the estimated smooth nonparametric covariance
    function with 95\% confidence band shown as the dashed lines. Solid vertical
    lines indicate the 50\% correlation scale (D50).}
  \label{fig:bloom:2}
\end{figure}

\begin{figure}[htbp]
  \centering \includegraphics[scale=0.9]{3_springbloom/figures/bloom_pink_ncf_summary.pdf}
  \caption[Comparison of the estimated 50\% correlation scale
    and y-intercept from the nonparametric covariance functions.]{Comparison of
    the estimated 50\% correlation scale (D50; 
    top panel) and y-intercept (CZD; bottom panel) for the nonparametric
    covariance functions fit to pink salmon residuals using all brood years of
    data (``Pink all'' from Fig. 2a), pink salmon residuals using only brood
    years 1997-2009 (``Pink short'' from Fig.  2b), and initiation date for the
    spring bloom (from Fig. 2c). Dots indicate point estimates for each metric
    and vertical lines give 95\% confidence intervals.}
  \label{fig:bloom:3}
\end{figure}

\begin{figure}[htbp]
  \centering \includegraphics[scale=0.8]{3_springbloom/figures/chl_ncf.pdf}
  \caption[Correlograms of correlations among grid cells for the monthly
    mean chl-a concentrations.]{Correlograms (pairwise correlations as a
    function of distance between location of data pairs) of correlations among
    grid cells for the monthly mean chl-a concentrations. Solid curves represent
    the estimated smooth nonparametric covariance function with the 95\%
    confidence band shown as the dashed lines. Solid vertical lines indicate the
    50\% correlation scale (D50).}
  \label{fig:bloom:4}
\end{figure}

\begin{figure}[htbp]
  \centering \includegraphics[scale=0.9]{3_springbloom/figures/chl_ncf_50.pdf}
  \caption[The 50\% correlation scale for chl-a concentration by month.]{The
    50\% correlation scale (D50) for chl-a concentration by month.  Solid
    vertical lines indicate 95\% confidence intervals for each month.
    Dotted horizontal line indicates the 50\% correlation scale for pink salmon
    using brood years 1997-2009; the grey shaded region indicates the 95\%
    confidence interval for that pink salmon 50\% correlation scale.}
  \label{fig:bloom:5}
\end{figure}

\begin{figure}[htbp]
  \centering \includegraphics[scale=0.7]{3_springbloom/figures/lme_coef.pdf}
  \caption[Estimates of the effects on salmon productivity of spring
    bloom initiation date and chl-a concentrations.]{Estimates of the effects on
    pink salmon productivity of spring bloom initiation date (left panels), mean
    April-May chl-a concentration (middle panels), and mean July-September chl-a
    concentration (right panels) from the single-stock and best-fit multi-stock
    models. In panels \emph{a-c} the ordinate gives the stock number, as defined
    in Table \ref{tab:bloom:1}, solid circles represent the estimated effect for
    either chl-a or the bloom initiation from the single stock models (i.e.,
    \(\gamma\) in eq.  \ref{eq:bloom:2}), and the solid vertical line gives the
    estimated region-wide effect for either the bloom initiation or chl-a from a
    multi-stock model (i.e., \(\gamma\) in eq.  \ref{eq:bloom:3}). Based on
    results from the single-stock analyses, separate multi-stock models were fit
    to northern and southern stocks for the spring bloom and April-May chl-a
    covariates, which are separated by a solid horizontal line. Bottom panels
    \emph{d-f} show histograms of the spring bloom and chl-a effects based on
    single-stock models (eq. \ref{eq:bloom:2}) and estimated probability density
    functions (smooth curves) of the single-stock model coefficients for
    northern and southern stocks.}
  \label{fig:bloom:6}
\end{figure}


\newpage
\section{Supporting materials}

\subsection{Comparison of SeaWiFS and MODIS chlorophyll-a data products}
\label{supp:bloom:A}

Lengthy periods of missing data in 2008-2010 for the SeaWiFS chl-a data set made
these years of data unsuitable for calculating an initiation date of the spring
bloom (Fig. \ref{fig:bloom:s1}). Therefore, we investigated the feasibility of
combining the SeaWiFS and MODIS chl-a data products. We used Pearson correlation
coefficients between the two time series of log\textsubscript{10} transformed
data, root mean squared log\textsubscript{10} error (RMSE), and
log\textsubscript{10} bias (mean of the absolute value of log\textsubscript{10}
transformed MODIS chl-a values minus SeaWiFS values) to quantify the covariation
and differences between the SeaWiFS and MODIS data sets for the first five
complete years of overlap, 2003-2007 \citep{Gregg2004, OReilly2000a, Zhang2006}
. We performed two primary comparisons. First, to assess the similarities
between the data sets across the entire study region and time period, we
calculated the average of each of the three metrics (i.e., correlation
coefficients, RMSE, and log\textsubscript{10} bias) for all grid cells and
years. Second, to assess differences among years, we calculated the averages of
the RMSE, as well as log\textsubscript{10} bias metrics for all grid cells by
year.

The average correlation between the MODIS and SeaWiFS chl-a data sets across
grid cells was 0.87, with correlations for individual grid cells ranging from
0.69 to 0.95. The RMSE of the differences between the SeaWiFS and MODIS chl-a
data sets over all years and grid cells was 0.16 and annually it ranged from
0.09 in 2007 to 0.22 in 2006 (Fig.  \ref{fig:bloom:s2}).  These RMSE values are
slightly smaller than RMSE values of the differences between satellite chl-a
estimates and field sampled chl-a data (RMSE of 0.22; \citep{OReilly2000a}),
indicating our RMSE values are within the range of background noise of the
algorithm used to produced the chl-a estimates. The log\textsubscript{10} bias
of MODIS data compared to SeaWiFS data over all grid cells and years was 0.012
and annually ranged from 0.04 in 2004 to -0.01 in 2007 (Supporting materials
Fig. \ref{fig:bloom:s1}) with 79\% of absolute differences being less than 0.1
and 90\% of differences being less than 0.2. These log\textsubscript{10} bias
values are generally small differences given the estimated chl-a values
(Supporting materials Fig.  \ref{fig:bloom:s5}).  In addition, the
log\textsubscript{10} bias values indicate that the MODIS values are slightly
higher than the SeaWiFS values at this temporal composite and spatial resolution
for most years.

Our comparison of the SeaWiFS and MODIS chl-a data products concurs with values
reported by other researchers over the same time and a similar region
\citep{Waite2013}. Given the strong covariation, low RMSE, and small
log\textsubscript{10} bias between chl-a data products, we deemed it appropriate
to concatenate the SeaWiFS and MODIS data products for our study region without
further data processing.


\subsection{Hatchery pink salmon marine survival}
\label{supp:bloom:B}

To further test the effects of the spring bloom initiation date and chl-a
concentrations on pink salmon dynamics, we compared the three chl-a metrics to
pink salmon marine survival rates for three hatchery stocks located around the
Gulf of Alaska (Armin F. Koernig, Kitoi Bay, and Port Armstrong). Marine
survival rates for each hatchery stock were estimated for release years
1998-2010 by dividing the total adult pink salmon returns resulting from
juveniles released in year \(t\) by the total number of juvenile pink salmon
released into marine waters in year \(t\). To estimate the association between
each of the three chl-a metrics and marine survival, we calculated Pearson
correlation coefficients between marine survival rates and the spring bloom
initiation date, average April-May chl-a concentration, and average
July-September chl-a concentration. Like the analysis in the main text, each
chl-a metric and the spring bloom initiation date were averaged over all grid
cells within 250 km of the ocean release location for each hatchery.

Marine survival rates for all three hatcheries were negatively correlated with
the spring bloom initiation date (Supporting materials Fig.
\ref{fig:bloom:s4}), which corresponds with our hierarchical model results that
indicated a negative region-wide effect of the spring bloom initiation on salmon
productivity for northern stocks. The correlations between marine survival rates
and spring chl-a concentration were mostly positive with two positive
correlations and a negative correlation (Supporting materials Fig.
\ref{fig:bloom:s4}), which corresponds with the results from our single-stock
and multi-stock models for northern stocks that indicated a weak positive
region-wide effect. The correlations for the late summer chl-a concentration
diverged from our single-stock and multi-stock model results with two positive
correlations and a negative correlation (Supporting materials Fig.
\ref{fig:bloom:s4}), however, the correlations were weak for all three
hatcheries (less than 0.35 for all hatcheries). On average, the absolute values
of correlations between marine survival and the spring bloom were higher
(average correlation = -0.34) than for the spring chl-a concentration (average
correlation = 0.27) and the late summer chl-a concentration (average correlation
= 0.23).

Results of the hatchery marine survival analysis broadly agreed with the results
from the single-stock and multi-stock models in the main text.  In particular,
the hatchery results support our conclusions that (1) an early spring bloom
timing is associated with increased productivity for northern pink salmon
stocks, and (2) the spring bloom initiation date is more strongly associated
with pink salmon productivity than phytoplankton biomass.


\subsection{Supplemental Figures}

\begin{figure}[htbp]
  \centering \includegraphics[scale=0.9]{3_springbloom/figures/S_chl_perc_na.pdf}
  \caption[Percent of data missing for SeaWifFS and MODIS chl-a data.]{Percent 
    of all grid cells with missing chl-a data by year (see Fig. 1
    of the main text) for SeaWiFS (left panel) and MODIS (right panel) chl-a
    data. Green line with closed circles indicates 1-day composites, blue line
    with filled squares indicates 5-day composites, and red line with filled
    triangles indicates 8-day composites.}
  \label{fig:bloom:s1}
\end{figure}


\begin{figure}[htbp]
  \centering
  \includegraphics[scale=0.9]{3_springbloom/figures/S_compare_rmse_bias.pdf}
  \caption[Annual comparisons of SeaWiFS and MODIS data sets.]{Annual 
    values for the root mean squared log\textsubscript{10} error
   (RMSE; top panel) and log\textsubscript{10} bias (bottom panel) calculated
    as MODIS minus corresponding SeaWiFS chl-a value.}
  \label{fig:bloom:s2}
\end{figure}

\begin{figure}[htbp]
  \centering
  \includegraphics[scale=0.9]{3_springbloom/figures/S_lm_hist.pdf}
  \caption[Histograms of the estimated $\alpha$ and $\beta$ coefficients from the
    single-stock models.]{Histograms of the estimated \(\alpha\) and \(\beta\)
    coefficients for the single stock models (eq. \ref{eq:bloom:2}) and
    estimated probability density functions (smooth solid black lines) for bloom
    initiation date (top row), average spring chl-a concentrations (middle row),
    and average late summer chl-a concentrations (bottom row).}
  \label{fig:bloom:s3}
\end{figure}

\begin{figure}[htbp]
  \centering
  \includegraphics[scale=0.9]{3_springbloom/figures/S_hatchery_cor.pdf}
  \caption[Correlation matrix between each of the hatchery marine survival time 
    series and the three chl-a metrics ]{Correlation matrix showing the Pearson
    correlation coefficients between each of the hatchery marine survival time
    series and the three chl-a metrics (* indicates significantly different from
    zero at P \textless{} 0.05).}
  \label{fig:bloom:s4}
\end{figure}

\begin{figure}[htbp]
  \centering
  \includegraphics[scale=0.9]{3_springbloom/figures/S_chl_month_avg.pdf}
  \caption[Average monthly log\textsubscript{10} transformed chlorophyll-a
    values across all grid cells and years.]{Average monthly
    log\textsubscript{10} transformed chlorophyll-a values across all grid cells
    and years.}
  \label{fig:bloom:s5}
\end{figure}



